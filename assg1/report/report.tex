\documentclass[12pt,a4paper]{article}
\usepackage[T1]{fontenc}
\usepackage[utf8]{inputenc}
\usepackage{polski}
\usepackage{minted}
\usepackage{pgf}
\usepackage{enumitem}
\usepackage{amsmath}
\usepackage[tableposition=top]{caption}

\renewcommand\refname{Bibliografia}

\begin{document}

\begin{titlepage}
  \centering
  \vspace*{2cm}
  {\scshape\large Politechnika Wrocławska \par}
  \vspace{1cm}
  {\itshape\large Sztuczna Inteligencja i Inżynieria Wiedzy\par}
  \vspace{1.5cm}
  {\LARGE\bfseries Implementacja i badanie algorytmów ewolucyjnych na bazie problemu komiwojażera\par}
  \vspace{1cm}
  {\Large Sprawozdanie nr 1\par}
  \vfill
  {\large Karol Belina, 242499\par grupa śr. 11:15}
  \vfill
  {\large \today\par}
\end{titlepage}

\tableofcontents

\section*{Streszczenie}

W sprawozdaniu opisano

\section{Wstęp}

Problem komiwojażera (ang. Travelling salesperson problem) jest zagadnieniem optymalizacyjnym polegającym na znalezieniu mimalnego cyklu w grafie ważonym obejmującego wszystkie wierzchołki. Dane wejściowe to lista wierzchołków wraz z ich współrzędnymi. Do rozwiązania tego problemu można użyć różnych metod m.in. algorytmu zachłannego, algorytmu wspinaczkowego (ang. Hill climbing algorithm) lub algorytmu genetycznego.

\section{Implementacja}

\subsection{Format plików}

Dane wejściowe pochodzą z biblioteki TSPLIB 95 \cite{tsplib95}. W programie zaimplementowano uproszczony parser plików TSPLIB skupiający się jedynie na Symetrycznym problemie komiwojażera w przestrzeni dwuwymiarowej. Parser jest w stanie wykryć dwa typy wag: \verb|EUC_2D| oraz \verb|GEO| i użyć odpowiedniej funkcji odległości do obliczenia macierzy odległości. Macierz odległości zwalnia program z potrzeby przechowywania w pamięci współrzędnych. Dodatkowo, wszystkie odległości liczone są tylko raz.

\subsection{Model}

Podczas tworzenia modelu szczególną uwagę zwrócono na jego modularność. Zauważono, że do zdefiniowania problemu wymagana jest jedynie wiedza o typie rozwiązania i miary jakości tego rozwiązania, jak również definicja funkcji przystosowania mapująca rozwiązania na miarę jakości. Wszystkie te wymagania zawarto w implementacji jako jedna wspólna cecha \texttt{Problem} (ang. trait).
\begin{listing}[H]
  \begin{minted}[breaklines,linenos,frame=single]{rust}
pub trait Problem {
  type Solution: Clone;
  type Measure: Clone;

  fn fitness(&self, solution: &Self::Solution) -> Self::Measure;
}
  \end{minted}
  \caption{Definicja cechy \texttt{Problem}.}
\end{listing}
Algorytm genetyczny typowany jest powyższą cechą z użyciem mechanizmu generyczności. Modularność widoczna jest również w operatorach genetycznych, które implementują wspólne cechy i są wymienialne w kotekście algorytmu genetycznego.
\begin{listing}[H]
  \begin{minted}[breaklines,linenos,frame=single]{rust}
fn select<'a>(
  &self, population: &'a Vec<Individual<Self::Problem>>
) -> &'a Individual<Self::Problem>;
  \end{minted}
  \caption{Definicja metody \texttt{select} przykładowej cechy \texttt{Select}, czyli jednego z czterech operatorów genetycznych. Metoda z wektora osobników wybiera i zwraca referencję na pojedynczego osobnika.}
\end{listing}

\section{Badania i eksperymenty}

Algorytm dla tych samych parametrów zostanie uruchomiony 10 razy.

\subsection{Eksperyment nr 1}

\begin{description}[align=left,leftmargin=2.5cm,style=multiline]
  \item [Cel] Zbadanie wpływu prawdopodobieństwa mutacji na rezultaty działania algorytmu ewolucyjnego.
  \item [Założenia] Wszystkie testy przeprowadzane na danych pochodzą z pliku \texttt{kroA200.tsp}. Stałe wartości wszystkich parametrów poza $P_m$ wynoszą: $P_x = 0.7$, $\text{gen} = 250$, $\text{pop\_size} = 10000$, $\text{tour\_size} = 5$. Inicjalizacja jest losowa. Dla selekcji, krzyżowania i mutacji operatorami są kolejno: \textit{turniej}, \textit{OX}, oraz \textit{swap}.
  \item [Przebieg] Algorytm zostanie uruchomiony dla różnych wartości $P_m$, rozpoczynając od $0$, zwiększając o względnie małe wartości aż do osiągnięcia wartości $1$.
\end{description}
\begin{figure}[H]
  \begin{center}
    %% Creator: Matplotlib, PGF backend
%%
%% To include the figure in your LaTeX document, write
%%   \input{<filename>.pgf}
%%
%% Make sure the required packages are loaded in your preamble
%%   \usepackage{pgf}
%%
%% and, on pdftex
%%   \usepackage[utf8]{inputenc}\DeclareUnicodeCharacter{2212}{-}
%%
%% or, on luatex and xetex
%%   \usepackage{unicode-math}
%%
%% Figures using additional raster images can only be included by \input if
%% they are in the same directory as the main LaTeX file. For loading figures
%% from other directories you can use the `import` package
%%   \usepackage{import}
%%
%% and then include the figures with
%%   \import{<path to file>}{<filename>.pgf}
%%
%% Matplotlib used the following preamble
%%
\begingroup%
\makeatletter%
\begin{pgfpicture}%
\pgfpathrectangle{\pgfpointorigin}{\pgfqpoint{5.397490in}{3.000000in}}%
\pgfusepath{use as bounding box, clip}%
\begin{pgfscope}%
\pgfsetbuttcap%
\pgfsetmiterjoin%
\definecolor{currentfill}{rgb}{1.000000,1.000000,1.000000}%
\pgfsetfillcolor{currentfill}%
\pgfsetlinewidth{0.000000pt}%
\definecolor{currentstroke}{rgb}{1.000000,1.000000,1.000000}%
\pgfsetstrokecolor{currentstroke}%
\pgfsetdash{}{0pt}%
\pgfpathmoveto{\pgfqpoint{0.000000in}{0.000000in}}%
\pgfpathlineto{\pgfqpoint{5.397490in}{0.000000in}}%
\pgfpathlineto{\pgfqpoint{5.397490in}{3.000000in}}%
\pgfpathlineto{\pgfqpoint{0.000000in}{3.000000in}}%
\pgfpathclose%
\pgfusepath{fill}%
\end{pgfscope}%
\begin{pgfscope}%
\pgfsetbuttcap%
\pgfsetmiterjoin%
\definecolor{currentfill}{rgb}{1.000000,1.000000,1.000000}%
\pgfsetfillcolor{currentfill}%
\pgfsetlinewidth{0.000000pt}%
\definecolor{currentstroke}{rgb}{0.000000,0.000000,0.000000}%
\pgfsetstrokecolor{currentstroke}%
\pgfsetstrokeopacity{0.000000}%
\pgfsetdash{}{0pt}%
\pgfpathmoveto{\pgfqpoint{0.809624in}{0.450000in}}%
\pgfpathlineto{\pgfqpoint{4.587867in}{0.450000in}}%
\pgfpathlineto{\pgfqpoint{4.587867in}{2.940000in}}%
\pgfpathlineto{\pgfqpoint{0.809624in}{2.940000in}}%
\pgfpathclose%
\pgfusepath{fill}%
\end{pgfscope}%
\begin{pgfscope}%
\pgfsetbuttcap%
\pgfsetroundjoin%
\definecolor{currentfill}{rgb}{0.000000,0.000000,0.000000}%
\pgfsetfillcolor{currentfill}%
\pgfsetlinewidth{0.803000pt}%
\definecolor{currentstroke}{rgb}{0.000000,0.000000,0.000000}%
\pgfsetstrokecolor{currentstroke}%
\pgfsetdash{}{0pt}%
\pgfsys@defobject{currentmarker}{\pgfqpoint{0.000000in}{-0.048611in}}{\pgfqpoint{0.000000in}{0.000000in}}{%
\pgfpathmoveto{\pgfqpoint{0.000000in}{0.000000in}}%
\pgfpathlineto{\pgfqpoint{0.000000in}{-0.048611in}}%
\pgfusepath{stroke,fill}%
}%
\begin{pgfscope}%
\pgfsys@transformshift{0.809624in}{0.450000in}%
\pgfsys@useobject{currentmarker}{}%
\end{pgfscope}%
\end{pgfscope}%
\begin{pgfscope}%
\definecolor{textcolor}{rgb}{0.000000,0.000000,0.000000}%
\pgfsetstrokecolor{textcolor}%
\pgfsetfillcolor{textcolor}%
\pgftext[x=0.809624in,y=0.352778in,,top]{\color{textcolor}\rmfamily\fontsize{10.000000}{12.000000}\selectfont \(\displaystyle 0\)}%
\end{pgfscope}%
\begin{pgfscope}%
\pgfsetbuttcap%
\pgfsetroundjoin%
\definecolor{currentfill}{rgb}{0.000000,0.000000,0.000000}%
\pgfsetfillcolor{currentfill}%
\pgfsetlinewidth{0.803000pt}%
\definecolor{currentstroke}{rgb}{0.000000,0.000000,0.000000}%
\pgfsetstrokecolor{currentstroke}%
\pgfsetdash{}{0pt}%
\pgfsys@defobject{currentmarker}{\pgfqpoint{0.000000in}{-0.048611in}}{\pgfqpoint{0.000000in}{0.000000in}}{%
\pgfpathmoveto{\pgfqpoint{0.000000in}{0.000000in}}%
\pgfpathlineto{\pgfqpoint{0.000000in}{-0.048611in}}%
\pgfusepath{stroke,fill}%
}%
\begin{pgfscope}%
\pgfsys@transformshift{1.568307in}{0.450000in}%
\pgfsys@useobject{currentmarker}{}%
\end{pgfscope}%
\end{pgfscope}%
\begin{pgfscope}%
\definecolor{textcolor}{rgb}{0.000000,0.000000,0.000000}%
\pgfsetstrokecolor{textcolor}%
\pgfsetfillcolor{textcolor}%
\pgftext[x=1.568307in,y=0.352778in,,top]{\color{textcolor}\rmfamily\fontsize{10.000000}{12.000000}\selectfont \(\displaystyle 50\)}%
\end{pgfscope}%
\begin{pgfscope}%
\pgfsetbuttcap%
\pgfsetroundjoin%
\definecolor{currentfill}{rgb}{0.000000,0.000000,0.000000}%
\pgfsetfillcolor{currentfill}%
\pgfsetlinewidth{0.803000pt}%
\definecolor{currentstroke}{rgb}{0.000000,0.000000,0.000000}%
\pgfsetstrokecolor{currentstroke}%
\pgfsetdash{}{0pt}%
\pgfsys@defobject{currentmarker}{\pgfqpoint{0.000000in}{-0.048611in}}{\pgfqpoint{0.000000in}{0.000000in}}{%
\pgfpathmoveto{\pgfqpoint{0.000000in}{0.000000in}}%
\pgfpathlineto{\pgfqpoint{0.000000in}{-0.048611in}}%
\pgfusepath{stroke,fill}%
}%
\begin{pgfscope}%
\pgfsys@transformshift{2.326990in}{0.450000in}%
\pgfsys@useobject{currentmarker}{}%
\end{pgfscope}%
\end{pgfscope}%
\begin{pgfscope}%
\definecolor{textcolor}{rgb}{0.000000,0.000000,0.000000}%
\pgfsetstrokecolor{textcolor}%
\pgfsetfillcolor{textcolor}%
\pgftext[x=2.326990in,y=0.352778in,,top]{\color{textcolor}\rmfamily\fontsize{10.000000}{12.000000}\selectfont \(\displaystyle 100\)}%
\end{pgfscope}%
\begin{pgfscope}%
\pgfsetbuttcap%
\pgfsetroundjoin%
\definecolor{currentfill}{rgb}{0.000000,0.000000,0.000000}%
\pgfsetfillcolor{currentfill}%
\pgfsetlinewidth{0.803000pt}%
\definecolor{currentstroke}{rgb}{0.000000,0.000000,0.000000}%
\pgfsetstrokecolor{currentstroke}%
\pgfsetdash{}{0pt}%
\pgfsys@defobject{currentmarker}{\pgfqpoint{0.000000in}{-0.048611in}}{\pgfqpoint{0.000000in}{0.000000in}}{%
\pgfpathmoveto{\pgfqpoint{0.000000in}{0.000000in}}%
\pgfpathlineto{\pgfqpoint{0.000000in}{-0.048611in}}%
\pgfusepath{stroke,fill}%
}%
\begin{pgfscope}%
\pgfsys@transformshift{3.085673in}{0.450000in}%
\pgfsys@useobject{currentmarker}{}%
\end{pgfscope}%
\end{pgfscope}%
\begin{pgfscope}%
\definecolor{textcolor}{rgb}{0.000000,0.000000,0.000000}%
\pgfsetstrokecolor{textcolor}%
\pgfsetfillcolor{textcolor}%
\pgftext[x=3.085673in,y=0.352778in,,top]{\color{textcolor}\rmfamily\fontsize{10.000000}{12.000000}\selectfont \(\displaystyle 150\)}%
\end{pgfscope}%
\begin{pgfscope}%
\pgfsetbuttcap%
\pgfsetroundjoin%
\definecolor{currentfill}{rgb}{0.000000,0.000000,0.000000}%
\pgfsetfillcolor{currentfill}%
\pgfsetlinewidth{0.803000pt}%
\definecolor{currentstroke}{rgb}{0.000000,0.000000,0.000000}%
\pgfsetstrokecolor{currentstroke}%
\pgfsetdash{}{0pt}%
\pgfsys@defobject{currentmarker}{\pgfqpoint{0.000000in}{-0.048611in}}{\pgfqpoint{0.000000in}{0.000000in}}{%
\pgfpathmoveto{\pgfqpoint{0.000000in}{0.000000in}}%
\pgfpathlineto{\pgfqpoint{0.000000in}{-0.048611in}}%
\pgfusepath{stroke,fill}%
}%
\begin{pgfscope}%
\pgfsys@transformshift{3.844357in}{0.450000in}%
\pgfsys@useobject{currentmarker}{}%
\end{pgfscope}%
\end{pgfscope}%
\begin{pgfscope}%
\definecolor{textcolor}{rgb}{0.000000,0.000000,0.000000}%
\pgfsetstrokecolor{textcolor}%
\pgfsetfillcolor{textcolor}%
\pgftext[x=3.844357in,y=0.352778in,,top]{\color{textcolor}\rmfamily\fontsize{10.000000}{12.000000}\selectfont \(\displaystyle 200\)}%
\end{pgfscope}%
\begin{pgfscope}%
\definecolor{textcolor}{rgb}{0.000000,0.000000,0.000000}%
\pgfsetstrokecolor{textcolor}%
\pgfsetfillcolor{textcolor}%
\pgftext[x=2.698745in,y=0.173766in,,top]{\color{textcolor}\rmfamily\fontsize{10.000000}{12.000000}\selectfont Nr pokolenia}%
\end{pgfscope}%
\begin{pgfscope}%
\pgfsetbuttcap%
\pgfsetroundjoin%
\definecolor{currentfill}{rgb}{0.000000,0.000000,0.000000}%
\pgfsetfillcolor{currentfill}%
\pgfsetlinewidth{0.803000pt}%
\definecolor{currentstroke}{rgb}{0.000000,0.000000,0.000000}%
\pgfsetstrokecolor{currentstroke}%
\pgfsetdash{}{0pt}%
\pgfsys@defobject{currentmarker}{\pgfqpoint{-0.048611in}{0.000000in}}{\pgfqpoint{0.000000in}{0.000000in}}{%
\pgfpathmoveto{\pgfqpoint{0.000000in}{0.000000in}}%
\pgfpathlineto{\pgfqpoint{-0.048611in}{0.000000in}}%
\pgfusepath{stroke,fill}%
}%
\begin{pgfscope}%
\pgfsys@transformshift{0.809624in}{0.588611in}%
\pgfsys@useobject{currentmarker}{}%
\end{pgfscope}%
\end{pgfscope}%
\begin{pgfscope}%
\definecolor{textcolor}{rgb}{0.000000,0.000000,0.000000}%
\pgfsetstrokecolor{textcolor}%
\pgfsetfillcolor{textcolor}%
\pgftext[x=0.365178in, y=0.540386in, left, base]{\color{textcolor}\rmfamily\fontsize{10.000000}{12.000000}\selectfont \(\displaystyle 50000\)}%
\end{pgfscope}%
\begin{pgfscope}%
\pgfsetbuttcap%
\pgfsetroundjoin%
\definecolor{currentfill}{rgb}{0.000000,0.000000,0.000000}%
\pgfsetfillcolor{currentfill}%
\pgfsetlinewidth{0.803000pt}%
\definecolor{currentstroke}{rgb}{0.000000,0.000000,0.000000}%
\pgfsetstrokecolor{currentstroke}%
\pgfsetdash{}{0pt}%
\pgfsys@defobject{currentmarker}{\pgfqpoint{-0.048611in}{0.000000in}}{\pgfqpoint{0.000000in}{0.000000in}}{%
\pgfpathmoveto{\pgfqpoint{0.000000in}{0.000000in}}%
\pgfpathlineto{\pgfqpoint{-0.048611in}{0.000000in}}%
\pgfusepath{stroke,fill}%
}%
\begin{pgfscope}%
\pgfsys@transformshift{0.809624in}{0.924524in}%
\pgfsys@useobject{currentmarker}{}%
\end{pgfscope}%
\end{pgfscope}%
\begin{pgfscope}%
\definecolor{textcolor}{rgb}{0.000000,0.000000,0.000000}%
\pgfsetstrokecolor{textcolor}%
\pgfsetfillcolor{textcolor}%
\pgftext[x=0.295733in, y=0.876298in, left, base]{\color{textcolor}\rmfamily\fontsize{10.000000}{12.000000}\selectfont \(\displaystyle 100000\)}%
\end{pgfscope}%
\begin{pgfscope}%
\pgfsetbuttcap%
\pgfsetroundjoin%
\definecolor{currentfill}{rgb}{0.000000,0.000000,0.000000}%
\pgfsetfillcolor{currentfill}%
\pgfsetlinewidth{0.803000pt}%
\definecolor{currentstroke}{rgb}{0.000000,0.000000,0.000000}%
\pgfsetstrokecolor{currentstroke}%
\pgfsetdash{}{0pt}%
\pgfsys@defobject{currentmarker}{\pgfqpoint{-0.048611in}{0.000000in}}{\pgfqpoint{0.000000in}{0.000000in}}{%
\pgfpathmoveto{\pgfqpoint{0.000000in}{0.000000in}}%
\pgfpathlineto{\pgfqpoint{-0.048611in}{0.000000in}}%
\pgfusepath{stroke,fill}%
}%
\begin{pgfscope}%
\pgfsys@transformshift{0.809624in}{1.260436in}%
\pgfsys@useobject{currentmarker}{}%
\end{pgfscope}%
\end{pgfscope}%
\begin{pgfscope}%
\definecolor{textcolor}{rgb}{0.000000,0.000000,0.000000}%
\pgfsetstrokecolor{textcolor}%
\pgfsetfillcolor{textcolor}%
\pgftext[x=0.295733in, y=1.212211in, left, base]{\color{textcolor}\rmfamily\fontsize{10.000000}{12.000000}\selectfont \(\displaystyle 150000\)}%
\end{pgfscope}%
\begin{pgfscope}%
\pgfsetbuttcap%
\pgfsetroundjoin%
\definecolor{currentfill}{rgb}{0.000000,0.000000,0.000000}%
\pgfsetfillcolor{currentfill}%
\pgfsetlinewidth{0.803000pt}%
\definecolor{currentstroke}{rgb}{0.000000,0.000000,0.000000}%
\pgfsetstrokecolor{currentstroke}%
\pgfsetdash{}{0pt}%
\pgfsys@defobject{currentmarker}{\pgfqpoint{-0.048611in}{0.000000in}}{\pgfqpoint{0.000000in}{0.000000in}}{%
\pgfpathmoveto{\pgfqpoint{0.000000in}{0.000000in}}%
\pgfpathlineto{\pgfqpoint{-0.048611in}{0.000000in}}%
\pgfusepath{stroke,fill}%
}%
\begin{pgfscope}%
\pgfsys@transformshift{0.809624in}{1.596349in}%
\pgfsys@useobject{currentmarker}{}%
\end{pgfscope}%
\end{pgfscope}%
\begin{pgfscope}%
\definecolor{textcolor}{rgb}{0.000000,0.000000,0.000000}%
\pgfsetstrokecolor{textcolor}%
\pgfsetfillcolor{textcolor}%
\pgftext[x=0.295733in, y=1.548124in, left, base]{\color{textcolor}\rmfamily\fontsize{10.000000}{12.000000}\selectfont \(\displaystyle 200000\)}%
\end{pgfscope}%
\begin{pgfscope}%
\pgfsetbuttcap%
\pgfsetroundjoin%
\definecolor{currentfill}{rgb}{0.000000,0.000000,0.000000}%
\pgfsetfillcolor{currentfill}%
\pgfsetlinewidth{0.803000pt}%
\definecolor{currentstroke}{rgb}{0.000000,0.000000,0.000000}%
\pgfsetstrokecolor{currentstroke}%
\pgfsetdash{}{0pt}%
\pgfsys@defobject{currentmarker}{\pgfqpoint{-0.048611in}{0.000000in}}{\pgfqpoint{0.000000in}{0.000000in}}{%
\pgfpathmoveto{\pgfqpoint{0.000000in}{0.000000in}}%
\pgfpathlineto{\pgfqpoint{-0.048611in}{0.000000in}}%
\pgfusepath{stroke,fill}%
}%
\begin{pgfscope}%
\pgfsys@transformshift{0.809624in}{1.932262in}%
\pgfsys@useobject{currentmarker}{}%
\end{pgfscope}%
\end{pgfscope}%
\begin{pgfscope}%
\definecolor{textcolor}{rgb}{0.000000,0.000000,0.000000}%
\pgfsetstrokecolor{textcolor}%
\pgfsetfillcolor{textcolor}%
\pgftext[x=0.295733in, y=1.884037in, left, base]{\color{textcolor}\rmfamily\fontsize{10.000000}{12.000000}\selectfont \(\displaystyle 250000\)}%
\end{pgfscope}%
\begin{pgfscope}%
\pgfsetbuttcap%
\pgfsetroundjoin%
\definecolor{currentfill}{rgb}{0.000000,0.000000,0.000000}%
\pgfsetfillcolor{currentfill}%
\pgfsetlinewidth{0.803000pt}%
\definecolor{currentstroke}{rgb}{0.000000,0.000000,0.000000}%
\pgfsetstrokecolor{currentstroke}%
\pgfsetdash{}{0pt}%
\pgfsys@defobject{currentmarker}{\pgfqpoint{-0.048611in}{0.000000in}}{\pgfqpoint{0.000000in}{0.000000in}}{%
\pgfpathmoveto{\pgfqpoint{0.000000in}{0.000000in}}%
\pgfpathlineto{\pgfqpoint{-0.048611in}{0.000000in}}%
\pgfusepath{stroke,fill}%
}%
\begin{pgfscope}%
\pgfsys@transformshift{0.809624in}{2.268175in}%
\pgfsys@useobject{currentmarker}{}%
\end{pgfscope}%
\end{pgfscope}%
\begin{pgfscope}%
\definecolor{textcolor}{rgb}{0.000000,0.000000,0.000000}%
\pgfsetstrokecolor{textcolor}%
\pgfsetfillcolor{textcolor}%
\pgftext[x=0.295733in, y=2.219949in, left, base]{\color{textcolor}\rmfamily\fontsize{10.000000}{12.000000}\selectfont \(\displaystyle 300000\)}%
\end{pgfscope}%
\begin{pgfscope}%
\pgfsetbuttcap%
\pgfsetroundjoin%
\definecolor{currentfill}{rgb}{0.000000,0.000000,0.000000}%
\pgfsetfillcolor{currentfill}%
\pgfsetlinewidth{0.803000pt}%
\definecolor{currentstroke}{rgb}{0.000000,0.000000,0.000000}%
\pgfsetstrokecolor{currentstroke}%
\pgfsetdash{}{0pt}%
\pgfsys@defobject{currentmarker}{\pgfqpoint{-0.048611in}{0.000000in}}{\pgfqpoint{0.000000in}{0.000000in}}{%
\pgfpathmoveto{\pgfqpoint{0.000000in}{0.000000in}}%
\pgfpathlineto{\pgfqpoint{-0.048611in}{0.000000in}}%
\pgfusepath{stroke,fill}%
}%
\begin{pgfscope}%
\pgfsys@transformshift{0.809624in}{2.604087in}%
\pgfsys@useobject{currentmarker}{}%
\end{pgfscope}%
\end{pgfscope}%
\begin{pgfscope}%
\definecolor{textcolor}{rgb}{0.000000,0.000000,0.000000}%
\pgfsetstrokecolor{textcolor}%
\pgfsetfillcolor{textcolor}%
\pgftext[x=0.295733in, y=2.555862in, left, base]{\color{textcolor}\rmfamily\fontsize{10.000000}{12.000000}\selectfont \(\displaystyle 350000\)}%
\end{pgfscope}%
\begin{pgfscope}%
\pgfsetbuttcap%
\pgfsetroundjoin%
\definecolor{currentfill}{rgb}{0.000000,0.000000,0.000000}%
\pgfsetfillcolor{currentfill}%
\pgfsetlinewidth{0.803000pt}%
\definecolor{currentstroke}{rgb}{0.000000,0.000000,0.000000}%
\pgfsetstrokecolor{currentstroke}%
\pgfsetdash{}{0pt}%
\pgfsys@defobject{currentmarker}{\pgfqpoint{-0.048611in}{0.000000in}}{\pgfqpoint{0.000000in}{0.000000in}}{%
\pgfpathmoveto{\pgfqpoint{0.000000in}{0.000000in}}%
\pgfpathlineto{\pgfqpoint{-0.048611in}{0.000000in}}%
\pgfusepath{stroke,fill}%
}%
\begin{pgfscope}%
\pgfsys@transformshift{0.809624in}{2.940000in}%
\pgfsys@useobject{currentmarker}{}%
\end{pgfscope}%
\end{pgfscope}%
\begin{pgfscope}%
\definecolor{textcolor}{rgb}{0.000000,0.000000,0.000000}%
\pgfsetstrokecolor{textcolor}%
\pgfsetfillcolor{textcolor}%
\pgftext[x=0.295733in, y=2.891775in, left, base]{\color{textcolor}\rmfamily\fontsize{10.000000}{12.000000}\selectfont \(\displaystyle 400000\)}%
\end{pgfscope}%
\begin{pgfscope}%
\definecolor{textcolor}{rgb}{0.000000,0.000000,0.000000}%
\pgfsetstrokecolor{textcolor}%
\pgfsetfillcolor{textcolor}%
\pgftext[x=0.240178in,y=1.695000in,,bottom,rotate=90.000000]{\color{textcolor}\rmfamily\fontsize{10.000000}{12.000000}\selectfont Wartość funkcji przystosowania}%
\end{pgfscope}%
\begin{pgfscope}%
\pgfpathrectangle{\pgfqpoint{0.809624in}{0.450000in}}{\pgfqpoint{3.778243in}{2.490000in}}%
\pgfusepath{clip}%
\pgfsetrectcap%
\pgfsetroundjoin%
\pgfsetlinewidth{1.003750pt}%
\definecolor{currentstroke}{rgb}{0.000000,0.000000,1.000000}%
\pgfsetstrokecolor{currentstroke}%
\pgfsetdash{}{0pt}%
\pgfpathmoveto{\pgfqpoint{0.809624in}{2.754905in}}%
\pgfpathlineto{\pgfqpoint{0.824797in}{2.691129in}}%
\pgfpathlineto{\pgfqpoint{0.839971in}{2.669718in}}%
\pgfpathlineto{\pgfqpoint{0.855144in}{2.638377in}}%
\pgfpathlineto{\pgfqpoint{0.870318in}{2.577308in}}%
\pgfpathlineto{\pgfqpoint{0.885492in}{2.561621in}}%
\pgfpathlineto{\pgfqpoint{0.900666in}{2.539364in}}%
\pgfpathlineto{\pgfqpoint{0.915839in}{2.471489in}}%
\pgfpathlineto{\pgfqpoint{0.931013in}{2.485161in}}%
\pgfpathlineto{\pgfqpoint{0.946187in}{2.434384in}}%
\pgfpathlineto{\pgfqpoint{0.961360in}{2.412966in}}%
\pgfpathlineto{\pgfqpoint{0.976534in}{2.399604in}}%
\pgfpathlineto{\pgfqpoint{0.991707in}{2.378011in}}%
\pgfpathlineto{\pgfqpoint{1.022055in}{2.309015in}}%
\pgfpathlineto{\pgfqpoint{1.037228in}{2.301773in}}%
\pgfpathlineto{\pgfqpoint{1.052402in}{2.258232in}}%
\pgfpathlineto{\pgfqpoint{1.067576in}{2.256539in}}%
\pgfpathlineto{\pgfqpoint{1.082750in}{2.237291in}}%
\pgfpathlineto{\pgfqpoint{1.097923in}{2.247167in}}%
\pgfpathlineto{\pgfqpoint{1.143444in}{2.171304in}}%
\pgfpathlineto{\pgfqpoint{1.158618in}{2.179144in}}%
\pgfpathlineto{\pgfqpoint{1.173791in}{2.173588in}}%
\pgfpathlineto{\pgfqpoint{1.188965in}{2.130746in}}%
\pgfpathlineto{\pgfqpoint{1.204139in}{2.079183in}}%
\pgfpathlineto{\pgfqpoint{1.219313in}{2.086305in}}%
\pgfpathlineto{\pgfqpoint{1.234486in}{2.073923in}}%
\pgfpathlineto{\pgfqpoint{1.249660in}{2.097591in}}%
\pgfpathlineto{\pgfqpoint{1.264833in}{2.037759in}}%
\pgfpathlineto{\pgfqpoint{1.280007in}{2.033647in}}%
\pgfpathlineto{\pgfqpoint{1.295181in}{2.000230in}}%
\pgfpathlineto{\pgfqpoint{1.310355in}{2.006512in}}%
\pgfpathlineto{\pgfqpoint{1.325528in}{1.992229in}}%
\pgfpathlineto{\pgfqpoint{1.340702in}{1.956972in}}%
\pgfpathlineto{\pgfqpoint{1.355876in}{1.973236in}}%
\pgfpathlineto{\pgfqpoint{1.371049in}{1.924408in}}%
\pgfpathlineto{\pgfqpoint{1.386223in}{1.937341in}}%
\pgfpathlineto{\pgfqpoint{1.401396in}{1.915473in}}%
\pgfpathlineto{\pgfqpoint{1.416570in}{1.909165in}}%
\pgfpathlineto{\pgfqpoint{1.431744in}{1.943172in}}%
\pgfpathlineto{\pgfqpoint{1.446917in}{1.902641in}}%
\pgfpathlineto{\pgfqpoint{1.462091in}{1.895668in}}%
\pgfpathlineto{\pgfqpoint{1.477265in}{1.853181in}}%
\pgfpathlineto{\pgfqpoint{1.492439in}{1.899255in}}%
\pgfpathlineto{\pgfqpoint{1.507612in}{1.843131in}}%
\pgfpathlineto{\pgfqpoint{1.522786in}{1.819086in}}%
\pgfpathlineto{\pgfqpoint{1.537960in}{1.846134in}}%
\pgfpathlineto{\pgfqpoint{1.553133in}{1.798051in}}%
\pgfpathlineto{\pgfqpoint{1.568307in}{1.787174in}}%
\pgfpathlineto{\pgfqpoint{1.583481in}{1.767739in}}%
\pgfpathlineto{\pgfqpoint{1.598654in}{1.779448in}}%
\pgfpathlineto{\pgfqpoint{1.613828in}{1.782297in}}%
\pgfpathlineto{\pgfqpoint{1.629001in}{1.730701in}}%
\pgfpathlineto{\pgfqpoint{1.644175in}{1.705333in}}%
\pgfpathlineto{\pgfqpoint{1.659349in}{1.723197in}}%
\pgfpathlineto{\pgfqpoint{1.674522in}{1.696552in}}%
\pgfpathlineto{\pgfqpoint{1.689696in}{1.678984in}}%
\pgfpathlineto{\pgfqpoint{1.704870in}{1.680045in}}%
\pgfpathlineto{\pgfqpoint{1.720043in}{1.673233in}}%
\pgfpathlineto{\pgfqpoint{1.735217in}{1.664056in}}%
\pgfpathlineto{\pgfqpoint{1.765564in}{1.633609in}}%
\pgfpathlineto{\pgfqpoint{1.780738in}{1.623921in}}%
\pgfpathlineto{\pgfqpoint{1.795912in}{1.619198in}}%
\pgfpathlineto{\pgfqpoint{1.811085in}{1.618828in}}%
\pgfpathlineto{\pgfqpoint{1.826259in}{1.594280in}}%
\pgfpathlineto{\pgfqpoint{1.841433in}{1.586245in}}%
\pgfpathlineto{\pgfqpoint{1.856606in}{1.562543in}}%
\pgfpathlineto{\pgfqpoint{1.871780in}{1.574253in}}%
\pgfpathlineto{\pgfqpoint{1.886954in}{1.528072in}}%
\pgfpathlineto{\pgfqpoint{1.902127in}{1.518720in}}%
\pgfpathlineto{\pgfqpoint{1.917301in}{1.538149in}}%
\pgfpathlineto{\pgfqpoint{1.947648in}{1.479727in}}%
\pgfpathlineto{\pgfqpoint{1.962822in}{1.480493in}}%
\pgfpathlineto{\pgfqpoint{1.977996in}{1.446277in}}%
\pgfpathlineto{\pgfqpoint{1.993170in}{1.442387in}}%
\pgfpathlineto{\pgfqpoint{2.008343in}{1.445833in}}%
\pgfpathlineto{\pgfqpoint{2.023517in}{1.422602in}}%
\pgfpathlineto{\pgfqpoint{2.038690in}{1.434412in}}%
\pgfpathlineto{\pgfqpoint{2.053864in}{1.434083in}}%
\pgfpathlineto{\pgfqpoint{2.069038in}{1.407533in}}%
\pgfpathlineto{\pgfqpoint{2.084211in}{1.405604in}}%
\pgfpathlineto{\pgfqpoint{2.114559in}{1.381513in}}%
\pgfpathlineto{\pgfqpoint{2.129733in}{1.389084in}}%
\pgfpathlineto{\pgfqpoint{2.144906in}{1.371046in}}%
\pgfpathlineto{\pgfqpoint{2.160080in}{1.368889in}}%
\pgfpathlineto{\pgfqpoint{2.175254in}{1.340646in}}%
\pgfpathlineto{\pgfqpoint{2.190427in}{1.352322in}}%
\pgfpathlineto{\pgfqpoint{2.205601in}{1.337515in}}%
\pgfpathlineto{\pgfqpoint{2.220775in}{1.329413in}}%
\pgfpathlineto{\pgfqpoint{2.235948in}{1.340390in}}%
\pgfpathlineto{\pgfqpoint{2.251122in}{1.305496in}}%
\pgfpathlineto{\pgfqpoint{2.266296in}{1.307659in}}%
\pgfpathlineto{\pgfqpoint{2.281469in}{1.286073in}}%
\pgfpathlineto{\pgfqpoint{2.296643in}{1.287350in}}%
\pgfpathlineto{\pgfqpoint{2.311816in}{1.281626in}}%
\pgfpathlineto{\pgfqpoint{2.326990in}{1.272281in}}%
\pgfpathlineto{\pgfqpoint{2.342164in}{1.258266in}}%
\pgfpathlineto{\pgfqpoint{2.357337in}{1.268619in}}%
\pgfpathlineto{\pgfqpoint{2.372511in}{1.245152in}}%
\pgfpathlineto{\pgfqpoint{2.387685in}{1.248525in}}%
\pgfpathlineto{\pgfqpoint{2.402858in}{1.242102in}}%
\pgfpathlineto{\pgfqpoint{2.418032in}{1.230023in}}%
\pgfpathlineto{\pgfqpoint{2.433206in}{1.222015in}}%
\pgfpathlineto{\pgfqpoint{2.448379in}{1.221773in}}%
\pgfpathlineto{\pgfqpoint{2.463553in}{1.198561in}}%
\pgfpathlineto{\pgfqpoint{2.478727in}{1.194974in}}%
\pgfpathlineto{\pgfqpoint{2.493900in}{1.189028in}}%
\pgfpathlineto{\pgfqpoint{2.509074in}{1.178581in}}%
\pgfpathlineto{\pgfqpoint{2.524248in}{1.180126in}}%
\pgfpathlineto{\pgfqpoint{2.554595in}{1.171776in}}%
\pgfpathlineto{\pgfqpoint{2.569769in}{1.150472in}}%
\pgfpathlineto{\pgfqpoint{2.584942in}{1.173697in}}%
\pgfpathlineto{\pgfqpoint{2.600116in}{1.142988in}}%
\pgfpathlineto{\pgfqpoint{2.615290in}{1.125467in}}%
\pgfpathlineto{\pgfqpoint{2.630463in}{1.134382in}}%
\pgfpathlineto{\pgfqpoint{2.645637in}{1.125910in}}%
\pgfpathlineto{\pgfqpoint{2.660811in}{1.122477in}}%
\pgfpathlineto{\pgfqpoint{2.675984in}{1.108967in}}%
\pgfpathlineto{\pgfqpoint{2.691158in}{1.107751in}}%
\pgfpathlineto{\pgfqpoint{2.706332in}{1.111661in}}%
\pgfpathlineto{\pgfqpoint{2.721506in}{1.089309in}}%
\pgfpathlineto{\pgfqpoint{2.751853in}{1.071237in}}%
\pgfpathlineto{\pgfqpoint{2.767027in}{1.072554in}}%
\pgfpathlineto{\pgfqpoint{2.782200in}{1.066756in}}%
\pgfpathlineto{\pgfqpoint{2.797374in}{1.056396in}}%
\pgfpathlineto{\pgfqpoint{2.812547in}{1.052809in}}%
\pgfpathlineto{\pgfqpoint{2.827721in}{1.057041in}}%
\pgfpathlineto{\pgfqpoint{2.842895in}{1.048785in}}%
\pgfpathlineto{\pgfqpoint{2.858068in}{1.052406in}}%
\pgfpathlineto{\pgfqpoint{2.873242in}{1.044660in}}%
\pgfpathlineto{\pgfqpoint{2.888416in}{1.040514in}}%
\pgfpathlineto{\pgfqpoint{2.903589in}{1.020669in}}%
\pgfpathlineto{\pgfqpoint{2.918763in}{1.031431in}}%
\pgfpathlineto{\pgfqpoint{2.933937in}{1.017195in}}%
\pgfpathlineto{\pgfqpoint{2.949110in}{1.026614in}}%
\pgfpathlineto{\pgfqpoint{2.964284in}{1.011015in}}%
\pgfpathlineto{\pgfqpoint{2.979458in}{1.007380in}}%
\pgfpathlineto{\pgfqpoint{3.009805in}{1.009442in}}%
\pgfpathlineto{\pgfqpoint{3.024979in}{0.997054in}}%
\pgfpathlineto{\pgfqpoint{3.040153in}{0.990725in}}%
\pgfpathlineto{\pgfqpoint{3.055326in}{0.991001in}}%
\pgfpathlineto{\pgfqpoint{3.070500in}{0.985888in}}%
\pgfpathlineto{\pgfqpoint{3.085673in}{0.993688in}}%
\pgfpathlineto{\pgfqpoint{3.100847in}{0.977652in}}%
\pgfpathlineto{\pgfqpoint{3.116021in}{0.977275in}}%
\pgfpathlineto{\pgfqpoint{3.131194in}{0.986856in}}%
\pgfpathlineto{\pgfqpoint{3.146368in}{0.961011in}}%
\pgfpathlineto{\pgfqpoint{3.161542in}{0.962912in}}%
\pgfpathlineto{\pgfqpoint{3.176715in}{0.975394in}}%
\pgfpathlineto{\pgfqpoint{3.191889in}{0.957329in}}%
\pgfpathlineto{\pgfqpoint{3.222236in}{0.961568in}}%
\pgfpathlineto{\pgfqpoint{3.237410in}{0.976913in}}%
\pgfpathlineto{\pgfqpoint{3.252584in}{0.964780in}}%
\pgfpathlineto{\pgfqpoint{3.267757in}{0.961857in}}%
\pgfpathlineto{\pgfqpoint{3.282931in}{0.968515in}}%
\pgfpathlineto{\pgfqpoint{3.298105in}{0.954232in}}%
\pgfpathlineto{\pgfqpoint{3.313278in}{0.957873in}}%
\pgfpathlineto{\pgfqpoint{3.328452in}{0.952915in}}%
\pgfpathlineto{\pgfqpoint{3.343626in}{0.950423in}}%
\pgfpathlineto{\pgfqpoint{3.358799in}{0.945989in}}%
\pgfpathlineto{\pgfqpoint{3.373973in}{0.951625in}}%
\pgfpathlineto{\pgfqpoint{3.389147in}{0.947923in}}%
\pgfpathlineto{\pgfqpoint{3.404321in}{0.945458in}}%
\pgfpathlineto{\pgfqpoint{3.419494in}{0.933372in}}%
\pgfpathlineto{\pgfqpoint{3.434668in}{0.925243in}}%
\pgfpathlineto{\pgfqpoint{3.449842in}{0.922172in}}%
\pgfpathlineto{\pgfqpoint{3.465015in}{0.926546in}}%
\pgfpathlineto{\pgfqpoint{3.480189in}{0.928951in}}%
\pgfpathlineto{\pgfqpoint{3.495363in}{0.912370in}}%
\pgfpathlineto{\pgfqpoint{3.510536in}{0.915817in}}%
\pgfpathlineto{\pgfqpoint{3.525710in}{0.907029in}}%
\pgfpathlineto{\pgfqpoint{3.540884in}{0.904591in}}%
\pgfpathlineto{\pgfqpoint{3.556057in}{0.890301in}}%
\pgfpathlineto{\pgfqpoint{3.571231in}{0.883576in}}%
\pgfpathlineto{\pgfqpoint{4.587867in}{0.883576in}}%
\pgfpathlineto{\pgfqpoint{4.587867in}{0.883576in}}%
\pgfusepath{stroke}%
\end{pgfscope}%
\begin{pgfscope}%
\pgfpathrectangle{\pgfqpoint{0.809624in}{0.450000in}}{\pgfqpoint{3.778243in}{2.490000in}}%
\pgfusepath{clip}%
\pgfsetrectcap%
\pgfsetroundjoin%
\pgfsetlinewidth{1.003750pt}%
\definecolor{currentstroke}{rgb}{0.411765,0.411765,0.411765}%
\pgfsetstrokecolor{currentstroke}%
\pgfsetdash{}{0pt}%
\pgfpathmoveto{\pgfqpoint{0.809624in}{2.191367in}}%
\pgfpathlineto{\pgfqpoint{0.824797in}{2.157188in}}%
\pgfpathlineto{\pgfqpoint{0.839971in}{2.111441in}}%
\pgfpathlineto{\pgfqpoint{0.855144in}{2.090910in}}%
\pgfpathlineto{\pgfqpoint{0.870318in}{2.045438in}}%
\pgfpathlineto{\pgfqpoint{0.885492in}{2.010994in}}%
\pgfpathlineto{\pgfqpoint{0.900666in}{1.987339in}}%
\pgfpathlineto{\pgfqpoint{0.915839in}{1.959512in}}%
\pgfpathlineto{\pgfqpoint{0.931013in}{1.947397in}}%
\pgfpathlineto{\pgfqpoint{0.946187in}{1.920217in}}%
\pgfpathlineto{\pgfqpoint{0.961360in}{1.899111in}}%
\pgfpathlineto{\pgfqpoint{0.976534in}{1.881898in}}%
\pgfpathlineto{\pgfqpoint{0.991707in}{1.869911in}}%
\pgfpathlineto{\pgfqpoint{1.006881in}{1.847660in}}%
\pgfpathlineto{\pgfqpoint{1.022055in}{1.821481in}}%
\pgfpathlineto{\pgfqpoint{1.037228in}{1.808857in}}%
\pgfpathlineto{\pgfqpoint{1.052402in}{1.793864in}}%
\pgfpathlineto{\pgfqpoint{1.067576in}{1.772283in}}%
\pgfpathlineto{\pgfqpoint{1.082750in}{1.757071in}}%
\pgfpathlineto{\pgfqpoint{1.097923in}{1.744438in}}%
\pgfpathlineto{\pgfqpoint{1.128270in}{1.716400in}}%
\pgfpathlineto{\pgfqpoint{1.143444in}{1.698622in}}%
\pgfpathlineto{\pgfqpoint{1.173791in}{1.675887in}}%
\pgfpathlineto{\pgfqpoint{1.188965in}{1.659958in}}%
\pgfpathlineto{\pgfqpoint{1.204139in}{1.647677in}}%
\pgfpathlineto{\pgfqpoint{1.219313in}{1.638221in}}%
\pgfpathlineto{\pgfqpoint{1.234486in}{1.624326in}}%
\pgfpathlineto{\pgfqpoint{1.249660in}{1.616364in}}%
\pgfpathlineto{\pgfqpoint{1.264833in}{1.606311in}}%
\pgfpathlineto{\pgfqpoint{1.280007in}{1.581887in}}%
\pgfpathlineto{\pgfqpoint{1.310355in}{1.563268in}}%
\pgfpathlineto{\pgfqpoint{1.325528in}{1.558664in}}%
\pgfpathlineto{\pgfqpoint{1.371049in}{1.526077in}}%
\pgfpathlineto{\pgfqpoint{1.386223in}{1.519625in}}%
\pgfpathlineto{\pgfqpoint{1.401396in}{1.508982in}}%
\pgfpathlineto{\pgfqpoint{1.416570in}{1.495561in}}%
\pgfpathlineto{\pgfqpoint{1.431744in}{1.491558in}}%
\pgfpathlineto{\pgfqpoint{1.477265in}{1.462821in}}%
\pgfpathlineto{\pgfqpoint{1.492439in}{1.458118in}}%
\pgfpathlineto{\pgfqpoint{1.507612in}{1.443246in}}%
\pgfpathlineto{\pgfqpoint{1.522786in}{1.436940in}}%
\pgfpathlineto{\pgfqpoint{1.537960in}{1.427686in}}%
\pgfpathlineto{\pgfqpoint{1.553133in}{1.422419in}}%
\pgfpathlineto{\pgfqpoint{1.568307in}{1.415236in}}%
\pgfpathlineto{\pgfqpoint{1.583481in}{1.402305in}}%
\pgfpathlineto{\pgfqpoint{1.613828in}{1.390369in}}%
\pgfpathlineto{\pgfqpoint{1.629001in}{1.377984in}}%
\pgfpathlineto{\pgfqpoint{1.644175in}{1.367501in}}%
\pgfpathlineto{\pgfqpoint{1.659349in}{1.362260in}}%
\pgfpathlineto{\pgfqpoint{1.674522in}{1.354208in}}%
\pgfpathlineto{\pgfqpoint{1.689696in}{1.351472in}}%
\pgfpathlineto{\pgfqpoint{1.704870in}{1.344509in}}%
\pgfpathlineto{\pgfqpoint{1.720043in}{1.327448in}}%
\pgfpathlineto{\pgfqpoint{1.735217in}{1.319881in}}%
\pgfpathlineto{\pgfqpoint{1.765564in}{1.301412in}}%
\pgfpathlineto{\pgfqpoint{1.780738in}{1.296573in}}%
\pgfpathlineto{\pgfqpoint{1.795912in}{1.287218in}}%
\pgfpathlineto{\pgfqpoint{1.826259in}{1.272988in}}%
\pgfpathlineto{\pgfqpoint{1.917301in}{1.223789in}}%
\pgfpathlineto{\pgfqpoint{1.932475in}{1.216653in}}%
\pgfpathlineto{\pgfqpoint{1.947648in}{1.211270in}}%
\pgfpathlineto{\pgfqpoint{1.977996in}{1.189543in}}%
\pgfpathlineto{\pgfqpoint{2.023517in}{1.174933in}}%
\pgfpathlineto{\pgfqpoint{2.038690in}{1.169784in}}%
\pgfpathlineto{\pgfqpoint{2.053864in}{1.160709in}}%
\pgfpathlineto{\pgfqpoint{2.069038in}{1.157645in}}%
\pgfpathlineto{\pgfqpoint{2.084211in}{1.150784in}}%
\pgfpathlineto{\pgfqpoint{2.099385in}{1.141974in}}%
\pgfpathlineto{\pgfqpoint{2.114559in}{1.137832in}}%
\pgfpathlineto{\pgfqpoint{2.129733in}{1.129517in}}%
\pgfpathlineto{\pgfqpoint{2.160080in}{1.118415in}}%
\pgfpathlineto{\pgfqpoint{2.175254in}{1.114712in}}%
\pgfpathlineto{\pgfqpoint{2.190427in}{1.106754in}}%
\pgfpathlineto{\pgfqpoint{2.205601in}{1.100317in}}%
\pgfpathlineto{\pgfqpoint{2.220775in}{1.091159in}}%
\pgfpathlineto{\pgfqpoint{2.266296in}{1.073320in}}%
\pgfpathlineto{\pgfqpoint{2.281469in}{1.069782in}}%
\pgfpathlineto{\pgfqpoint{2.311816in}{1.057233in}}%
\pgfpathlineto{\pgfqpoint{2.326990in}{1.052132in}}%
\pgfpathlineto{\pgfqpoint{2.342164in}{1.049739in}}%
\pgfpathlineto{\pgfqpoint{2.357337in}{1.044465in}}%
\pgfpathlineto{\pgfqpoint{2.372511in}{1.037607in}}%
\pgfpathlineto{\pgfqpoint{2.387685in}{1.034703in}}%
\pgfpathlineto{\pgfqpoint{2.402858in}{1.026829in}}%
\pgfpathlineto{\pgfqpoint{2.418032in}{1.023124in}}%
\pgfpathlineto{\pgfqpoint{2.554595in}{0.978648in}}%
\pgfpathlineto{\pgfqpoint{2.600116in}{0.965263in}}%
\pgfpathlineto{\pgfqpoint{2.615290in}{0.960433in}}%
\pgfpathlineto{\pgfqpoint{2.630463in}{0.958149in}}%
\pgfpathlineto{\pgfqpoint{2.645637in}{0.951833in}}%
\pgfpathlineto{\pgfqpoint{2.691158in}{0.939767in}}%
\pgfpathlineto{\pgfqpoint{2.706332in}{0.937034in}}%
\pgfpathlineto{\pgfqpoint{2.721506in}{0.932465in}}%
\pgfpathlineto{\pgfqpoint{2.751853in}{0.926554in}}%
\pgfpathlineto{\pgfqpoint{2.767027in}{0.921584in}}%
\pgfpathlineto{\pgfqpoint{2.797374in}{0.914720in}}%
\pgfpathlineto{\pgfqpoint{2.812547in}{0.910663in}}%
\pgfpathlineto{\pgfqpoint{2.827721in}{0.908475in}}%
\pgfpathlineto{\pgfqpoint{2.888416in}{0.893742in}}%
\pgfpathlineto{\pgfqpoint{2.903589in}{0.891471in}}%
\pgfpathlineto{\pgfqpoint{2.918763in}{0.886816in}}%
\pgfpathlineto{\pgfqpoint{2.933937in}{0.883536in}}%
\pgfpathlineto{\pgfqpoint{2.949110in}{0.881481in}}%
\pgfpathlineto{\pgfqpoint{2.979458in}{0.875002in}}%
\pgfpathlineto{\pgfqpoint{2.994632in}{0.872659in}}%
\pgfpathlineto{\pgfqpoint{3.009805in}{0.868868in}}%
\pgfpathlineto{\pgfqpoint{3.070500in}{0.858719in}}%
\pgfpathlineto{\pgfqpoint{3.085673in}{0.857435in}}%
\pgfpathlineto{\pgfqpoint{3.100847in}{0.854888in}}%
\pgfpathlineto{\pgfqpoint{3.161542in}{0.848657in}}%
\pgfpathlineto{\pgfqpoint{3.191889in}{0.844692in}}%
\pgfpathlineto{\pgfqpoint{3.237410in}{0.840849in}}%
\pgfpathlineto{\pgfqpoint{3.313278in}{0.835184in}}%
\pgfpathlineto{\pgfqpoint{3.419494in}{0.828746in}}%
\pgfpathlineto{\pgfqpoint{3.540884in}{0.824608in}}%
\pgfpathlineto{\pgfqpoint{3.859530in}{0.822274in}}%
\pgfpathlineto{\pgfqpoint{4.587867in}{0.822216in}}%
\pgfpathlineto{\pgfqpoint{4.587867in}{0.822216in}}%
\pgfusepath{stroke}%
\end{pgfscope}%
\begin{pgfscope}%
\pgfpathrectangle{\pgfqpoint{0.809624in}{0.450000in}}{\pgfqpoint{3.778243in}{2.490000in}}%
\pgfusepath{clip}%
\pgfsetrectcap%
\pgfsetroundjoin%
\pgfsetlinewidth{1.003750pt}%
\definecolor{currentstroke}{rgb}{0.000000,0.500000,0.000000}%
\pgfsetstrokecolor{currentstroke}%
\pgfsetdash{}{0pt}%
\pgfpathmoveto{\pgfqpoint{0.809624in}{2.164935in}}%
\pgfpathlineto{\pgfqpoint{0.839971in}{2.077658in}}%
\pgfpathlineto{\pgfqpoint{0.855144in}{2.057248in}}%
\pgfpathlineto{\pgfqpoint{0.870318in}{1.949118in}}%
\pgfpathlineto{\pgfqpoint{0.885492in}{1.972558in}}%
\pgfpathlineto{\pgfqpoint{0.900666in}{1.939121in}}%
\pgfpathlineto{\pgfqpoint{0.915839in}{1.921882in}}%
\pgfpathlineto{\pgfqpoint{0.931013in}{1.885221in}}%
\pgfpathlineto{\pgfqpoint{0.946187in}{1.873356in}}%
\pgfpathlineto{\pgfqpoint{0.961360in}{1.873356in}}%
\pgfpathlineto{\pgfqpoint{0.976534in}{1.854162in}}%
\pgfpathlineto{\pgfqpoint{0.991707in}{1.848304in}}%
\pgfpathlineto{\pgfqpoint{1.006881in}{1.803500in}}%
\pgfpathlineto{\pgfqpoint{1.022055in}{1.783668in}}%
\pgfpathlineto{\pgfqpoint{1.037228in}{1.776042in}}%
\pgfpathlineto{\pgfqpoint{1.052402in}{1.765723in}}%
\pgfpathlineto{\pgfqpoint{1.067576in}{1.744520in}}%
\pgfpathlineto{\pgfqpoint{1.082750in}{1.734866in}}%
\pgfpathlineto{\pgfqpoint{1.097923in}{1.715020in}}%
\pgfpathlineto{\pgfqpoint{1.113097in}{1.698944in}}%
\pgfpathlineto{\pgfqpoint{1.128270in}{1.668960in}}%
\pgfpathlineto{\pgfqpoint{1.143444in}{1.666535in}}%
\pgfpathlineto{\pgfqpoint{1.158618in}{1.657982in}}%
\pgfpathlineto{\pgfqpoint{1.173791in}{1.640219in}}%
\pgfpathlineto{\pgfqpoint{1.188965in}{1.628966in}}%
\pgfpathlineto{\pgfqpoint{1.204139in}{1.615711in}}%
\pgfpathlineto{\pgfqpoint{1.219313in}{1.592385in}}%
\pgfpathlineto{\pgfqpoint{1.234486in}{1.590094in}}%
\pgfpathlineto{\pgfqpoint{1.249660in}{1.584995in}}%
\pgfpathlineto{\pgfqpoint{1.264833in}{1.570067in}}%
\pgfpathlineto{\pgfqpoint{1.280007in}{1.529516in}}%
\pgfpathlineto{\pgfqpoint{1.295181in}{1.526452in}}%
\pgfpathlineto{\pgfqpoint{1.340702in}{1.525472in}}%
\pgfpathlineto{\pgfqpoint{1.355876in}{1.507077in}}%
\pgfpathlineto{\pgfqpoint{1.371049in}{1.483597in}}%
\pgfpathlineto{\pgfqpoint{1.386223in}{1.477396in}}%
\pgfpathlineto{\pgfqpoint{1.401396in}{1.476590in}}%
\pgfpathlineto{\pgfqpoint{1.416570in}{1.452162in}}%
\pgfpathlineto{\pgfqpoint{1.431744in}{1.457194in}}%
\pgfpathlineto{\pgfqpoint{1.446917in}{1.455602in}}%
\pgfpathlineto{\pgfqpoint{1.462091in}{1.437053in}}%
\pgfpathlineto{\pgfqpoint{1.477265in}{1.432128in}}%
\pgfpathlineto{\pgfqpoint{1.492439in}{1.420633in}}%
\pgfpathlineto{\pgfqpoint{1.507612in}{1.419182in}}%
\pgfpathlineto{\pgfqpoint{1.522786in}{1.414466in}}%
\pgfpathlineto{\pgfqpoint{1.537960in}{1.387774in}}%
\pgfpathlineto{\pgfqpoint{1.553133in}{1.387774in}}%
\pgfpathlineto{\pgfqpoint{1.568307in}{1.382863in}}%
\pgfpathlineto{\pgfqpoint{1.583481in}{1.360028in}}%
\pgfpathlineto{\pgfqpoint{1.598654in}{1.360028in}}%
\pgfpathlineto{\pgfqpoint{1.613828in}{1.349803in}}%
\pgfpathlineto{\pgfqpoint{1.629001in}{1.358530in}}%
\pgfpathlineto{\pgfqpoint{1.644175in}{1.345624in}}%
\pgfpathlineto{\pgfqpoint{1.659349in}{1.341989in}}%
\pgfpathlineto{\pgfqpoint{1.674522in}{1.330911in}}%
\pgfpathlineto{\pgfqpoint{1.689696in}{1.330911in}}%
\pgfpathlineto{\pgfqpoint{1.704870in}{1.323434in}}%
\pgfpathlineto{\pgfqpoint{1.720043in}{1.297185in}}%
\pgfpathlineto{\pgfqpoint{1.735217in}{1.281713in}}%
\pgfpathlineto{\pgfqpoint{1.750391in}{1.275633in}}%
\pgfpathlineto{\pgfqpoint{1.765564in}{1.262687in}}%
\pgfpathlineto{\pgfqpoint{1.780738in}{1.261592in}}%
\pgfpathlineto{\pgfqpoint{1.795912in}{1.258703in}}%
\pgfpathlineto{\pgfqpoint{1.811085in}{1.250131in}}%
\pgfpathlineto{\pgfqpoint{1.826259in}{1.234242in}}%
\pgfpathlineto{\pgfqpoint{1.841433in}{1.231225in}}%
\pgfpathlineto{\pgfqpoint{1.856606in}{1.211810in}}%
\pgfpathlineto{\pgfqpoint{1.871780in}{1.210789in}}%
\pgfpathlineto{\pgfqpoint{1.886954in}{1.192219in}}%
\pgfpathlineto{\pgfqpoint{1.902127in}{1.182747in}}%
\pgfpathlineto{\pgfqpoint{1.917301in}{1.176237in}}%
\pgfpathlineto{\pgfqpoint{1.932475in}{1.161604in}}%
\pgfpathlineto{\pgfqpoint{1.962822in}{1.161295in}}%
\pgfpathlineto{\pgfqpoint{1.977996in}{1.134530in}}%
\pgfpathlineto{\pgfqpoint{1.993170in}{1.122793in}}%
\pgfpathlineto{\pgfqpoint{2.008343in}{1.124889in}}%
\pgfpathlineto{\pgfqpoint{2.023517in}{1.123418in}}%
\pgfpathlineto{\pgfqpoint{2.038690in}{1.110283in}}%
\pgfpathlineto{\pgfqpoint{2.053864in}{1.110283in}}%
\pgfpathlineto{\pgfqpoint{2.069038in}{1.107583in}}%
\pgfpathlineto{\pgfqpoint{2.084211in}{1.101509in}}%
\pgfpathlineto{\pgfqpoint{2.099385in}{1.084949in}}%
\pgfpathlineto{\pgfqpoint{2.114559in}{1.075946in}}%
\pgfpathlineto{\pgfqpoint{2.129733in}{1.075946in}}%
\pgfpathlineto{\pgfqpoint{2.144906in}{1.070095in}}%
\pgfpathlineto{\pgfqpoint{2.160080in}{1.060602in}}%
\pgfpathlineto{\pgfqpoint{2.175254in}{1.062214in}}%
\pgfpathlineto{\pgfqpoint{2.190427in}{1.058526in}}%
\pgfpathlineto{\pgfqpoint{2.205601in}{1.048093in}}%
\pgfpathlineto{\pgfqpoint{2.220775in}{1.036268in}}%
\pgfpathlineto{\pgfqpoint{2.235948in}{1.016618in}}%
\pgfpathlineto{\pgfqpoint{2.281469in}{1.017256in}}%
\pgfpathlineto{\pgfqpoint{2.296643in}{1.008918in}}%
\pgfpathlineto{\pgfqpoint{2.311816in}{1.003161in}}%
\pgfpathlineto{\pgfqpoint{2.326990in}{1.003161in}}%
\pgfpathlineto{\pgfqpoint{2.342164in}{0.998734in}}%
\pgfpathlineto{\pgfqpoint{2.357337in}{0.989046in}}%
\pgfpathlineto{\pgfqpoint{2.372511in}{0.986150in}}%
\pgfpathlineto{\pgfqpoint{2.387685in}{0.973587in}}%
\pgfpathlineto{\pgfqpoint{2.402858in}{0.966298in}}%
\pgfpathlineto{\pgfqpoint{2.418032in}{0.966298in}}%
\pgfpathlineto{\pgfqpoint{2.448379in}{0.954957in}}%
\pgfpathlineto{\pgfqpoint{2.463553in}{0.952398in}}%
\pgfpathlineto{\pgfqpoint{2.478727in}{0.938397in}}%
\pgfpathlineto{\pgfqpoint{2.493900in}{0.938289in}}%
\pgfpathlineto{\pgfqpoint{2.524248in}{0.923610in}}%
\pgfpathlineto{\pgfqpoint{2.539421in}{0.919949in}}%
\pgfpathlineto{\pgfqpoint{2.554595in}{0.912894in}}%
\pgfpathlineto{\pgfqpoint{2.584942in}{0.912847in}}%
\pgfpathlineto{\pgfqpoint{2.600116in}{0.909616in}}%
\pgfpathlineto{\pgfqpoint{2.615290in}{0.904940in}}%
\pgfpathlineto{\pgfqpoint{2.630463in}{0.904463in}}%
\pgfpathlineto{\pgfqpoint{2.645637in}{0.895387in}}%
\pgfpathlineto{\pgfqpoint{2.675984in}{0.892720in}}%
\pgfpathlineto{\pgfqpoint{2.691158in}{0.888910in}}%
\pgfpathlineto{\pgfqpoint{2.706332in}{0.882750in}}%
\pgfpathlineto{\pgfqpoint{2.736679in}{0.879102in}}%
\pgfpathlineto{\pgfqpoint{2.751853in}{0.873673in}}%
\pgfpathlineto{\pgfqpoint{2.767027in}{0.870395in}}%
\pgfpathlineto{\pgfqpoint{2.782200in}{0.868581in}}%
\pgfpathlineto{\pgfqpoint{2.797374in}{0.863690in}}%
\pgfpathlineto{\pgfqpoint{2.812547in}{0.856380in}}%
\pgfpathlineto{\pgfqpoint{2.827721in}{0.856380in}}%
\pgfpathlineto{\pgfqpoint{2.888416in}{0.840949in}}%
\pgfpathlineto{\pgfqpoint{2.903589in}{0.838638in}}%
\pgfpathlineto{\pgfqpoint{2.918763in}{0.829413in}}%
\pgfpathlineto{\pgfqpoint{2.933937in}{0.821802in}}%
\pgfpathlineto{\pgfqpoint{2.949110in}{0.821802in}}%
\pgfpathlineto{\pgfqpoint{2.964284in}{0.815413in}}%
\pgfpathlineto{\pgfqpoint{2.979458in}{0.810253in}}%
\pgfpathlineto{\pgfqpoint{3.024979in}{0.800720in}}%
\pgfpathlineto{\pgfqpoint{3.040153in}{0.793101in}}%
\pgfpathlineto{\pgfqpoint{3.055326in}{0.792987in}}%
\pgfpathlineto{\pgfqpoint{3.100847in}{0.784126in}}%
\pgfpathlineto{\pgfqpoint{3.131194in}{0.781633in}}%
\pgfpathlineto{\pgfqpoint{3.176715in}{0.777125in}}%
\pgfpathlineto{\pgfqpoint{3.207063in}{0.775338in}}%
\pgfpathlineto{\pgfqpoint{3.252584in}{0.772920in}}%
\pgfpathlineto{\pgfqpoint{3.267757in}{0.770897in}}%
\pgfpathlineto{\pgfqpoint{3.465015in}{0.763010in}}%
\pgfpathlineto{\pgfqpoint{3.571231in}{0.762043in}}%
\pgfpathlineto{\pgfqpoint{3.616752in}{0.761559in}}%
\pgfpathlineto{\pgfqpoint{4.587867in}{0.761317in}}%
\pgfpathlineto{\pgfqpoint{4.587867in}{0.761317in}}%
\pgfusepath{stroke}%
\end{pgfscope}%
\begin{pgfscope}%
\pgfsetrectcap%
\pgfsetmiterjoin%
\pgfsetlinewidth{0.401500pt}%
\definecolor{currentstroke}{rgb}{0.000000,0.000000,0.000000}%
\pgfsetstrokecolor{currentstroke}%
\pgfsetdash{}{0pt}%
\pgfpathmoveto{\pgfqpoint{0.809624in}{0.450000in}}%
\pgfpathlineto{\pgfqpoint{0.809624in}{2.940000in}}%
\pgfusepath{stroke}%
\end{pgfscope}%
\begin{pgfscope}%
\pgfsetrectcap%
\pgfsetmiterjoin%
\pgfsetlinewidth{0.401500pt}%
\definecolor{currentstroke}{rgb}{0.000000,0.000000,0.000000}%
\pgfsetstrokecolor{currentstroke}%
\pgfsetdash{}{0pt}%
\pgfpathmoveto{\pgfqpoint{4.587867in}{0.450000in}}%
\pgfpathlineto{\pgfqpoint{4.587867in}{2.940000in}}%
\pgfusepath{stroke}%
\end{pgfscope}%
\begin{pgfscope}%
\pgfsetrectcap%
\pgfsetmiterjoin%
\pgfsetlinewidth{0.401500pt}%
\definecolor{currentstroke}{rgb}{0.000000,0.000000,0.000000}%
\pgfsetstrokecolor{currentstroke}%
\pgfsetdash{}{0pt}%
\pgfpathmoveto{\pgfqpoint{0.809624in}{0.450000in}}%
\pgfpathlineto{\pgfqpoint{4.587867in}{0.450000in}}%
\pgfusepath{stroke}%
\end{pgfscope}%
\begin{pgfscope}%
\pgfsetrectcap%
\pgfsetmiterjoin%
\pgfsetlinewidth{0.401500pt}%
\definecolor{currentstroke}{rgb}{0.000000,0.000000,0.000000}%
\pgfsetstrokecolor{currentstroke}%
\pgfsetdash{}{0pt}%
\pgfpathmoveto{\pgfqpoint{0.809624in}{2.940000in}}%
\pgfpathlineto{\pgfqpoint{4.587867in}{2.940000in}}%
\pgfusepath{stroke}%
\end{pgfscope}%
\begin{pgfscope}%
\pgfsetbuttcap%
\pgfsetmiterjoin%
\definecolor{currentfill}{rgb}{1.000000,1.000000,1.000000}%
\pgfsetfillcolor{currentfill}%
\pgfsetfillopacity{0.800000}%
\pgfsetlinewidth{1.003750pt}%
\definecolor{currentstroke}{rgb}{0.800000,0.800000,0.800000}%
\pgfsetstrokecolor{currentstroke}%
\pgfsetstrokeopacity{0.800000}%
\pgfsetdash{}{0pt}%
\pgfpathmoveto{\pgfqpoint{3.717110in}{2.247871in}}%
\pgfpathlineto{\pgfqpoint{4.490644in}{2.247871in}}%
\pgfpathquadraticcurveto{\pgfqpoint{4.518422in}{2.247871in}}{\pgfqpoint{4.518422in}{2.275648in}}%
\pgfpathlineto{\pgfqpoint{4.518422in}{2.842778in}}%
\pgfpathquadraticcurveto{\pgfqpoint{4.518422in}{2.870556in}}{\pgfqpoint{4.490644in}{2.870556in}}%
\pgfpathlineto{\pgfqpoint{3.717110in}{2.870556in}}%
\pgfpathquadraticcurveto{\pgfqpoint{3.689332in}{2.870556in}}{\pgfqpoint{3.689332in}{2.842778in}}%
\pgfpathlineto{\pgfqpoint{3.689332in}{2.275648in}}%
\pgfpathquadraticcurveto{\pgfqpoint{3.689332in}{2.247871in}}{\pgfqpoint{3.717110in}{2.247871in}}%
\pgfpathclose%
\pgfusepath{stroke,fill}%
\end{pgfscope}%
\begin{pgfscope}%
\pgfsetrectcap%
\pgfsetroundjoin%
\pgfsetlinewidth{1.003750pt}%
\definecolor{currentstroke}{rgb}{0.000000,0.000000,1.000000}%
\pgfsetstrokecolor{currentstroke}%
\pgfsetdash{}{0pt}%
\pgfpathmoveto{\pgfqpoint{3.744888in}{2.766389in}}%
\pgfpathlineto{\pgfqpoint{4.022666in}{2.766389in}}%
\pgfusepath{stroke}%
\end{pgfscope}%
\begin{pgfscope}%
\definecolor{textcolor}{rgb}{0.000000,0.000000,0.000000}%
\pgfsetstrokecolor{textcolor}%
\pgfsetfillcolor{textcolor}%
\pgftext[x=4.133777in,y=2.717778in,left,base]{\color{textcolor}\rmfamily\fontsize{10.000000}{12.000000}\selectfont worst}%
\end{pgfscope}%
\begin{pgfscope}%
\pgfsetrectcap%
\pgfsetroundjoin%
\pgfsetlinewidth{1.003750pt}%
\definecolor{currentstroke}{rgb}{0.411765,0.411765,0.411765}%
\pgfsetstrokecolor{currentstroke}%
\pgfsetdash{}{0pt}%
\pgfpathmoveto{\pgfqpoint{3.744888in}{2.572716in}}%
\pgfpathlineto{\pgfqpoint{4.022666in}{2.572716in}}%
\pgfusepath{stroke}%
\end{pgfscope}%
\begin{pgfscope}%
\definecolor{textcolor}{rgb}{0.000000,0.000000,0.000000}%
\pgfsetstrokecolor{textcolor}%
\pgfsetfillcolor{textcolor}%
\pgftext[x=4.133777in,y=2.524105in,left,base]{\color{textcolor}\rmfamily\fontsize{10.000000}{12.000000}\selectfont avg}%
\end{pgfscope}%
\begin{pgfscope}%
\pgfsetrectcap%
\pgfsetroundjoin%
\pgfsetlinewidth{1.003750pt}%
\definecolor{currentstroke}{rgb}{0.000000,0.500000,0.000000}%
\pgfsetstrokecolor{currentstroke}%
\pgfsetdash{}{0pt}%
\pgfpathmoveto{\pgfqpoint{3.744888in}{2.379043in}}%
\pgfpathlineto{\pgfqpoint{4.022666in}{2.379043in}}%
\pgfusepath{stroke}%
\end{pgfscope}%
\begin{pgfscope}%
\definecolor{textcolor}{rgb}{0.000000,0.000000,0.000000}%
\pgfsetstrokecolor{textcolor}%
\pgfsetfillcolor{textcolor}%
\pgftext[x=4.133777in,y=2.330432in,left,base]{\color{textcolor}\rmfamily\fontsize{10.000000}{12.000000}\selectfont best}%
\end{pgfscope}%
\end{pgfpicture}%
\makeatother%
\endgroup%

  \end{center}
  \caption{Przebieg algorytmu dla $P_m = 0$.}
\end{figure}
\begin{figure}[H]
  \begin{center}
    %% Creator: Matplotlib, PGF backend
%%
%% To include the figure in your LaTeX document, write
%%   \input{<filename>.pgf}
%%
%% Make sure the required packages are loaded in your preamble
%%   \usepackage{pgf}
%%
%% and, on pdftex
%%   \usepackage[utf8]{inputenc}\DeclareUnicodeCharacter{2212}{-}
%%
%% or, on luatex and xetex
%%   \usepackage{unicode-math}
%%
%% Figures using additional raster images can only be included by \input if
%% they are in the same directory as the main LaTeX file. For loading figures
%% from other directories you can use the `import` package
%%   \usepackage{import}
%%
%% and then include the figures with
%%   \import{<path to file>}{<filename>.pgf}
%%
%% Matplotlib used the following preamble
%%
\begingroup%
\makeatletter%
\begin{pgfpicture}%
\pgfpathrectangle{\pgfpointorigin}{\pgfqpoint{5.397490in}{3.000000in}}%
\pgfusepath{use as bounding box, clip}%
\begin{pgfscope}%
\pgfsetbuttcap%
\pgfsetmiterjoin%
\definecolor{currentfill}{rgb}{1.000000,1.000000,1.000000}%
\pgfsetfillcolor{currentfill}%
\pgfsetlinewidth{0.000000pt}%
\definecolor{currentstroke}{rgb}{1.000000,1.000000,1.000000}%
\pgfsetstrokecolor{currentstroke}%
\pgfsetdash{}{0pt}%
\pgfpathmoveto{\pgfqpoint{0.000000in}{0.000000in}}%
\pgfpathlineto{\pgfqpoint{5.397490in}{0.000000in}}%
\pgfpathlineto{\pgfqpoint{5.397490in}{3.000000in}}%
\pgfpathlineto{\pgfqpoint{0.000000in}{3.000000in}}%
\pgfpathclose%
\pgfusepath{fill}%
\end{pgfscope}%
\begin{pgfscope}%
\pgfsetbuttcap%
\pgfsetmiterjoin%
\definecolor{currentfill}{rgb}{1.000000,1.000000,1.000000}%
\pgfsetfillcolor{currentfill}%
\pgfsetlinewidth{0.000000pt}%
\definecolor{currentstroke}{rgb}{0.000000,0.000000,0.000000}%
\pgfsetstrokecolor{currentstroke}%
\pgfsetstrokeopacity{0.000000}%
\pgfsetdash{}{0pt}%
\pgfpathmoveto{\pgfqpoint{0.809624in}{0.450000in}}%
\pgfpathlineto{\pgfqpoint{4.587867in}{0.450000in}}%
\pgfpathlineto{\pgfqpoint{4.587867in}{2.940000in}}%
\pgfpathlineto{\pgfqpoint{0.809624in}{2.940000in}}%
\pgfpathclose%
\pgfusepath{fill}%
\end{pgfscope}%
\begin{pgfscope}%
\pgfsetbuttcap%
\pgfsetroundjoin%
\definecolor{currentfill}{rgb}{0.000000,0.000000,0.000000}%
\pgfsetfillcolor{currentfill}%
\pgfsetlinewidth{0.803000pt}%
\definecolor{currentstroke}{rgb}{0.000000,0.000000,0.000000}%
\pgfsetstrokecolor{currentstroke}%
\pgfsetdash{}{0pt}%
\pgfsys@defobject{currentmarker}{\pgfqpoint{0.000000in}{-0.048611in}}{\pgfqpoint{0.000000in}{0.000000in}}{%
\pgfpathmoveto{\pgfqpoint{0.000000in}{0.000000in}}%
\pgfpathlineto{\pgfqpoint{0.000000in}{-0.048611in}}%
\pgfusepath{stroke,fill}%
}%
\begin{pgfscope}%
\pgfsys@transformshift{0.809624in}{0.450000in}%
\pgfsys@useobject{currentmarker}{}%
\end{pgfscope}%
\end{pgfscope}%
\begin{pgfscope}%
\definecolor{textcolor}{rgb}{0.000000,0.000000,0.000000}%
\pgfsetstrokecolor{textcolor}%
\pgfsetfillcolor{textcolor}%
\pgftext[x=0.809624in,y=0.352778in,,top]{\color{textcolor}\rmfamily\fontsize{10.000000}{12.000000}\selectfont \(\displaystyle 0\)}%
\end{pgfscope}%
\begin{pgfscope}%
\pgfsetbuttcap%
\pgfsetroundjoin%
\definecolor{currentfill}{rgb}{0.000000,0.000000,0.000000}%
\pgfsetfillcolor{currentfill}%
\pgfsetlinewidth{0.803000pt}%
\definecolor{currentstroke}{rgb}{0.000000,0.000000,0.000000}%
\pgfsetstrokecolor{currentstroke}%
\pgfsetdash{}{0pt}%
\pgfsys@defobject{currentmarker}{\pgfqpoint{0.000000in}{-0.048611in}}{\pgfqpoint{0.000000in}{0.000000in}}{%
\pgfpathmoveto{\pgfqpoint{0.000000in}{0.000000in}}%
\pgfpathlineto{\pgfqpoint{0.000000in}{-0.048611in}}%
\pgfusepath{stroke,fill}%
}%
\begin{pgfscope}%
\pgfsys@transformshift{1.568307in}{0.450000in}%
\pgfsys@useobject{currentmarker}{}%
\end{pgfscope}%
\end{pgfscope}%
\begin{pgfscope}%
\definecolor{textcolor}{rgb}{0.000000,0.000000,0.000000}%
\pgfsetstrokecolor{textcolor}%
\pgfsetfillcolor{textcolor}%
\pgftext[x=1.568307in,y=0.352778in,,top]{\color{textcolor}\rmfamily\fontsize{10.000000}{12.000000}\selectfont \(\displaystyle 50\)}%
\end{pgfscope}%
\begin{pgfscope}%
\pgfsetbuttcap%
\pgfsetroundjoin%
\definecolor{currentfill}{rgb}{0.000000,0.000000,0.000000}%
\pgfsetfillcolor{currentfill}%
\pgfsetlinewidth{0.803000pt}%
\definecolor{currentstroke}{rgb}{0.000000,0.000000,0.000000}%
\pgfsetstrokecolor{currentstroke}%
\pgfsetdash{}{0pt}%
\pgfsys@defobject{currentmarker}{\pgfqpoint{0.000000in}{-0.048611in}}{\pgfqpoint{0.000000in}{0.000000in}}{%
\pgfpathmoveto{\pgfqpoint{0.000000in}{0.000000in}}%
\pgfpathlineto{\pgfqpoint{0.000000in}{-0.048611in}}%
\pgfusepath{stroke,fill}%
}%
\begin{pgfscope}%
\pgfsys@transformshift{2.326990in}{0.450000in}%
\pgfsys@useobject{currentmarker}{}%
\end{pgfscope}%
\end{pgfscope}%
\begin{pgfscope}%
\definecolor{textcolor}{rgb}{0.000000,0.000000,0.000000}%
\pgfsetstrokecolor{textcolor}%
\pgfsetfillcolor{textcolor}%
\pgftext[x=2.326990in,y=0.352778in,,top]{\color{textcolor}\rmfamily\fontsize{10.000000}{12.000000}\selectfont \(\displaystyle 100\)}%
\end{pgfscope}%
\begin{pgfscope}%
\pgfsetbuttcap%
\pgfsetroundjoin%
\definecolor{currentfill}{rgb}{0.000000,0.000000,0.000000}%
\pgfsetfillcolor{currentfill}%
\pgfsetlinewidth{0.803000pt}%
\definecolor{currentstroke}{rgb}{0.000000,0.000000,0.000000}%
\pgfsetstrokecolor{currentstroke}%
\pgfsetdash{}{0pt}%
\pgfsys@defobject{currentmarker}{\pgfqpoint{0.000000in}{-0.048611in}}{\pgfqpoint{0.000000in}{0.000000in}}{%
\pgfpathmoveto{\pgfqpoint{0.000000in}{0.000000in}}%
\pgfpathlineto{\pgfqpoint{0.000000in}{-0.048611in}}%
\pgfusepath{stroke,fill}%
}%
\begin{pgfscope}%
\pgfsys@transformshift{3.085673in}{0.450000in}%
\pgfsys@useobject{currentmarker}{}%
\end{pgfscope}%
\end{pgfscope}%
\begin{pgfscope}%
\definecolor{textcolor}{rgb}{0.000000,0.000000,0.000000}%
\pgfsetstrokecolor{textcolor}%
\pgfsetfillcolor{textcolor}%
\pgftext[x=3.085673in,y=0.352778in,,top]{\color{textcolor}\rmfamily\fontsize{10.000000}{12.000000}\selectfont \(\displaystyle 150\)}%
\end{pgfscope}%
\begin{pgfscope}%
\pgfsetbuttcap%
\pgfsetroundjoin%
\definecolor{currentfill}{rgb}{0.000000,0.000000,0.000000}%
\pgfsetfillcolor{currentfill}%
\pgfsetlinewidth{0.803000pt}%
\definecolor{currentstroke}{rgb}{0.000000,0.000000,0.000000}%
\pgfsetstrokecolor{currentstroke}%
\pgfsetdash{}{0pt}%
\pgfsys@defobject{currentmarker}{\pgfqpoint{0.000000in}{-0.048611in}}{\pgfqpoint{0.000000in}{0.000000in}}{%
\pgfpathmoveto{\pgfqpoint{0.000000in}{0.000000in}}%
\pgfpathlineto{\pgfqpoint{0.000000in}{-0.048611in}}%
\pgfusepath{stroke,fill}%
}%
\begin{pgfscope}%
\pgfsys@transformshift{3.844357in}{0.450000in}%
\pgfsys@useobject{currentmarker}{}%
\end{pgfscope}%
\end{pgfscope}%
\begin{pgfscope}%
\definecolor{textcolor}{rgb}{0.000000,0.000000,0.000000}%
\pgfsetstrokecolor{textcolor}%
\pgfsetfillcolor{textcolor}%
\pgftext[x=3.844357in,y=0.352778in,,top]{\color{textcolor}\rmfamily\fontsize{10.000000}{12.000000}\selectfont \(\displaystyle 200\)}%
\end{pgfscope}%
\begin{pgfscope}%
\definecolor{textcolor}{rgb}{0.000000,0.000000,0.000000}%
\pgfsetstrokecolor{textcolor}%
\pgfsetfillcolor{textcolor}%
\pgftext[x=2.698745in,y=0.173766in,,top]{\color{textcolor}\rmfamily\fontsize{10.000000}{12.000000}\selectfont Nr pokolenia}%
\end{pgfscope}%
\begin{pgfscope}%
\pgfsetbuttcap%
\pgfsetroundjoin%
\definecolor{currentfill}{rgb}{0.000000,0.000000,0.000000}%
\pgfsetfillcolor{currentfill}%
\pgfsetlinewidth{0.803000pt}%
\definecolor{currentstroke}{rgb}{0.000000,0.000000,0.000000}%
\pgfsetstrokecolor{currentstroke}%
\pgfsetdash{}{0pt}%
\pgfsys@defobject{currentmarker}{\pgfqpoint{-0.048611in}{0.000000in}}{\pgfqpoint{0.000000in}{0.000000in}}{%
\pgfpathmoveto{\pgfqpoint{0.000000in}{0.000000in}}%
\pgfpathlineto{\pgfqpoint{-0.048611in}{0.000000in}}%
\pgfusepath{stroke,fill}%
}%
\begin{pgfscope}%
\pgfsys@transformshift{0.809624in}{0.588611in}%
\pgfsys@useobject{currentmarker}{}%
\end{pgfscope}%
\end{pgfscope}%
\begin{pgfscope}%
\definecolor{textcolor}{rgb}{0.000000,0.000000,0.000000}%
\pgfsetstrokecolor{textcolor}%
\pgfsetfillcolor{textcolor}%
\pgftext[x=0.365178in, y=0.540386in, left, base]{\color{textcolor}\rmfamily\fontsize{10.000000}{12.000000}\selectfont \(\displaystyle 50000\)}%
\end{pgfscope}%
\begin{pgfscope}%
\pgfsetbuttcap%
\pgfsetroundjoin%
\definecolor{currentfill}{rgb}{0.000000,0.000000,0.000000}%
\pgfsetfillcolor{currentfill}%
\pgfsetlinewidth{0.803000pt}%
\definecolor{currentstroke}{rgb}{0.000000,0.000000,0.000000}%
\pgfsetstrokecolor{currentstroke}%
\pgfsetdash{}{0pt}%
\pgfsys@defobject{currentmarker}{\pgfqpoint{-0.048611in}{0.000000in}}{\pgfqpoint{0.000000in}{0.000000in}}{%
\pgfpathmoveto{\pgfqpoint{0.000000in}{0.000000in}}%
\pgfpathlineto{\pgfqpoint{-0.048611in}{0.000000in}}%
\pgfusepath{stroke,fill}%
}%
\begin{pgfscope}%
\pgfsys@transformshift{0.809624in}{0.924524in}%
\pgfsys@useobject{currentmarker}{}%
\end{pgfscope}%
\end{pgfscope}%
\begin{pgfscope}%
\definecolor{textcolor}{rgb}{0.000000,0.000000,0.000000}%
\pgfsetstrokecolor{textcolor}%
\pgfsetfillcolor{textcolor}%
\pgftext[x=0.295733in, y=0.876298in, left, base]{\color{textcolor}\rmfamily\fontsize{10.000000}{12.000000}\selectfont \(\displaystyle 100000\)}%
\end{pgfscope}%
\begin{pgfscope}%
\pgfsetbuttcap%
\pgfsetroundjoin%
\definecolor{currentfill}{rgb}{0.000000,0.000000,0.000000}%
\pgfsetfillcolor{currentfill}%
\pgfsetlinewidth{0.803000pt}%
\definecolor{currentstroke}{rgb}{0.000000,0.000000,0.000000}%
\pgfsetstrokecolor{currentstroke}%
\pgfsetdash{}{0pt}%
\pgfsys@defobject{currentmarker}{\pgfqpoint{-0.048611in}{0.000000in}}{\pgfqpoint{0.000000in}{0.000000in}}{%
\pgfpathmoveto{\pgfqpoint{0.000000in}{0.000000in}}%
\pgfpathlineto{\pgfqpoint{-0.048611in}{0.000000in}}%
\pgfusepath{stroke,fill}%
}%
\begin{pgfscope}%
\pgfsys@transformshift{0.809624in}{1.260436in}%
\pgfsys@useobject{currentmarker}{}%
\end{pgfscope}%
\end{pgfscope}%
\begin{pgfscope}%
\definecolor{textcolor}{rgb}{0.000000,0.000000,0.000000}%
\pgfsetstrokecolor{textcolor}%
\pgfsetfillcolor{textcolor}%
\pgftext[x=0.295733in, y=1.212211in, left, base]{\color{textcolor}\rmfamily\fontsize{10.000000}{12.000000}\selectfont \(\displaystyle 150000\)}%
\end{pgfscope}%
\begin{pgfscope}%
\pgfsetbuttcap%
\pgfsetroundjoin%
\definecolor{currentfill}{rgb}{0.000000,0.000000,0.000000}%
\pgfsetfillcolor{currentfill}%
\pgfsetlinewidth{0.803000pt}%
\definecolor{currentstroke}{rgb}{0.000000,0.000000,0.000000}%
\pgfsetstrokecolor{currentstroke}%
\pgfsetdash{}{0pt}%
\pgfsys@defobject{currentmarker}{\pgfqpoint{-0.048611in}{0.000000in}}{\pgfqpoint{0.000000in}{0.000000in}}{%
\pgfpathmoveto{\pgfqpoint{0.000000in}{0.000000in}}%
\pgfpathlineto{\pgfqpoint{-0.048611in}{0.000000in}}%
\pgfusepath{stroke,fill}%
}%
\begin{pgfscope}%
\pgfsys@transformshift{0.809624in}{1.596349in}%
\pgfsys@useobject{currentmarker}{}%
\end{pgfscope}%
\end{pgfscope}%
\begin{pgfscope}%
\definecolor{textcolor}{rgb}{0.000000,0.000000,0.000000}%
\pgfsetstrokecolor{textcolor}%
\pgfsetfillcolor{textcolor}%
\pgftext[x=0.295733in, y=1.548124in, left, base]{\color{textcolor}\rmfamily\fontsize{10.000000}{12.000000}\selectfont \(\displaystyle 200000\)}%
\end{pgfscope}%
\begin{pgfscope}%
\pgfsetbuttcap%
\pgfsetroundjoin%
\definecolor{currentfill}{rgb}{0.000000,0.000000,0.000000}%
\pgfsetfillcolor{currentfill}%
\pgfsetlinewidth{0.803000pt}%
\definecolor{currentstroke}{rgb}{0.000000,0.000000,0.000000}%
\pgfsetstrokecolor{currentstroke}%
\pgfsetdash{}{0pt}%
\pgfsys@defobject{currentmarker}{\pgfqpoint{-0.048611in}{0.000000in}}{\pgfqpoint{0.000000in}{0.000000in}}{%
\pgfpathmoveto{\pgfqpoint{0.000000in}{0.000000in}}%
\pgfpathlineto{\pgfqpoint{-0.048611in}{0.000000in}}%
\pgfusepath{stroke,fill}%
}%
\begin{pgfscope}%
\pgfsys@transformshift{0.809624in}{1.932262in}%
\pgfsys@useobject{currentmarker}{}%
\end{pgfscope}%
\end{pgfscope}%
\begin{pgfscope}%
\definecolor{textcolor}{rgb}{0.000000,0.000000,0.000000}%
\pgfsetstrokecolor{textcolor}%
\pgfsetfillcolor{textcolor}%
\pgftext[x=0.295733in, y=1.884037in, left, base]{\color{textcolor}\rmfamily\fontsize{10.000000}{12.000000}\selectfont \(\displaystyle 250000\)}%
\end{pgfscope}%
\begin{pgfscope}%
\pgfsetbuttcap%
\pgfsetroundjoin%
\definecolor{currentfill}{rgb}{0.000000,0.000000,0.000000}%
\pgfsetfillcolor{currentfill}%
\pgfsetlinewidth{0.803000pt}%
\definecolor{currentstroke}{rgb}{0.000000,0.000000,0.000000}%
\pgfsetstrokecolor{currentstroke}%
\pgfsetdash{}{0pt}%
\pgfsys@defobject{currentmarker}{\pgfqpoint{-0.048611in}{0.000000in}}{\pgfqpoint{0.000000in}{0.000000in}}{%
\pgfpathmoveto{\pgfqpoint{0.000000in}{0.000000in}}%
\pgfpathlineto{\pgfqpoint{-0.048611in}{0.000000in}}%
\pgfusepath{stroke,fill}%
}%
\begin{pgfscope}%
\pgfsys@transformshift{0.809624in}{2.268175in}%
\pgfsys@useobject{currentmarker}{}%
\end{pgfscope}%
\end{pgfscope}%
\begin{pgfscope}%
\definecolor{textcolor}{rgb}{0.000000,0.000000,0.000000}%
\pgfsetstrokecolor{textcolor}%
\pgfsetfillcolor{textcolor}%
\pgftext[x=0.295733in, y=2.219949in, left, base]{\color{textcolor}\rmfamily\fontsize{10.000000}{12.000000}\selectfont \(\displaystyle 300000\)}%
\end{pgfscope}%
\begin{pgfscope}%
\pgfsetbuttcap%
\pgfsetroundjoin%
\definecolor{currentfill}{rgb}{0.000000,0.000000,0.000000}%
\pgfsetfillcolor{currentfill}%
\pgfsetlinewidth{0.803000pt}%
\definecolor{currentstroke}{rgb}{0.000000,0.000000,0.000000}%
\pgfsetstrokecolor{currentstroke}%
\pgfsetdash{}{0pt}%
\pgfsys@defobject{currentmarker}{\pgfqpoint{-0.048611in}{0.000000in}}{\pgfqpoint{0.000000in}{0.000000in}}{%
\pgfpathmoveto{\pgfqpoint{0.000000in}{0.000000in}}%
\pgfpathlineto{\pgfqpoint{-0.048611in}{0.000000in}}%
\pgfusepath{stroke,fill}%
}%
\begin{pgfscope}%
\pgfsys@transformshift{0.809624in}{2.604087in}%
\pgfsys@useobject{currentmarker}{}%
\end{pgfscope}%
\end{pgfscope}%
\begin{pgfscope}%
\definecolor{textcolor}{rgb}{0.000000,0.000000,0.000000}%
\pgfsetstrokecolor{textcolor}%
\pgfsetfillcolor{textcolor}%
\pgftext[x=0.295733in, y=2.555862in, left, base]{\color{textcolor}\rmfamily\fontsize{10.000000}{12.000000}\selectfont \(\displaystyle 350000\)}%
\end{pgfscope}%
\begin{pgfscope}%
\pgfsetbuttcap%
\pgfsetroundjoin%
\definecolor{currentfill}{rgb}{0.000000,0.000000,0.000000}%
\pgfsetfillcolor{currentfill}%
\pgfsetlinewidth{0.803000pt}%
\definecolor{currentstroke}{rgb}{0.000000,0.000000,0.000000}%
\pgfsetstrokecolor{currentstroke}%
\pgfsetdash{}{0pt}%
\pgfsys@defobject{currentmarker}{\pgfqpoint{-0.048611in}{0.000000in}}{\pgfqpoint{0.000000in}{0.000000in}}{%
\pgfpathmoveto{\pgfqpoint{0.000000in}{0.000000in}}%
\pgfpathlineto{\pgfqpoint{-0.048611in}{0.000000in}}%
\pgfusepath{stroke,fill}%
}%
\begin{pgfscope}%
\pgfsys@transformshift{0.809624in}{2.940000in}%
\pgfsys@useobject{currentmarker}{}%
\end{pgfscope}%
\end{pgfscope}%
\begin{pgfscope}%
\definecolor{textcolor}{rgb}{0.000000,0.000000,0.000000}%
\pgfsetstrokecolor{textcolor}%
\pgfsetfillcolor{textcolor}%
\pgftext[x=0.295733in, y=2.891775in, left, base]{\color{textcolor}\rmfamily\fontsize{10.000000}{12.000000}\selectfont \(\displaystyle 400000\)}%
\end{pgfscope}%
\begin{pgfscope}%
\definecolor{textcolor}{rgb}{0.000000,0.000000,0.000000}%
\pgfsetstrokecolor{textcolor}%
\pgfsetfillcolor{textcolor}%
\pgftext[x=0.240178in,y=1.695000in,,bottom,rotate=90.000000]{\color{textcolor}\rmfamily\fontsize{10.000000}{12.000000}\selectfont Wartość funkcji przystosowania}%
\end{pgfscope}%
\begin{pgfscope}%
\pgfpathrectangle{\pgfqpoint{0.809624in}{0.450000in}}{\pgfqpoint{3.778243in}{2.490000in}}%
\pgfusepath{clip}%
\pgfsetrectcap%
\pgfsetroundjoin%
\pgfsetlinewidth{1.003750pt}%
\definecolor{currentstroke}{rgb}{0.000000,0.000000,1.000000}%
\pgfsetstrokecolor{currentstroke}%
\pgfsetdash{}{0pt}%
\pgfpathmoveto{\pgfqpoint{0.809624in}{2.793730in}}%
\pgfpathlineto{\pgfqpoint{0.824797in}{2.692667in}}%
\pgfpathlineto{\pgfqpoint{0.839971in}{2.660467in}}%
\pgfpathlineto{\pgfqpoint{0.855144in}{2.649583in}}%
\pgfpathlineto{\pgfqpoint{0.870318in}{2.604363in}}%
\pgfpathlineto{\pgfqpoint{0.885492in}{2.657954in}}%
\pgfpathlineto{\pgfqpoint{0.900666in}{2.531927in}}%
\pgfpathlineto{\pgfqpoint{0.915839in}{2.512692in}}%
\pgfpathlineto{\pgfqpoint{0.931013in}{2.518860in}}%
\pgfpathlineto{\pgfqpoint{0.946187in}{2.532881in}}%
\pgfpathlineto{\pgfqpoint{0.961360in}{2.480438in}}%
\pgfpathlineto{\pgfqpoint{0.976534in}{2.494459in}}%
\pgfpathlineto{\pgfqpoint{0.991707in}{2.485819in}}%
\pgfpathlineto{\pgfqpoint{1.006881in}{2.414962in}}%
\pgfpathlineto{\pgfqpoint{1.022055in}{2.430729in}}%
\pgfpathlineto{\pgfqpoint{1.037228in}{2.405173in}}%
\pgfpathlineto{\pgfqpoint{1.052402in}{2.388599in}}%
\pgfpathlineto{\pgfqpoint{1.067576in}{2.361028in}}%
\pgfpathlineto{\pgfqpoint{1.082750in}{2.383037in}}%
\pgfpathlineto{\pgfqpoint{1.113097in}{2.347457in}}%
\pgfpathlineto{\pgfqpoint{1.128270in}{2.352563in}}%
\pgfpathlineto{\pgfqpoint{1.143444in}{2.342868in}}%
\pgfpathlineto{\pgfqpoint{1.173791in}{2.355196in}}%
\pgfpathlineto{\pgfqpoint{1.188965in}{2.331474in}}%
\pgfpathlineto{\pgfqpoint{1.204139in}{2.318501in}}%
\pgfpathlineto{\pgfqpoint{1.219313in}{2.315283in}}%
\pgfpathlineto{\pgfqpoint{1.234486in}{2.308330in}}%
\pgfpathlineto{\pgfqpoint{1.249660in}{2.326294in}}%
\pgfpathlineto{\pgfqpoint{1.264833in}{2.288833in}}%
\pgfpathlineto{\pgfqpoint{1.280007in}{2.319468in}}%
\pgfpathlineto{\pgfqpoint{1.295181in}{2.296210in}}%
\pgfpathlineto{\pgfqpoint{1.310355in}{2.275995in}}%
\pgfpathlineto{\pgfqpoint{1.340702in}{2.280973in}}%
\pgfpathlineto{\pgfqpoint{1.355876in}{2.265796in}}%
\pgfpathlineto{\pgfqpoint{1.371049in}{2.253596in}}%
\pgfpathlineto{\pgfqpoint{1.386223in}{2.275504in}}%
\pgfpathlineto{\pgfqpoint{1.401396in}{2.286099in}}%
\pgfpathlineto{\pgfqpoint{1.416570in}{2.226877in}}%
\pgfpathlineto{\pgfqpoint{1.431744in}{2.246784in}}%
\pgfpathlineto{\pgfqpoint{1.446917in}{2.222255in}}%
\pgfpathlineto{\pgfqpoint{1.462091in}{2.267241in}}%
\pgfpathlineto{\pgfqpoint{1.477265in}{2.273106in}}%
\pgfpathlineto{\pgfqpoint{1.492439in}{2.241604in}}%
\pgfpathlineto{\pgfqpoint{1.507612in}{2.228544in}}%
\pgfpathlineto{\pgfqpoint{1.522786in}{2.209323in}}%
\pgfpathlineto{\pgfqpoint{1.537960in}{2.292938in}}%
\pgfpathlineto{\pgfqpoint{1.553133in}{2.235692in}}%
\pgfpathlineto{\pgfqpoint{1.568307in}{2.219521in}}%
\pgfpathlineto{\pgfqpoint{1.583481in}{2.247845in}}%
\pgfpathlineto{\pgfqpoint{1.598654in}{2.215436in}}%
\pgfpathlineto{\pgfqpoint{1.613828in}{2.205332in}}%
\pgfpathlineto{\pgfqpoint{1.629001in}{2.214744in}}%
\pgfpathlineto{\pgfqpoint{1.644175in}{2.185439in}}%
\pgfpathlineto{\pgfqpoint{1.659349in}{2.181919in}}%
\pgfpathlineto{\pgfqpoint{1.674522in}{2.183377in}}%
\pgfpathlineto{\pgfqpoint{1.689696in}{2.178882in}}%
\pgfpathlineto{\pgfqpoint{1.704870in}{2.238023in}}%
\pgfpathlineto{\pgfqpoint{1.720043in}{2.169544in}}%
\pgfpathlineto{\pgfqpoint{1.735217in}{2.215053in}}%
\pgfpathlineto{\pgfqpoint{1.750391in}{2.186890in}}%
\pgfpathlineto{\pgfqpoint{1.765564in}{2.204613in}}%
\pgfpathlineto{\pgfqpoint{1.780738in}{2.164257in}}%
\pgfpathlineto{\pgfqpoint{1.795912in}{2.153964in}}%
\pgfpathlineto{\pgfqpoint{1.811085in}{2.168066in}}%
\pgfpathlineto{\pgfqpoint{1.826259in}{2.157471in}}%
\pgfpathlineto{\pgfqpoint{1.841433in}{2.227999in}}%
\pgfpathlineto{\pgfqpoint{1.856606in}{2.199561in}}%
\pgfpathlineto{\pgfqpoint{1.871780in}{2.216498in}}%
\pgfpathlineto{\pgfqpoint{1.886954in}{2.191163in}}%
\pgfpathlineto{\pgfqpoint{1.902127in}{2.148791in}}%
\pgfpathlineto{\pgfqpoint{1.917301in}{2.177821in}}%
\pgfpathlineto{\pgfqpoint{1.932475in}{2.136981in}}%
\pgfpathlineto{\pgfqpoint{1.947648in}{2.178808in}}%
\pgfpathlineto{\pgfqpoint{1.962822in}{2.159715in}}%
\pgfpathlineto{\pgfqpoint{1.977996in}{2.153575in}}%
\pgfpathlineto{\pgfqpoint{1.993170in}{2.168207in}}%
\pgfpathlineto{\pgfqpoint{2.008343in}{2.176813in}}%
\pgfpathlineto{\pgfqpoint{2.023517in}{2.138647in}}%
\pgfpathlineto{\pgfqpoint{2.038690in}{2.175899in}}%
\pgfpathlineto{\pgfqpoint{2.053864in}{2.182147in}}%
\pgfpathlineto{\pgfqpoint{2.069038in}{2.157498in}}%
\pgfpathlineto{\pgfqpoint{2.084211in}{2.165822in}}%
\pgfpathlineto{\pgfqpoint{2.099385in}{2.156940in}}%
\pgfpathlineto{\pgfqpoint{2.114559in}{2.169201in}}%
\pgfpathlineto{\pgfqpoint{2.129733in}{2.172171in}}%
\pgfpathlineto{\pgfqpoint{2.144906in}{2.122731in}}%
\pgfpathlineto{\pgfqpoint{2.160080in}{2.133857in}}%
\pgfpathlineto{\pgfqpoint{2.190427in}{2.143161in}}%
\pgfpathlineto{\pgfqpoint{2.205601in}{2.153958in}}%
\pgfpathlineto{\pgfqpoint{2.220775in}{2.152466in}}%
\pgfpathlineto{\pgfqpoint{2.235948in}{2.136954in}}%
\pgfpathlineto{\pgfqpoint{2.251122in}{2.138512in}}%
\pgfpathlineto{\pgfqpoint{2.266296in}{2.178110in}}%
\pgfpathlineto{\pgfqpoint{2.281469in}{2.142758in}}%
\pgfpathlineto{\pgfqpoint{2.296643in}{2.156430in}}%
\pgfpathlineto{\pgfqpoint{2.311816in}{2.114998in}}%
\pgfpathlineto{\pgfqpoint{2.326990in}{2.137921in}}%
\pgfpathlineto{\pgfqpoint{2.342164in}{2.149752in}}%
\pgfpathlineto{\pgfqpoint{2.357337in}{2.137753in}}%
\pgfpathlineto{\pgfqpoint{2.372511in}{2.106829in}}%
\pgfpathlineto{\pgfqpoint{2.387685in}{2.140044in}}%
\pgfpathlineto{\pgfqpoint{2.402858in}{2.122026in}}%
\pgfpathlineto{\pgfqpoint{2.418032in}{2.121361in}}%
\pgfpathlineto{\pgfqpoint{2.433206in}{2.149248in}}%
\pgfpathlineto{\pgfqpoint{2.448379in}{2.167804in}}%
\pgfpathlineto{\pgfqpoint{2.463553in}{2.122839in}}%
\pgfpathlineto{\pgfqpoint{2.478727in}{2.120575in}}%
\pgfpathlineto{\pgfqpoint{2.493900in}{2.142550in}}%
\pgfpathlineto{\pgfqpoint{2.509074in}{2.111209in}}%
\pgfpathlineto{\pgfqpoint{2.524248in}{2.143410in}}%
\pgfpathlineto{\pgfqpoint{2.539421in}{2.105559in}}%
\pgfpathlineto{\pgfqpoint{2.554595in}{2.195261in}}%
\pgfpathlineto{\pgfqpoint{2.569769in}{2.152533in}}%
\pgfpathlineto{\pgfqpoint{2.584942in}{2.130256in}}%
\pgfpathlineto{\pgfqpoint{2.600116in}{2.122671in}}%
\pgfpathlineto{\pgfqpoint{2.615290in}{2.105794in}}%
\pgfpathlineto{\pgfqpoint{2.630463in}{2.130114in}}%
\pgfpathlineto{\pgfqpoint{2.645637in}{2.086238in}}%
\pgfpathlineto{\pgfqpoint{2.660811in}{2.110289in}}%
\pgfpathlineto{\pgfqpoint{2.675984in}{2.142039in}}%
\pgfpathlineto{\pgfqpoint{2.691158in}{2.138801in}}%
\pgfpathlineto{\pgfqpoint{2.706332in}{2.102946in}}%
\pgfpathlineto{\pgfqpoint{2.721506in}{2.104108in}}%
\pgfpathlineto{\pgfqpoint{2.736679in}{2.091552in}}%
\pgfpathlineto{\pgfqpoint{2.751853in}{2.096496in}}%
\pgfpathlineto{\pgfqpoint{2.767027in}{2.103833in}}%
\pgfpathlineto{\pgfqpoint{2.782200in}{2.132217in}}%
\pgfpathlineto{\pgfqpoint{2.797374in}{2.100742in}}%
\pgfpathlineto{\pgfqpoint{2.812547in}{2.096510in}}%
\pgfpathlineto{\pgfqpoint{2.827721in}{2.130706in}}%
\pgfpathlineto{\pgfqpoint{2.842895in}{2.092163in}}%
\pgfpathlineto{\pgfqpoint{2.858068in}{2.227422in}}%
\pgfpathlineto{\pgfqpoint{2.873242in}{2.129960in}}%
\pgfpathlineto{\pgfqpoint{2.888416in}{2.120487in}}%
\pgfpathlineto{\pgfqpoint{2.903589in}{2.118532in}}%
\pgfpathlineto{\pgfqpoint{2.918763in}{2.082583in}}%
\pgfpathlineto{\pgfqpoint{2.933937in}{2.118149in}}%
\pgfpathlineto{\pgfqpoint{2.949110in}{2.186286in}}%
\pgfpathlineto{\pgfqpoint{2.964284in}{2.090719in}}%
\pgfpathlineto{\pgfqpoint{2.979458in}{2.084538in}}%
\pgfpathlineto{\pgfqpoint{2.994632in}{2.130827in}}%
\pgfpathlineto{\pgfqpoint{3.009805in}{2.101515in}}%
\pgfpathlineto{\pgfqpoint{3.024979in}{2.069428in}}%
\pgfpathlineto{\pgfqpoint{3.040153in}{2.091552in}}%
\pgfpathlineto{\pgfqpoint{3.055326in}{2.110511in}}%
\pgfpathlineto{\pgfqpoint{3.070500in}{2.098492in}}%
\pgfpathlineto{\pgfqpoint{3.085673in}{2.141153in}}%
\pgfpathlineto{\pgfqpoint{3.100847in}{2.117659in}}%
\pgfpathlineto{\pgfqpoint{3.116021in}{2.088125in}}%
\pgfpathlineto{\pgfqpoint{3.131194in}{2.076395in}}%
\pgfpathlineto{\pgfqpoint{3.146368in}{2.124048in}}%
\pgfpathlineto{\pgfqpoint{3.161542in}{2.144310in}}%
\pgfpathlineto{\pgfqpoint{3.176715in}{2.120541in}}%
\pgfpathlineto{\pgfqpoint{3.191889in}{2.121730in}}%
\pgfpathlineto{\pgfqpoint{3.207063in}{2.111209in}}%
\pgfpathlineto{\pgfqpoint{3.222236in}{2.132661in}}%
\pgfpathlineto{\pgfqpoint{3.237410in}{2.103671in}}%
\pgfpathlineto{\pgfqpoint{3.252584in}{2.090416in}}%
\pgfpathlineto{\pgfqpoint{3.267757in}{2.142584in}}%
\pgfpathlineto{\pgfqpoint{3.282931in}{2.118136in}}%
\pgfpathlineto{\pgfqpoint{3.313278in}{2.120749in}}%
\pgfpathlineto{\pgfqpoint{3.328452in}{2.109745in}}%
\pgfpathlineto{\pgfqpoint{3.343626in}{2.146594in}}%
\pgfpathlineto{\pgfqpoint{3.358799in}{2.114333in}}%
\pgfpathlineto{\pgfqpoint{3.373973in}{2.153286in}}%
\pgfpathlineto{\pgfqpoint{3.389147in}{2.130175in}}%
\pgfpathlineto{\pgfqpoint{3.404321in}{2.072351in}}%
\pgfpathlineto{\pgfqpoint{3.419494in}{2.144552in}}%
\pgfpathlineto{\pgfqpoint{3.434668in}{2.099325in}}%
\pgfpathlineto{\pgfqpoint{3.449842in}{2.088488in}}%
\pgfpathlineto{\pgfqpoint{3.465015in}{2.091122in}}%
\pgfpathlineto{\pgfqpoint{3.480189in}{2.100366in}}%
\pgfpathlineto{\pgfqpoint{3.495363in}{2.058175in}}%
\pgfpathlineto{\pgfqpoint{3.540884in}{2.116423in}}%
\pgfpathlineto{\pgfqpoint{3.556057in}{2.133212in}}%
\pgfpathlineto{\pgfqpoint{3.571231in}{2.115173in}}%
\pgfpathlineto{\pgfqpoint{3.586404in}{2.111491in}}%
\pgfpathlineto{\pgfqpoint{3.601578in}{2.097961in}}%
\pgfpathlineto{\pgfqpoint{3.616752in}{2.130840in}}%
\pgfpathlineto{\pgfqpoint{3.631925in}{2.077450in}}%
\pgfpathlineto{\pgfqpoint{3.647099in}{2.095576in}}%
\pgfpathlineto{\pgfqpoint{3.662273in}{2.104464in}}%
\pgfpathlineto{\pgfqpoint{3.677446in}{2.130088in}}%
\pgfpathlineto{\pgfqpoint{3.692620in}{2.080191in}}%
\pgfpathlineto{\pgfqpoint{3.707794in}{2.127669in}}%
\pgfpathlineto{\pgfqpoint{3.722967in}{2.109402in}}%
\pgfpathlineto{\pgfqpoint{3.738141in}{2.096113in}}%
\pgfpathlineto{\pgfqpoint{3.753315in}{2.120098in}}%
\pgfpathlineto{\pgfqpoint{3.768488in}{2.161287in}}%
\pgfpathlineto{\pgfqpoint{3.783662in}{2.208745in}}%
\pgfpathlineto{\pgfqpoint{3.798836in}{2.101985in}}%
\pgfpathlineto{\pgfqpoint{3.814009in}{2.115435in}}%
\pgfpathlineto{\pgfqpoint{3.829183in}{2.121958in}}%
\pgfpathlineto{\pgfqpoint{3.844357in}{2.092042in}}%
\pgfpathlineto{\pgfqpoint{3.859530in}{2.166601in}}%
\pgfpathlineto{\pgfqpoint{3.889878in}{2.113460in}}%
\pgfpathlineto{\pgfqpoint{3.905052in}{2.079560in}}%
\pgfpathlineto{\pgfqpoint{3.920225in}{2.173931in}}%
\pgfpathlineto{\pgfqpoint{3.935399in}{2.096342in}}%
\pgfpathlineto{\pgfqpoint{3.950573in}{2.095919in}}%
\pgfpathlineto{\pgfqpoint{3.965746in}{2.143000in}}%
\pgfpathlineto{\pgfqpoint{3.980920in}{2.109543in}}%
\pgfpathlineto{\pgfqpoint{3.996094in}{2.112405in}}%
\pgfpathlineto{\pgfqpoint{4.011267in}{2.085129in}}%
\pgfpathlineto{\pgfqpoint{4.026441in}{2.234523in}}%
\pgfpathlineto{\pgfqpoint{4.041614in}{2.103853in}}%
\pgfpathlineto{\pgfqpoint{4.056788in}{2.087017in}}%
\pgfpathlineto{\pgfqpoint{4.071962in}{2.154697in}}%
\pgfpathlineto{\pgfqpoint{4.087136in}{2.097390in}}%
\pgfpathlineto{\pgfqpoint{4.102309in}{2.124679in}}%
\pgfpathlineto{\pgfqpoint{4.117483in}{2.118774in}}%
\pgfpathlineto{\pgfqpoint{4.132657in}{2.093621in}}%
\pgfpathlineto{\pgfqpoint{4.147830in}{2.199998in}}%
\pgfpathlineto{\pgfqpoint{4.163004in}{2.133359in}}%
\pgfpathlineto{\pgfqpoint{4.178178in}{2.117760in}}%
\pgfpathlineto{\pgfqpoint{4.193351in}{2.170726in}}%
\pgfpathlineto{\pgfqpoint{4.208525in}{2.104316in}}%
\pgfpathlineto{\pgfqpoint{4.223699in}{2.107467in}}%
\pgfpathlineto{\pgfqpoint{4.254046in}{2.110497in}}%
\pgfpathlineto{\pgfqpoint{4.269220in}{2.100628in}}%
\pgfpathlineto{\pgfqpoint{4.284393in}{2.163074in}}%
\pgfpathlineto{\pgfqpoint{4.299567in}{2.120151in}}%
\pgfpathlineto{\pgfqpoint{4.314741in}{2.103356in}}%
\pgfpathlineto{\pgfqpoint{4.329914in}{2.172238in}}%
\pgfpathlineto{\pgfqpoint{4.345088in}{2.164263in}}%
\pgfpathlineto{\pgfqpoint{4.360262in}{2.095710in}}%
\pgfpathlineto{\pgfqpoint{4.375435in}{2.106943in}}%
\pgfpathlineto{\pgfqpoint{4.390609in}{2.077376in}}%
\pgfpathlineto{\pgfqpoint{4.405782in}{2.127696in}}%
\pgfpathlineto{\pgfqpoint{4.420956in}{2.117827in}}%
\pgfpathlineto{\pgfqpoint{4.436130in}{2.115092in}}%
\pgfpathlineto{\pgfqpoint{4.451303in}{2.110806in}}%
\pgfpathlineto{\pgfqpoint{4.466477in}{2.126843in}}%
\pgfpathlineto{\pgfqpoint{4.481651in}{2.131391in}}%
\pgfpathlineto{\pgfqpoint{4.496824in}{2.093446in}}%
\pgfpathlineto{\pgfqpoint{4.511998in}{2.082146in}}%
\pgfpathlineto{\pgfqpoint{4.542345in}{2.125687in}}%
\pgfpathlineto{\pgfqpoint{4.557519in}{2.092553in}}%
\pgfpathlineto{\pgfqpoint{4.572693in}{2.120306in}}%
\pgfpathlineto{\pgfqpoint{4.587867in}{2.058363in}}%
\pgfpathlineto{\pgfqpoint{4.587867in}{2.058363in}}%
\pgfusepath{stroke}%
\end{pgfscope}%
\begin{pgfscope}%
\pgfpathrectangle{\pgfqpoint{0.809624in}{0.450000in}}{\pgfqpoint{3.778243in}{2.490000in}}%
\pgfusepath{clip}%
\pgfsetrectcap%
\pgfsetroundjoin%
\pgfsetlinewidth{1.003750pt}%
\definecolor{currentstroke}{rgb}{0.411765,0.411765,0.411765}%
\pgfsetstrokecolor{currentstroke}%
\pgfsetdash{}{0pt}%
\pgfpathmoveto{\pgfqpoint{0.809624in}{2.196272in}}%
\pgfpathlineto{\pgfqpoint{0.824797in}{2.155509in}}%
\pgfpathlineto{\pgfqpoint{0.839971in}{2.130457in}}%
\pgfpathlineto{\pgfqpoint{0.855144in}{2.093269in}}%
\pgfpathlineto{\pgfqpoint{0.870318in}{2.065628in}}%
\pgfpathlineto{\pgfqpoint{0.885492in}{2.026414in}}%
\pgfpathlineto{\pgfqpoint{0.915839in}{1.990177in}}%
\pgfpathlineto{\pgfqpoint{0.931013in}{1.965427in}}%
\pgfpathlineto{\pgfqpoint{0.946187in}{1.938287in}}%
\pgfpathlineto{\pgfqpoint{0.961360in}{1.930479in}}%
\pgfpathlineto{\pgfqpoint{0.976534in}{1.910922in}}%
\pgfpathlineto{\pgfqpoint{1.006881in}{1.884435in}}%
\pgfpathlineto{\pgfqpoint{1.022055in}{1.875640in}}%
\pgfpathlineto{\pgfqpoint{1.037228in}{1.859188in}}%
\pgfpathlineto{\pgfqpoint{1.067576in}{1.843056in}}%
\pgfpathlineto{\pgfqpoint{1.097923in}{1.821143in}}%
\pgfpathlineto{\pgfqpoint{1.113097in}{1.823861in}}%
\pgfpathlineto{\pgfqpoint{1.128270in}{1.810025in}}%
\pgfpathlineto{\pgfqpoint{1.143444in}{1.786867in}}%
\pgfpathlineto{\pgfqpoint{1.158618in}{1.779833in}}%
\pgfpathlineto{\pgfqpoint{1.173791in}{1.778623in}}%
\pgfpathlineto{\pgfqpoint{1.188965in}{1.767027in}}%
\pgfpathlineto{\pgfqpoint{1.204139in}{1.773438in}}%
\pgfpathlineto{\pgfqpoint{1.219313in}{1.762007in}}%
\pgfpathlineto{\pgfqpoint{1.234486in}{1.761299in}}%
\pgfpathlineto{\pgfqpoint{1.249660in}{1.751879in}}%
\pgfpathlineto{\pgfqpoint{1.264833in}{1.747034in}}%
\pgfpathlineto{\pgfqpoint{1.280007in}{1.719518in}}%
\pgfpathlineto{\pgfqpoint{1.295181in}{1.730774in}}%
\pgfpathlineto{\pgfqpoint{1.310355in}{1.721164in}}%
\pgfpathlineto{\pgfqpoint{1.325528in}{1.724633in}}%
\pgfpathlineto{\pgfqpoint{1.340702in}{1.719234in}}%
\pgfpathlineto{\pgfqpoint{1.355876in}{1.718685in}}%
\pgfpathlineto{\pgfqpoint{1.371049in}{1.720619in}}%
\pgfpathlineto{\pgfqpoint{1.386223in}{1.716224in}}%
\pgfpathlineto{\pgfqpoint{1.401396in}{1.714292in}}%
\pgfpathlineto{\pgfqpoint{1.416570in}{1.702514in}}%
\pgfpathlineto{\pgfqpoint{1.431744in}{1.694156in}}%
\pgfpathlineto{\pgfqpoint{1.462091in}{1.684885in}}%
\pgfpathlineto{\pgfqpoint{1.477265in}{1.681114in}}%
\pgfpathlineto{\pgfqpoint{1.492439in}{1.671880in}}%
\pgfpathlineto{\pgfqpoint{1.507612in}{1.663961in}}%
\pgfpathlineto{\pgfqpoint{1.522786in}{1.660314in}}%
\pgfpathlineto{\pgfqpoint{1.537960in}{1.671651in}}%
\pgfpathlineto{\pgfqpoint{1.568307in}{1.656576in}}%
\pgfpathlineto{\pgfqpoint{1.583481in}{1.655878in}}%
\pgfpathlineto{\pgfqpoint{1.598654in}{1.652701in}}%
\pgfpathlineto{\pgfqpoint{1.629001in}{1.640209in}}%
\pgfpathlineto{\pgfqpoint{1.644175in}{1.638871in}}%
\pgfpathlineto{\pgfqpoint{1.659349in}{1.641997in}}%
\pgfpathlineto{\pgfqpoint{1.689696in}{1.625834in}}%
\pgfpathlineto{\pgfqpoint{1.704870in}{1.634201in}}%
\pgfpathlineto{\pgfqpoint{1.735217in}{1.625578in}}%
\pgfpathlineto{\pgfqpoint{1.765564in}{1.626050in}}%
\pgfpathlineto{\pgfqpoint{1.780738in}{1.616922in}}%
\pgfpathlineto{\pgfqpoint{1.795912in}{1.621840in}}%
\pgfpathlineto{\pgfqpoint{1.811085in}{1.619715in}}%
\pgfpathlineto{\pgfqpoint{1.826259in}{1.619670in}}%
\pgfpathlineto{\pgfqpoint{1.841433in}{1.605968in}}%
\pgfpathlineto{\pgfqpoint{1.856606in}{1.613931in}}%
\pgfpathlineto{\pgfqpoint{1.871780in}{1.616617in}}%
\pgfpathlineto{\pgfqpoint{1.886954in}{1.605837in}}%
\pgfpathlineto{\pgfqpoint{1.917301in}{1.604071in}}%
\pgfpathlineto{\pgfqpoint{1.932475in}{1.602299in}}%
\pgfpathlineto{\pgfqpoint{1.947648in}{1.603063in}}%
\pgfpathlineto{\pgfqpoint{1.962822in}{1.599792in}}%
\pgfpathlineto{\pgfqpoint{1.977996in}{1.599579in}}%
\pgfpathlineto{\pgfqpoint{1.993170in}{1.596729in}}%
\pgfpathlineto{\pgfqpoint{2.008343in}{1.599976in}}%
\pgfpathlineto{\pgfqpoint{2.023517in}{1.592790in}}%
\pgfpathlineto{\pgfqpoint{2.038690in}{1.597630in}}%
\pgfpathlineto{\pgfqpoint{2.053864in}{1.591032in}}%
\pgfpathlineto{\pgfqpoint{2.069038in}{1.592933in}}%
\pgfpathlineto{\pgfqpoint{2.084211in}{1.592285in}}%
\pgfpathlineto{\pgfqpoint{2.099385in}{1.582957in}}%
\pgfpathlineto{\pgfqpoint{2.114559in}{1.583474in}}%
\pgfpathlineto{\pgfqpoint{2.129733in}{1.585241in}}%
\pgfpathlineto{\pgfqpoint{2.144906in}{1.579621in}}%
\pgfpathlineto{\pgfqpoint{2.160080in}{1.583433in}}%
\pgfpathlineto{\pgfqpoint{2.175254in}{1.581105in}}%
\pgfpathlineto{\pgfqpoint{2.190427in}{1.585259in}}%
\pgfpathlineto{\pgfqpoint{2.205601in}{1.586366in}}%
\pgfpathlineto{\pgfqpoint{2.220775in}{1.586296in}}%
\pgfpathlineto{\pgfqpoint{2.235948in}{1.576155in}}%
\pgfpathlineto{\pgfqpoint{2.251122in}{1.563620in}}%
\pgfpathlineto{\pgfqpoint{2.266296in}{1.563743in}}%
\pgfpathlineto{\pgfqpoint{2.281469in}{1.573558in}}%
\pgfpathlineto{\pgfqpoint{2.296643in}{1.563854in}}%
\pgfpathlineto{\pgfqpoint{2.311816in}{1.564468in}}%
\pgfpathlineto{\pgfqpoint{2.326990in}{1.568356in}}%
\pgfpathlineto{\pgfqpoint{2.342164in}{1.567039in}}%
\pgfpathlineto{\pgfqpoint{2.357337in}{1.570288in}}%
\pgfpathlineto{\pgfqpoint{2.372511in}{1.565826in}}%
\pgfpathlineto{\pgfqpoint{2.387685in}{1.564262in}}%
\pgfpathlineto{\pgfqpoint{2.402858in}{1.552353in}}%
\pgfpathlineto{\pgfqpoint{2.418032in}{1.564007in}}%
\pgfpathlineto{\pgfqpoint{2.433206in}{1.549669in}}%
\pgfpathlineto{\pgfqpoint{2.448379in}{1.557125in}}%
\pgfpathlineto{\pgfqpoint{2.463553in}{1.555686in}}%
\pgfpathlineto{\pgfqpoint{2.478727in}{1.555925in}}%
\pgfpathlineto{\pgfqpoint{2.493900in}{1.546605in}}%
\pgfpathlineto{\pgfqpoint{2.509074in}{1.550598in}}%
\pgfpathlineto{\pgfqpoint{2.524248in}{1.548721in}}%
\pgfpathlineto{\pgfqpoint{2.539421in}{1.550492in}}%
\pgfpathlineto{\pgfqpoint{2.554595in}{1.545625in}}%
\pgfpathlineto{\pgfqpoint{2.569769in}{1.545155in}}%
\pgfpathlineto{\pgfqpoint{2.584942in}{1.546007in}}%
\pgfpathlineto{\pgfqpoint{2.600116in}{1.539648in}}%
\pgfpathlineto{\pgfqpoint{2.615290in}{1.541381in}}%
\pgfpathlineto{\pgfqpoint{2.630463in}{1.532413in}}%
\pgfpathlineto{\pgfqpoint{2.645637in}{1.539005in}}%
\pgfpathlineto{\pgfqpoint{2.660811in}{1.537372in}}%
\pgfpathlineto{\pgfqpoint{2.675984in}{1.529947in}}%
\pgfpathlineto{\pgfqpoint{2.706332in}{1.529969in}}%
\pgfpathlineto{\pgfqpoint{2.721506in}{1.535464in}}%
\pgfpathlineto{\pgfqpoint{2.736679in}{1.528639in}}%
\pgfpathlineto{\pgfqpoint{2.751853in}{1.525855in}}%
\pgfpathlineto{\pgfqpoint{2.767027in}{1.529288in}}%
\pgfpathlineto{\pgfqpoint{2.782200in}{1.527431in}}%
\pgfpathlineto{\pgfqpoint{2.812547in}{1.525335in}}%
\pgfpathlineto{\pgfqpoint{2.827721in}{1.521459in}}%
\pgfpathlineto{\pgfqpoint{2.858068in}{1.518284in}}%
\pgfpathlineto{\pgfqpoint{2.873242in}{1.514127in}}%
\pgfpathlineto{\pgfqpoint{2.888416in}{1.505516in}}%
\pgfpathlineto{\pgfqpoint{2.918763in}{1.501523in}}%
\pgfpathlineto{\pgfqpoint{2.933937in}{1.497911in}}%
\pgfpathlineto{\pgfqpoint{2.949110in}{1.503078in}}%
\pgfpathlineto{\pgfqpoint{2.979458in}{1.498221in}}%
\pgfpathlineto{\pgfqpoint{2.994632in}{1.498938in}}%
\pgfpathlineto{\pgfqpoint{3.009805in}{1.495866in}}%
\pgfpathlineto{\pgfqpoint{3.024979in}{1.488958in}}%
\pgfpathlineto{\pgfqpoint{3.040153in}{1.492201in}}%
\pgfpathlineto{\pgfqpoint{3.070500in}{1.484034in}}%
\pgfpathlineto{\pgfqpoint{3.085673in}{1.483617in}}%
\pgfpathlineto{\pgfqpoint{3.100847in}{1.478803in}}%
\pgfpathlineto{\pgfqpoint{3.116021in}{1.479912in}}%
\pgfpathlineto{\pgfqpoint{3.131194in}{1.475915in}}%
\pgfpathlineto{\pgfqpoint{3.146368in}{1.481753in}}%
\pgfpathlineto{\pgfqpoint{3.161542in}{1.479898in}}%
\pgfpathlineto{\pgfqpoint{3.176715in}{1.480319in}}%
\pgfpathlineto{\pgfqpoint{3.191889in}{1.473408in}}%
\pgfpathlineto{\pgfqpoint{3.207063in}{1.473951in}}%
\pgfpathlineto{\pgfqpoint{3.222236in}{1.472806in}}%
\pgfpathlineto{\pgfqpoint{3.237410in}{1.470362in}}%
\pgfpathlineto{\pgfqpoint{3.267757in}{1.471431in}}%
\pgfpathlineto{\pgfqpoint{3.282931in}{1.459473in}}%
\pgfpathlineto{\pgfqpoint{3.298105in}{1.460036in}}%
\pgfpathlineto{\pgfqpoint{3.313278in}{1.464274in}}%
\pgfpathlineto{\pgfqpoint{3.328452in}{1.458171in}}%
\pgfpathlineto{\pgfqpoint{3.343626in}{1.453786in}}%
\pgfpathlineto{\pgfqpoint{3.358799in}{1.452737in}}%
\pgfpathlineto{\pgfqpoint{3.373973in}{1.455498in}}%
\pgfpathlineto{\pgfqpoint{3.404321in}{1.456176in}}%
\pgfpathlineto{\pgfqpoint{3.434668in}{1.449274in}}%
\pgfpathlineto{\pgfqpoint{3.449842in}{1.442001in}}%
\pgfpathlineto{\pgfqpoint{3.465015in}{1.442032in}}%
\pgfpathlineto{\pgfqpoint{3.480189in}{1.438680in}}%
\pgfpathlineto{\pgfqpoint{3.495363in}{1.440052in}}%
\pgfpathlineto{\pgfqpoint{3.510536in}{1.437848in}}%
\pgfpathlineto{\pgfqpoint{3.525710in}{1.440438in}}%
\pgfpathlineto{\pgfqpoint{3.540884in}{1.423475in}}%
\pgfpathlineto{\pgfqpoint{3.556057in}{1.416081in}}%
\pgfpathlineto{\pgfqpoint{3.571231in}{1.422864in}}%
\pgfpathlineto{\pgfqpoint{3.616752in}{1.434068in}}%
\pgfpathlineto{\pgfqpoint{3.631925in}{1.429344in}}%
\pgfpathlineto{\pgfqpoint{3.662273in}{1.425398in}}%
\pgfpathlineto{\pgfqpoint{3.677446in}{1.432637in}}%
\pgfpathlineto{\pgfqpoint{3.692620in}{1.427521in}}%
\pgfpathlineto{\pgfqpoint{3.707794in}{1.414934in}}%
\pgfpathlineto{\pgfqpoint{3.722967in}{1.419609in}}%
\pgfpathlineto{\pgfqpoint{3.753315in}{1.419382in}}%
\pgfpathlineto{\pgfqpoint{3.768488in}{1.416037in}}%
\pgfpathlineto{\pgfqpoint{3.783662in}{1.419986in}}%
\pgfpathlineto{\pgfqpoint{3.798836in}{1.411651in}}%
\pgfpathlineto{\pgfqpoint{3.814009in}{1.417888in}}%
\pgfpathlineto{\pgfqpoint{3.829183in}{1.407740in}}%
\pgfpathlineto{\pgfqpoint{3.844357in}{1.406876in}}%
\pgfpathlineto{\pgfqpoint{3.859530in}{1.402143in}}%
\pgfpathlineto{\pgfqpoint{3.874704in}{1.395205in}}%
\pgfpathlineto{\pgfqpoint{3.889878in}{1.393311in}}%
\pgfpathlineto{\pgfqpoint{3.905052in}{1.389084in}}%
\pgfpathlineto{\pgfqpoint{3.950573in}{1.391749in}}%
\pgfpathlineto{\pgfqpoint{3.965746in}{1.384014in}}%
\pgfpathlineto{\pgfqpoint{3.980920in}{1.388897in}}%
\pgfpathlineto{\pgfqpoint{3.996094in}{1.385625in}}%
\pgfpathlineto{\pgfqpoint{4.011267in}{1.390967in}}%
\pgfpathlineto{\pgfqpoint{4.026441in}{1.392156in}}%
\pgfpathlineto{\pgfqpoint{4.041614in}{1.375769in}}%
\pgfpathlineto{\pgfqpoint{4.056788in}{1.383196in}}%
\pgfpathlineto{\pgfqpoint{4.071962in}{1.376220in}}%
\pgfpathlineto{\pgfqpoint{4.087136in}{1.379775in}}%
\pgfpathlineto{\pgfqpoint{4.102309in}{1.376248in}}%
\pgfpathlineto{\pgfqpoint{4.117483in}{1.379120in}}%
\pgfpathlineto{\pgfqpoint{4.132657in}{1.371371in}}%
\pgfpathlineto{\pgfqpoint{4.147830in}{1.355799in}}%
\pgfpathlineto{\pgfqpoint{4.163004in}{1.375588in}}%
\pgfpathlineto{\pgfqpoint{4.178178in}{1.371734in}}%
\pgfpathlineto{\pgfqpoint{4.193351in}{1.369364in}}%
\pgfpathlineto{\pgfqpoint{4.208525in}{1.368519in}}%
\pgfpathlineto{\pgfqpoint{4.223699in}{1.365814in}}%
\pgfpathlineto{\pgfqpoint{4.238872in}{1.367870in}}%
\pgfpathlineto{\pgfqpoint{4.254046in}{1.363789in}}%
\pgfpathlineto{\pgfqpoint{4.269220in}{1.367069in}}%
\pgfpathlineto{\pgfqpoint{4.284393in}{1.355332in}}%
\pgfpathlineto{\pgfqpoint{4.299567in}{1.358022in}}%
\pgfpathlineto{\pgfqpoint{4.314741in}{1.356927in}}%
\pgfpathlineto{\pgfqpoint{4.329914in}{1.359825in}}%
\pgfpathlineto{\pgfqpoint{4.360262in}{1.362685in}}%
\pgfpathlineto{\pgfqpoint{4.375435in}{1.359714in}}%
\pgfpathlineto{\pgfqpoint{4.390609in}{1.369919in}}%
\pgfpathlineto{\pgfqpoint{4.405782in}{1.364975in}}%
\pgfpathlineto{\pgfqpoint{4.420956in}{1.358467in}}%
\pgfpathlineto{\pgfqpoint{4.436130in}{1.356242in}}%
\pgfpathlineto{\pgfqpoint{4.451303in}{1.349325in}}%
\pgfpathlineto{\pgfqpoint{4.466477in}{1.355691in}}%
\pgfpathlineto{\pgfqpoint{4.481651in}{1.352782in}}%
\pgfpathlineto{\pgfqpoint{4.496824in}{1.358250in}}%
\pgfpathlineto{\pgfqpoint{4.527172in}{1.347051in}}%
\pgfpathlineto{\pgfqpoint{4.542345in}{1.349875in}}%
\pgfpathlineto{\pgfqpoint{4.557519in}{1.345762in}}%
\pgfpathlineto{\pgfqpoint{4.572693in}{1.347084in}}%
\pgfpathlineto{\pgfqpoint{4.587867in}{1.353728in}}%
\pgfpathlineto{\pgfqpoint{4.587867in}{1.353728in}}%
\pgfusepath{stroke}%
\end{pgfscope}%
\begin{pgfscope}%
\pgfpathrectangle{\pgfqpoint{0.809624in}{0.450000in}}{\pgfqpoint{3.778243in}{2.490000in}}%
\pgfusepath{clip}%
\pgfsetrectcap%
\pgfsetroundjoin%
\pgfsetlinewidth{1.003750pt}%
\definecolor{currentstroke}{rgb}{0.000000,0.500000,0.000000}%
\pgfsetstrokecolor{currentstroke}%
\pgfsetdash{}{0pt}%
\pgfpathmoveto{\pgfqpoint{0.809624in}{2.147347in}}%
\pgfpathlineto{\pgfqpoint{0.824797in}{2.109026in}}%
\pgfpathlineto{\pgfqpoint{0.839971in}{2.105808in}}%
\pgfpathlineto{\pgfqpoint{0.855144in}{2.075938in}}%
\pgfpathlineto{\pgfqpoint{0.870318in}{2.042260in}}%
\pgfpathlineto{\pgfqpoint{0.885492in}{1.971140in}}%
\pgfpathlineto{\pgfqpoint{0.900666in}{1.988433in}}%
\pgfpathlineto{\pgfqpoint{0.915839in}{1.975742in}}%
\pgfpathlineto{\pgfqpoint{0.931013in}{1.911630in}}%
\pgfpathlineto{\pgfqpoint{0.946187in}{1.911630in}}%
\pgfpathlineto{\pgfqpoint{0.961360in}{1.910246in}}%
\pgfpathlineto{\pgfqpoint{0.976534in}{1.879060in}}%
\pgfpathlineto{\pgfqpoint{0.991707in}{1.872940in}}%
\pgfpathlineto{\pgfqpoint{1.006881in}{1.858751in}}%
\pgfpathlineto{\pgfqpoint{1.022055in}{1.842546in}}%
\pgfpathlineto{\pgfqpoint{1.037228in}{1.828223in}}%
\pgfpathlineto{\pgfqpoint{1.052402in}{1.828169in}}%
\pgfpathlineto{\pgfqpoint{1.067576in}{1.822963in}}%
\pgfpathlineto{\pgfqpoint{1.082750in}{1.806812in}}%
\pgfpathlineto{\pgfqpoint{1.097923in}{1.792401in}}%
\pgfpathlineto{\pgfqpoint{1.113097in}{1.797729in}}%
\pgfpathlineto{\pgfqpoint{1.128270in}{1.785340in}}%
\pgfpathlineto{\pgfqpoint{1.143444in}{1.762982in}}%
\pgfpathlineto{\pgfqpoint{1.158618in}{1.759643in}}%
\pgfpathlineto{\pgfqpoint{1.173791in}{1.744903in}}%
\pgfpathlineto{\pgfqpoint{1.188965in}{1.712132in}}%
\pgfpathlineto{\pgfqpoint{1.204139in}{1.748027in}}%
\pgfpathlineto{\pgfqpoint{1.219313in}{1.737607in}}%
\pgfpathlineto{\pgfqpoint{1.234486in}{1.738165in}}%
\pgfpathlineto{\pgfqpoint{1.249660in}{1.731534in}}%
\pgfpathlineto{\pgfqpoint{1.264833in}{1.707006in}}%
\pgfpathlineto{\pgfqpoint{1.280007in}{1.689995in}}%
\pgfpathlineto{\pgfqpoint{1.295181in}{1.701188in}}%
\pgfpathlineto{\pgfqpoint{1.310355in}{1.681436in}}%
\pgfpathlineto{\pgfqpoint{1.325528in}{1.703855in}}%
\pgfpathlineto{\pgfqpoint{1.340702in}{1.700045in}}%
\pgfpathlineto{\pgfqpoint{1.355876in}{1.705454in}}%
\pgfpathlineto{\pgfqpoint{1.371049in}{1.699851in}}%
\pgfpathlineto{\pgfqpoint{1.386223in}{1.688329in}}%
\pgfpathlineto{\pgfqpoint{1.401396in}{1.691547in}}%
\pgfpathlineto{\pgfqpoint{1.416570in}{1.682699in}}%
\pgfpathlineto{\pgfqpoint{1.431744in}{1.654617in}}%
\pgfpathlineto{\pgfqpoint{1.446917in}{1.666273in}}%
\pgfpathlineto{\pgfqpoint{1.462091in}{1.659521in}}%
\pgfpathlineto{\pgfqpoint{1.477265in}{1.640334in}}%
\pgfpathlineto{\pgfqpoint{1.492439in}{1.630780in}}%
\pgfpathlineto{\pgfqpoint{1.507612in}{1.629712in}}%
\pgfpathlineto{\pgfqpoint{1.522786in}{1.634260in}}%
\pgfpathlineto{\pgfqpoint{1.537960in}{1.642268in}}%
\pgfpathlineto{\pgfqpoint{1.553133in}{1.637727in}}%
\pgfpathlineto{\pgfqpoint{1.568307in}{1.623572in}}%
\pgfpathlineto{\pgfqpoint{1.583481in}{1.626171in}}%
\pgfpathlineto{\pgfqpoint{1.598654in}{1.623565in}}%
\pgfpathlineto{\pgfqpoint{1.613828in}{1.623565in}}%
\pgfpathlineto{\pgfqpoint{1.629001in}{1.622107in}}%
\pgfpathlineto{\pgfqpoint{1.644175in}{1.616370in}}%
\pgfpathlineto{\pgfqpoint{1.659349in}{1.617532in}}%
\pgfpathlineto{\pgfqpoint{1.674522in}{1.590867in}}%
\pgfpathlineto{\pgfqpoint{1.689696in}{1.590262in}}%
\pgfpathlineto{\pgfqpoint{1.704870in}{1.602201in}}%
\pgfpathlineto{\pgfqpoint{1.720043in}{1.583887in}}%
\pgfpathlineto{\pgfqpoint{1.735217in}{1.612984in}}%
\pgfpathlineto{\pgfqpoint{1.750391in}{1.605002in}}%
\pgfpathlineto{\pgfqpoint{1.765564in}{1.605002in}}%
\pgfpathlineto{\pgfqpoint{1.780738in}{1.538317in}}%
\pgfpathlineto{\pgfqpoint{1.795912in}{1.572237in}}%
\pgfpathlineto{\pgfqpoint{1.811085in}{1.570759in}}%
\pgfpathlineto{\pgfqpoint{1.826259in}{1.581099in}}%
\pgfpathlineto{\pgfqpoint{1.841433in}{1.548341in}}%
\pgfpathlineto{\pgfqpoint{1.856606in}{1.584471in}}%
\pgfpathlineto{\pgfqpoint{1.871780in}{1.564666in}}%
\pgfpathlineto{\pgfqpoint{1.886954in}{1.568730in}}%
\pgfpathlineto{\pgfqpoint{1.902127in}{1.561683in}}%
\pgfpathlineto{\pgfqpoint{1.917301in}{1.540648in}}%
\pgfpathlineto{\pgfqpoint{1.932475in}{1.554703in}}%
\pgfpathlineto{\pgfqpoint{1.947648in}{1.548992in}}%
\pgfpathlineto{\pgfqpoint{1.962822in}{1.529536in}}%
\pgfpathlineto{\pgfqpoint{1.977996in}{1.535764in}}%
\pgfpathlineto{\pgfqpoint{1.993170in}{1.526513in}}%
\pgfpathlineto{\pgfqpoint{2.008343in}{1.559997in}}%
\pgfpathlineto{\pgfqpoint{2.023517in}{1.555710in}}%
\pgfpathlineto{\pgfqpoint{2.038690in}{1.560104in}}%
\pgfpathlineto{\pgfqpoint{2.053864in}{1.551955in}}%
\pgfpathlineto{\pgfqpoint{2.069038in}{1.541266in}}%
\pgfpathlineto{\pgfqpoint{2.084211in}{1.491222in}}%
\pgfpathlineto{\pgfqpoint{2.114559in}{1.491222in}}%
\pgfpathlineto{\pgfqpoint{2.129733in}{1.514373in}}%
\pgfpathlineto{\pgfqpoint{2.144906in}{1.514373in}}%
\pgfpathlineto{\pgfqpoint{2.160080in}{1.521071in}}%
\pgfpathlineto{\pgfqpoint{2.175254in}{1.516825in}}%
\pgfpathlineto{\pgfqpoint{2.190427in}{1.539956in}}%
\pgfpathlineto{\pgfqpoint{2.205601in}{1.523644in}}%
\pgfpathlineto{\pgfqpoint{2.220775in}{1.527977in}}%
\pgfpathlineto{\pgfqpoint{2.235948in}{1.538955in}}%
\pgfpathlineto{\pgfqpoint{2.251122in}{1.508468in}}%
\pgfpathlineto{\pgfqpoint{2.266296in}{1.516160in}}%
\pgfpathlineto{\pgfqpoint{2.281469in}{1.521091in}}%
\pgfpathlineto{\pgfqpoint{2.296643in}{1.509200in}}%
\pgfpathlineto{\pgfqpoint{2.311816in}{1.524336in}}%
\pgfpathlineto{\pgfqpoint{2.326990in}{1.490362in}}%
\pgfpathlineto{\pgfqpoint{2.357337in}{1.513983in}}%
\pgfpathlineto{\pgfqpoint{2.372511in}{1.508562in}}%
\pgfpathlineto{\pgfqpoint{2.387685in}{1.517793in}}%
\pgfpathlineto{\pgfqpoint{2.402858in}{1.509576in}}%
\pgfpathlineto{\pgfqpoint{2.418032in}{1.514205in}}%
\pgfpathlineto{\pgfqpoint{2.448379in}{1.469126in}}%
\pgfpathlineto{\pgfqpoint{2.463553in}{1.480043in}}%
\pgfpathlineto{\pgfqpoint{2.478727in}{1.480043in}}%
\pgfpathlineto{\pgfqpoint{2.493900in}{1.463899in}}%
\pgfpathlineto{\pgfqpoint{2.509074in}{1.469717in}}%
\pgfpathlineto{\pgfqpoint{2.524248in}{1.462925in}}%
\pgfpathlineto{\pgfqpoint{2.539421in}{1.467614in}}%
\pgfpathlineto{\pgfqpoint{2.554595in}{1.458450in}}%
\pgfpathlineto{\pgfqpoint{2.569769in}{1.463160in}}%
\pgfpathlineto{\pgfqpoint{2.584942in}{1.480291in}}%
\pgfpathlineto{\pgfqpoint{2.600116in}{1.444423in}}%
\pgfpathlineto{\pgfqpoint{2.615290in}{1.444973in}}%
\pgfpathlineto{\pgfqpoint{2.630463in}{1.412988in}}%
\pgfpathlineto{\pgfqpoint{2.645637in}{1.448957in}}%
\pgfpathlineto{\pgfqpoint{2.675984in}{1.449508in}}%
\pgfpathlineto{\pgfqpoint{2.691158in}{1.447788in}}%
\pgfpathlineto{\pgfqpoint{2.706332in}{1.453915in}}%
\pgfpathlineto{\pgfqpoint{2.721506in}{1.453909in}}%
\pgfpathlineto{\pgfqpoint{2.736679in}{1.428003in}}%
\pgfpathlineto{\pgfqpoint{2.767027in}{1.428003in}}%
\pgfpathlineto{\pgfqpoint{2.782200in}{1.417536in}}%
\pgfpathlineto{\pgfqpoint{2.797374in}{1.403972in}}%
\pgfpathlineto{\pgfqpoint{2.812547in}{1.434634in}}%
\pgfpathlineto{\pgfqpoint{2.827721in}{1.421856in}}%
\pgfpathlineto{\pgfqpoint{2.842895in}{1.431228in}}%
\pgfpathlineto{\pgfqpoint{2.858068in}{1.429985in}}%
\pgfpathlineto{\pgfqpoint{2.873242in}{1.430905in}}%
\pgfpathlineto{\pgfqpoint{2.888416in}{1.407062in}}%
\pgfpathlineto{\pgfqpoint{2.903589in}{1.407062in}}%
\pgfpathlineto{\pgfqpoint{2.918763in}{1.419881in}}%
\pgfpathlineto{\pgfqpoint{2.933937in}{1.391886in}}%
\pgfpathlineto{\pgfqpoint{2.949110in}{1.411073in}}%
\pgfpathlineto{\pgfqpoint{2.964284in}{1.392141in}}%
\pgfpathlineto{\pgfqpoint{2.979458in}{1.378510in}}%
\pgfpathlineto{\pgfqpoint{2.994632in}{1.403925in}}%
\pgfpathlineto{\pgfqpoint{3.009805in}{1.388453in}}%
\pgfpathlineto{\pgfqpoint{3.024979in}{1.378160in}}%
\pgfpathlineto{\pgfqpoint{3.040153in}{1.372853in}}%
\pgfpathlineto{\pgfqpoint{3.055326in}{1.369312in}}%
\pgfpathlineto{\pgfqpoint{3.070500in}{1.372853in}}%
\pgfpathlineto{\pgfqpoint{3.085673in}{1.355063in}}%
\pgfpathlineto{\pgfqpoint{3.100847in}{1.328687in}}%
\pgfpathlineto{\pgfqpoint{3.116021in}{1.354559in}}%
\pgfpathlineto{\pgfqpoint{3.131194in}{1.355251in}}%
\pgfpathlineto{\pgfqpoint{3.146368in}{1.346511in}}%
\pgfpathlineto{\pgfqpoint{3.161542in}{1.344710in}}%
\pgfpathlineto{\pgfqpoint{3.176715in}{1.348580in}}%
\pgfpathlineto{\pgfqpoint{3.191889in}{1.294108in}}%
\pgfpathlineto{\pgfqpoint{3.207063in}{1.311032in}}%
\pgfpathlineto{\pgfqpoint{3.222236in}{1.314250in}}%
\pgfpathlineto{\pgfqpoint{3.252584in}{1.314794in}}%
\pgfpathlineto{\pgfqpoint{3.267757in}{1.320256in}}%
\pgfpathlineto{\pgfqpoint{3.282931in}{1.314821in}}%
\pgfpathlineto{\pgfqpoint{3.298105in}{1.297877in}}%
\pgfpathlineto{\pgfqpoint{3.313278in}{1.293927in}}%
\pgfpathlineto{\pgfqpoint{3.328452in}{1.291918in}}%
\pgfpathlineto{\pgfqpoint{3.358799in}{1.291018in}}%
\pgfpathlineto{\pgfqpoint{3.373973in}{1.304622in}}%
\pgfpathlineto{\pgfqpoint{3.389147in}{1.307014in}}%
\pgfpathlineto{\pgfqpoint{3.404321in}{1.298939in}}%
\pgfpathlineto{\pgfqpoint{3.419494in}{1.287014in}}%
\pgfpathlineto{\pgfqpoint{3.434668in}{1.287014in}}%
\pgfpathlineto{\pgfqpoint{3.449842in}{1.280020in}}%
\pgfpathlineto{\pgfqpoint{3.465015in}{1.260752in}}%
\pgfpathlineto{\pgfqpoint{3.480189in}{1.263345in}}%
\pgfpathlineto{\pgfqpoint{3.495363in}{1.280329in}}%
\pgfpathlineto{\pgfqpoint{3.510536in}{1.280329in}}%
\pgfpathlineto{\pgfqpoint{3.525710in}{1.285502in}}%
\pgfpathlineto{\pgfqpoint{3.540884in}{1.271535in}}%
\pgfpathlineto{\pgfqpoint{3.556057in}{1.254766in}}%
\pgfpathlineto{\pgfqpoint{3.571231in}{1.252280in}}%
\pgfpathlineto{\pgfqpoint{3.586404in}{1.247820in}}%
\pgfpathlineto{\pgfqpoint{3.616752in}{1.257024in}}%
\pgfpathlineto{\pgfqpoint{3.631925in}{1.244501in}}%
\pgfpathlineto{\pgfqpoint{3.647099in}{1.235948in}}%
\pgfpathlineto{\pgfqpoint{3.662273in}{1.259328in}}%
\pgfpathlineto{\pgfqpoint{3.677446in}{1.258985in}}%
\pgfpathlineto{\pgfqpoint{3.692620in}{1.256506in}}%
\pgfpathlineto{\pgfqpoint{3.707794in}{1.210883in}}%
\pgfpathlineto{\pgfqpoint{3.722967in}{1.235492in}}%
\pgfpathlineto{\pgfqpoint{3.738141in}{1.238293in}}%
\pgfpathlineto{\pgfqpoint{3.753315in}{1.235505in}}%
\pgfpathlineto{\pgfqpoint{3.783662in}{1.235505in}}%
\pgfpathlineto{\pgfqpoint{3.798836in}{1.248673in}}%
\pgfpathlineto{\pgfqpoint{3.814009in}{1.237043in}}%
\pgfpathlineto{\pgfqpoint{3.829183in}{1.237265in}}%
\pgfpathlineto{\pgfqpoint{3.844357in}{1.230970in}}%
\pgfpathlineto{\pgfqpoint{3.859530in}{1.226845in}}%
\pgfpathlineto{\pgfqpoint{3.874704in}{1.246960in}}%
\pgfpathlineto{\pgfqpoint{3.889878in}{1.229123in}}%
\pgfpathlineto{\pgfqpoint{3.905052in}{1.230970in}}%
\pgfpathlineto{\pgfqpoint{3.920225in}{1.219636in}}%
\pgfpathlineto{\pgfqpoint{3.935399in}{1.218857in}}%
\pgfpathlineto{\pgfqpoint{3.950573in}{1.225206in}}%
\pgfpathlineto{\pgfqpoint{3.965746in}{1.220208in}}%
\pgfpathlineto{\pgfqpoint{3.980920in}{1.212972in}}%
\pgfpathlineto{\pgfqpoint{3.996094in}{1.209902in}}%
\pgfpathlineto{\pgfqpoint{4.026441in}{1.209586in}}%
\pgfpathlineto{\pgfqpoint{4.041614in}{1.201685in}}%
\pgfpathlineto{\pgfqpoint{4.071962in}{1.209317in}}%
\pgfpathlineto{\pgfqpoint{4.087136in}{1.202948in}}%
\pgfpathlineto{\pgfqpoint{4.102309in}{1.202948in}}%
\pgfpathlineto{\pgfqpoint{4.117483in}{1.196586in}}%
\pgfpathlineto{\pgfqpoint{4.132657in}{1.200254in}}%
\pgfpathlineto{\pgfqpoint{4.147830in}{1.162981in}}%
\pgfpathlineto{\pgfqpoint{4.163004in}{1.218266in}}%
\pgfpathlineto{\pgfqpoint{4.178178in}{1.203593in}}%
\pgfpathlineto{\pgfqpoint{4.193351in}{1.210641in}}%
\pgfpathlineto{\pgfqpoint{4.208525in}{1.200738in}}%
\pgfpathlineto{\pgfqpoint{4.223699in}{1.215377in}}%
\pgfpathlineto{\pgfqpoint{4.238872in}{1.199213in}}%
\pgfpathlineto{\pgfqpoint{4.254046in}{1.204043in}}%
\pgfpathlineto{\pgfqpoint{4.269220in}{1.196969in}}%
\pgfpathlineto{\pgfqpoint{4.284393in}{1.163942in}}%
\pgfpathlineto{\pgfqpoint{4.299567in}{1.163942in}}%
\pgfpathlineto{\pgfqpoint{4.314741in}{1.198118in}}%
\pgfpathlineto{\pgfqpoint{4.329914in}{1.185481in}}%
\pgfpathlineto{\pgfqpoint{4.360262in}{1.185481in}}%
\pgfpathlineto{\pgfqpoint{4.375435in}{1.199945in}}%
\pgfpathlineto{\pgfqpoint{4.390609in}{1.209828in}}%
\pgfpathlineto{\pgfqpoint{4.405782in}{1.195565in}}%
\pgfpathlineto{\pgfqpoint{4.420956in}{1.184366in}}%
\pgfpathlineto{\pgfqpoint{4.436130in}{1.197359in}}%
\pgfpathlineto{\pgfqpoint{4.451303in}{1.179743in}}%
\pgfpathlineto{\pgfqpoint{4.466477in}{1.183761in}}%
\pgfpathlineto{\pgfqpoint{4.481651in}{1.190808in}}%
\pgfpathlineto{\pgfqpoint{4.496824in}{1.170358in}}%
\pgfpathlineto{\pgfqpoint{4.511998in}{1.163485in}}%
\pgfpathlineto{\pgfqpoint{4.527172in}{1.163485in}}%
\pgfpathlineto{\pgfqpoint{4.542345in}{1.166515in}}%
\pgfpathlineto{\pgfqpoint{4.557519in}{1.172562in}}%
\pgfpathlineto{\pgfqpoint{4.572693in}{1.183808in}}%
\pgfpathlineto{\pgfqpoint{4.587867in}{1.184016in}}%
\pgfpathlineto{\pgfqpoint{4.587867in}{1.184016in}}%
\pgfusepath{stroke}%
\end{pgfscope}%
\begin{pgfscope}%
\pgfsetrectcap%
\pgfsetmiterjoin%
\pgfsetlinewidth{0.401500pt}%
\definecolor{currentstroke}{rgb}{0.000000,0.000000,0.000000}%
\pgfsetstrokecolor{currentstroke}%
\pgfsetdash{}{0pt}%
\pgfpathmoveto{\pgfqpoint{0.809624in}{0.450000in}}%
\pgfpathlineto{\pgfqpoint{0.809624in}{2.940000in}}%
\pgfusepath{stroke}%
\end{pgfscope}%
\begin{pgfscope}%
\pgfsetrectcap%
\pgfsetmiterjoin%
\pgfsetlinewidth{0.401500pt}%
\definecolor{currentstroke}{rgb}{0.000000,0.000000,0.000000}%
\pgfsetstrokecolor{currentstroke}%
\pgfsetdash{}{0pt}%
\pgfpathmoveto{\pgfqpoint{4.587867in}{0.450000in}}%
\pgfpathlineto{\pgfqpoint{4.587867in}{2.940000in}}%
\pgfusepath{stroke}%
\end{pgfscope}%
\begin{pgfscope}%
\pgfsetrectcap%
\pgfsetmiterjoin%
\pgfsetlinewidth{0.401500pt}%
\definecolor{currentstroke}{rgb}{0.000000,0.000000,0.000000}%
\pgfsetstrokecolor{currentstroke}%
\pgfsetdash{}{0pt}%
\pgfpathmoveto{\pgfqpoint{0.809624in}{0.450000in}}%
\pgfpathlineto{\pgfqpoint{4.587867in}{0.450000in}}%
\pgfusepath{stroke}%
\end{pgfscope}%
\begin{pgfscope}%
\pgfsetrectcap%
\pgfsetmiterjoin%
\pgfsetlinewidth{0.401500pt}%
\definecolor{currentstroke}{rgb}{0.000000,0.000000,0.000000}%
\pgfsetstrokecolor{currentstroke}%
\pgfsetdash{}{0pt}%
\pgfpathmoveto{\pgfqpoint{0.809624in}{2.940000in}}%
\pgfpathlineto{\pgfqpoint{4.587867in}{2.940000in}}%
\pgfusepath{stroke}%
\end{pgfscope}%
\begin{pgfscope}%
\pgfsetbuttcap%
\pgfsetmiterjoin%
\definecolor{currentfill}{rgb}{1.000000,1.000000,1.000000}%
\pgfsetfillcolor{currentfill}%
\pgfsetfillopacity{0.800000}%
\pgfsetlinewidth{1.003750pt}%
\definecolor{currentstroke}{rgb}{0.800000,0.800000,0.800000}%
\pgfsetstrokecolor{currentstroke}%
\pgfsetstrokeopacity{0.800000}%
\pgfsetdash{}{0pt}%
\pgfpathmoveto{\pgfqpoint{3.717110in}{2.247871in}}%
\pgfpathlineto{\pgfqpoint{4.490644in}{2.247871in}}%
\pgfpathquadraticcurveto{\pgfqpoint{4.518422in}{2.247871in}}{\pgfqpoint{4.518422in}{2.275648in}}%
\pgfpathlineto{\pgfqpoint{4.518422in}{2.842778in}}%
\pgfpathquadraticcurveto{\pgfqpoint{4.518422in}{2.870556in}}{\pgfqpoint{4.490644in}{2.870556in}}%
\pgfpathlineto{\pgfqpoint{3.717110in}{2.870556in}}%
\pgfpathquadraticcurveto{\pgfqpoint{3.689332in}{2.870556in}}{\pgfqpoint{3.689332in}{2.842778in}}%
\pgfpathlineto{\pgfqpoint{3.689332in}{2.275648in}}%
\pgfpathquadraticcurveto{\pgfqpoint{3.689332in}{2.247871in}}{\pgfqpoint{3.717110in}{2.247871in}}%
\pgfpathclose%
\pgfusepath{stroke,fill}%
\end{pgfscope}%
\begin{pgfscope}%
\pgfsetrectcap%
\pgfsetroundjoin%
\pgfsetlinewidth{1.003750pt}%
\definecolor{currentstroke}{rgb}{0.000000,0.000000,1.000000}%
\pgfsetstrokecolor{currentstroke}%
\pgfsetdash{}{0pt}%
\pgfpathmoveto{\pgfqpoint{3.744888in}{2.766389in}}%
\pgfpathlineto{\pgfqpoint{4.022666in}{2.766389in}}%
\pgfusepath{stroke}%
\end{pgfscope}%
\begin{pgfscope}%
\definecolor{textcolor}{rgb}{0.000000,0.000000,0.000000}%
\pgfsetstrokecolor{textcolor}%
\pgfsetfillcolor{textcolor}%
\pgftext[x=4.133777in,y=2.717778in,left,base]{\color{textcolor}\rmfamily\fontsize{10.000000}{12.000000}\selectfont worst}%
\end{pgfscope}%
\begin{pgfscope}%
\pgfsetrectcap%
\pgfsetroundjoin%
\pgfsetlinewidth{1.003750pt}%
\definecolor{currentstroke}{rgb}{0.411765,0.411765,0.411765}%
\pgfsetstrokecolor{currentstroke}%
\pgfsetdash{}{0pt}%
\pgfpathmoveto{\pgfqpoint{3.744888in}{2.572716in}}%
\pgfpathlineto{\pgfqpoint{4.022666in}{2.572716in}}%
\pgfusepath{stroke}%
\end{pgfscope}%
\begin{pgfscope}%
\definecolor{textcolor}{rgb}{0.000000,0.000000,0.000000}%
\pgfsetstrokecolor{textcolor}%
\pgfsetfillcolor{textcolor}%
\pgftext[x=4.133777in,y=2.524105in,left,base]{\color{textcolor}\rmfamily\fontsize{10.000000}{12.000000}\selectfont avg}%
\end{pgfscope}%
\begin{pgfscope}%
\pgfsetrectcap%
\pgfsetroundjoin%
\pgfsetlinewidth{1.003750pt}%
\definecolor{currentstroke}{rgb}{0.000000,0.500000,0.000000}%
\pgfsetstrokecolor{currentstroke}%
\pgfsetdash{}{0pt}%
\pgfpathmoveto{\pgfqpoint{3.744888in}{2.379043in}}%
\pgfpathlineto{\pgfqpoint{4.022666in}{2.379043in}}%
\pgfusepath{stroke}%
\end{pgfscope}%
\begin{pgfscope}%
\definecolor{textcolor}{rgb}{0.000000,0.000000,0.000000}%
\pgfsetstrokecolor{textcolor}%
\pgfsetfillcolor{textcolor}%
\pgftext[x=4.133777in,y=2.330432in,left,base]{\color{textcolor}\rmfamily\fontsize{10.000000}{12.000000}\selectfont best}%
\end{pgfscope}%
\end{pgfpicture}%
\makeatother%
\endgroup%

  \end{center}
  \caption{Przebieg algorytmu dla $P_m = 0.01$.}
\end{figure}
\begin{figure}[H]
  \begin{center}
    %% Creator: Matplotlib, PGF backend
%%
%% To include the figure in your LaTeX document, write
%%   \input{<filename>.pgf}
%%
%% Make sure the required packages are loaded in your preamble
%%   \usepackage{pgf}
%%
%% and, on pdftex
%%   \usepackage[utf8]{inputenc}\DeclareUnicodeCharacter{2212}{-}
%%
%% or, on luatex and xetex
%%   \usepackage{unicode-math}
%%
%% Figures using additional raster images can only be included by \input if
%% they are in the same directory as the main LaTeX file. For loading figures
%% from other directories you can use the `import` package
%%   \usepackage{import}
%%
%% and then include the figures with
%%   \import{<path to file>}{<filename>.pgf}
%%
%% Matplotlib used the following preamble
%%
\begingroup%
\makeatletter%
\begin{pgfpicture}%
\pgfpathrectangle{\pgfpointorigin}{\pgfqpoint{5.397490in}{3.000000in}}%
\pgfusepath{use as bounding box, clip}%
\begin{pgfscope}%
\pgfsetbuttcap%
\pgfsetmiterjoin%
\definecolor{currentfill}{rgb}{1.000000,1.000000,1.000000}%
\pgfsetfillcolor{currentfill}%
\pgfsetlinewidth{0.000000pt}%
\definecolor{currentstroke}{rgb}{1.000000,1.000000,1.000000}%
\pgfsetstrokecolor{currentstroke}%
\pgfsetdash{}{0pt}%
\pgfpathmoveto{\pgfqpoint{0.000000in}{0.000000in}}%
\pgfpathlineto{\pgfqpoint{5.397490in}{0.000000in}}%
\pgfpathlineto{\pgfqpoint{5.397490in}{3.000000in}}%
\pgfpathlineto{\pgfqpoint{0.000000in}{3.000000in}}%
\pgfpathclose%
\pgfusepath{fill}%
\end{pgfscope}%
\begin{pgfscope}%
\pgfsetbuttcap%
\pgfsetmiterjoin%
\definecolor{currentfill}{rgb}{1.000000,1.000000,1.000000}%
\pgfsetfillcolor{currentfill}%
\pgfsetlinewidth{0.000000pt}%
\definecolor{currentstroke}{rgb}{0.000000,0.000000,0.000000}%
\pgfsetstrokecolor{currentstroke}%
\pgfsetstrokeopacity{0.000000}%
\pgfsetdash{}{0pt}%
\pgfpathmoveto{\pgfqpoint{0.809624in}{0.450000in}}%
\pgfpathlineto{\pgfqpoint{4.587867in}{0.450000in}}%
\pgfpathlineto{\pgfqpoint{4.587867in}{2.940000in}}%
\pgfpathlineto{\pgfqpoint{0.809624in}{2.940000in}}%
\pgfpathclose%
\pgfusepath{fill}%
\end{pgfscope}%
\begin{pgfscope}%
\pgfsetbuttcap%
\pgfsetroundjoin%
\definecolor{currentfill}{rgb}{0.000000,0.000000,0.000000}%
\pgfsetfillcolor{currentfill}%
\pgfsetlinewidth{0.803000pt}%
\definecolor{currentstroke}{rgb}{0.000000,0.000000,0.000000}%
\pgfsetstrokecolor{currentstroke}%
\pgfsetdash{}{0pt}%
\pgfsys@defobject{currentmarker}{\pgfqpoint{0.000000in}{-0.048611in}}{\pgfqpoint{0.000000in}{0.000000in}}{%
\pgfpathmoveto{\pgfqpoint{0.000000in}{0.000000in}}%
\pgfpathlineto{\pgfqpoint{0.000000in}{-0.048611in}}%
\pgfusepath{stroke,fill}%
}%
\begin{pgfscope}%
\pgfsys@transformshift{0.809624in}{0.450000in}%
\pgfsys@useobject{currentmarker}{}%
\end{pgfscope}%
\end{pgfscope}%
\begin{pgfscope}%
\definecolor{textcolor}{rgb}{0.000000,0.000000,0.000000}%
\pgfsetstrokecolor{textcolor}%
\pgfsetfillcolor{textcolor}%
\pgftext[x=0.809624in,y=0.352778in,,top]{\color{textcolor}\rmfamily\fontsize{10.000000}{12.000000}\selectfont \(\displaystyle 0\)}%
\end{pgfscope}%
\begin{pgfscope}%
\pgfsetbuttcap%
\pgfsetroundjoin%
\definecolor{currentfill}{rgb}{0.000000,0.000000,0.000000}%
\pgfsetfillcolor{currentfill}%
\pgfsetlinewidth{0.803000pt}%
\definecolor{currentstroke}{rgb}{0.000000,0.000000,0.000000}%
\pgfsetstrokecolor{currentstroke}%
\pgfsetdash{}{0pt}%
\pgfsys@defobject{currentmarker}{\pgfqpoint{0.000000in}{-0.048611in}}{\pgfqpoint{0.000000in}{0.000000in}}{%
\pgfpathmoveto{\pgfqpoint{0.000000in}{0.000000in}}%
\pgfpathlineto{\pgfqpoint{0.000000in}{-0.048611in}}%
\pgfusepath{stroke,fill}%
}%
\begin{pgfscope}%
\pgfsys@transformshift{1.568307in}{0.450000in}%
\pgfsys@useobject{currentmarker}{}%
\end{pgfscope}%
\end{pgfscope}%
\begin{pgfscope}%
\definecolor{textcolor}{rgb}{0.000000,0.000000,0.000000}%
\pgfsetstrokecolor{textcolor}%
\pgfsetfillcolor{textcolor}%
\pgftext[x=1.568307in,y=0.352778in,,top]{\color{textcolor}\rmfamily\fontsize{10.000000}{12.000000}\selectfont \(\displaystyle 50\)}%
\end{pgfscope}%
\begin{pgfscope}%
\pgfsetbuttcap%
\pgfsetroundjoin%
\definecolor{currentfill}{rgb}{0.000000,0.000000,0.000000}%
\pgfsetfillcolor{currentfill}%
\pgfsetlinewidth{0.803000pt}%
\definecolor{currentstroke}{rgb}{0.000000,0.000000,0.000000}%
\pgfsetstrokecolor{currentstroke}%
\pgfsetdash{}{0pt}%
\pgfsys@defobject{currentmarker}{\pgfqpoint{0.000000in}{-0.048611in}}{\pgfqpoint{0.000000in}{0.000000in}}{%
\pgfpathmoveto{\pgfqpoint{0.000000in}{0.000000in}}%
\pgfpathlineto{\pgfqpoint{0.000000in}{-0.048611in}}%
\pgfusepath{stroke,fill}%
}%
\begin{pgfscope}%
\pgfsys@transformshift{2.326990in}{0.450000in}%
\pgfsys@useobject{currentmarker}{}%
\end{pgfscope}%
\end{pgfscope}%
\begin{pgfscope}%
\definecolor{textcolor}{rgb}{0.000000,0.000000,0.000000}%
\pgfsetstrokecolor{textcolor}%
\pgfsetfillcolor{textcolor}%
\pgftext[x=2.326990in,y=0.352778in,,top]{\color{textcolor}\rmfamily\fontsize{10.000000}{12.000000}\selectfont \(\displaystyle 100\)}%
\end{pgfscope}%
\begin{pgfscope}%
\pgfsetbuttcap%
\pgfsetroundjoin%
\definecolor{currentfill}{rgb}{0.000000,0.000000,0.000000}%
\pgfsetfillcolor{currentfill}%
\pgfsetlinewidth{0.803000pt}%
\definecolor{currentstroke}{rgb}{0.000000,0.000000,0.000000}%
\pgfsetstrokecolor{currentstroke}%
\pgfsetdash{}{0pt}%
\pgfsys@defobject{currentmarker}{\pgfqpoint{0.000000in}{-0.048611in}}{\pgfqpoint{0.000000in}{0.000000in}}{%
\pgfpathmoveto{\pgfqpoint{0.000000in}{0.000000in}}%
\pgfpathlineto{\pgfqpoint{0.000000in}{-0.048611in}}%
\pgfusepath{stroke,fill}%
}%
\begin{pgfscope}%
\pgfsys@transformshift{3.085673in}{0.450000in}%
\pgfsys@useobject{currentmarker}{}%
\end{pgfscope}%
\end{pgfscope}%
\begin{pgfscope}%
\definecolor{textcolor}{rgb}{0.000000,0.000000,0.000000}%
\pgfsetstrokecolor{textcolor}%
\pgfsetfillcolor{textcolor}%
\pgftext[x=3.085673in,y=0.352778in,,top]{\color{textcolor}\rmfamily\fontsize{10.000000}{12.000000}\selectfont \(\displaystyle 150\)}%
\end{pgfscope}%
\begin{pgfscope}%
\pgfsetbuttcap%
\pgfsetroundjoin%
\definecolor{currentfill}{rgb}{0.000000,0.000000,0.000000}%
\pgfsetfillcolor{currentfill}%
\pgfsetlinewidth{0.803000pt}%
\definecolor{currentstroke}{rgb}{0.000000,0.000000,0.000000}%
\pgfsetstrokecolor{currentstroke}%
\pgfsetdash{}{0pt}%
\pgfsys@defobject{currentmarker}{\pgfqpoint{0.000000in}{-0.048611in}}{\pgfqpoint{0.000000in}{0.000000in}}{%
\pgfpathmoveto{\pgfqpoint{0.000000in}{0.000000in}}%
\pgfpathlineto{\pgfqpoint{0.000000in}{-0.048611in}}%
\pgfusepath{stroke,fill}%
}%
\begin{pgfscope}%
\pgfsys@transformshift{3.844357in}{0.450000in}%
\pgfsys@useobject{currentmarker}{}%
\end{pgfscope}%
\end{pgfscope}%
\begin{pgfscope}%
\definecolor{textcolor}{rgb}{0.000000,0.000000,0.000000}%
\pgfsetstrokecolor{textcolor}%
\pgfsetfillcolor{textcolor}%
\pgftext[x=3.844357in,y=0.352778in,,top]{\color{textcolor}\rmfamily\fontsize{10.000000}{12.000000}\selectfont \(\displaystyle 200\)}%
\end{pgfscope}%
\begin{pgfscope}%
\definecolor{textcolor}{rgb}{0.000000,0.000000,0.000000}%
\pgfsetstrokecolor{textcolor}%
\pgfsetfillcolor{textcolor}%
\pgftext[x=2.698745in,y=0.173766in,,top]{\color{textcolor}\rmfamily\fontsize{10.000000}{12.000000}\selectfont Nr pokolenia}%
\end{pgfscope}%
\begin{pgfscope}%
\pgfsetbuttcap%
\pgfsetroundjoin%
\definecolor{currentfill}{rgb}{0.000000,0.000000,0.000000}%
\pgfsetfillcolor{currentfill}%
\pgfsetlinewidth{0.803000pt}%
\definecolor{currentstroke}{rgb}{0.000000,0.000000,0.000000}%
\pgfsetstrokecolor{currentstroke}%
\pgfsetdash{}{0pt}%
\pgfsys@defobject{currentmarker}{\pgfqpoint{-0.048611in}{0.000000in}}{\pgfqpoint{0.000000in}{0.000000in}}{%
\pgfpathmoveto{\pgfqpoint{0.000000in}{0.000000in}}%
\pgfpathlineto{\pgfqpoint{-0.048611in}{0.000000in}}%
\pgfusepath{stroke,fill}%
}%
\begin{pgfscope}%
\pgfsys@transformshift{0.809624in}{0.588611in}%
\pgfsys@useobject{currentmarker}{}%
\end{pgfscope}%
\end{pgfscope}%
\begin{pgfscope}%
\definecolor{textcolor}{rgb}{0.000000,0.000000,0.000000}%
\pgfsetstrokecolor{textcolor}%
\pgfsetfillcolor{textcolor}%
\pgftext[x=0.365178in, y=0.540386in, left, base]{\color{textcolor}\rmfamily\fontsize{10.000000}{12.000000}\selectfont \(\displaystyle 50000\)}%
\end{pgfscope}%
\begin{pgfscope}%
\pgfsetbuttcap%
\pgfsetroundjoin%
\definecolor{currentfill}{rgb}{0.000000,0.000000,0.000000}%
\pgfsetfillcolor{currentfill}%
\pgfsetlinewidth{0.803000pt}%
\definecolor{currentstroke}{rgb}{0.000000,0.000000,0.000000}%
\pgfsetstrokecolor{currentstroke}%
\pgfsetdash{}{0pt}%
\pgfsys@defobject{currentmarker}{\pgfqpoint{-0.048611in}{0.000000in}}{\pgfqpoint{0.000000in}{0.000000in}}{%
\pgfpathmoveto{\pgfqpoint{0.000000in}{0.000000in}}%
\pgfpathlineto{\pgfqpoint{-0.048611in}{0.000000in}}%
\pgfusepath{stroke,fill}%
}%
\begin{pgfscope}%
\pgfsys@transformshift{0.809624in}{0.924524in}%
\pgfsys@useobject{currentmarker}{}%
\end{pgfscope}%
\end{pgfscope}%
\begin{pgfscope}%
\definecolor{textcolor}{rgb}{0.000000,0.000000,0.000000}%
\pgfsetstrokecolor{textcolor}%
\pgfsetfillcolor{textcolor}%
\pgftext[x=0.295733in, y=0.876298in, left, base]{\color{textcolor}\rmfamily\fontsize{10.000000}{12.000000}\selectfont \(\displaystyle 100000\)}%
\end{pgfscope}%
\begin{pgfscope}%
\pgfsetbuttcap%
\pgfsetroundjoin%
\definecolor{currentfill}{rgb}{0.000000,0.000000,0.000000}%
\pgfsetfillcolor{currentfill}%
\pgfsetlinewidth{0.803000pt}%
\definecolor{currentstroke}{rgb}{0.000000,0.000000,0.000000}%
\pgfsetstrokecolor{currentstroke}%
\pgfsetdash{}{0pt}%
\pgfsys@defobject{currentmarker}{\pgfqpoint{-0.048611in}{0.000000in}}{\pgfqpoint{0.000000in}{0.000000in}}{%
\pgfpathmoveto{\pgfqpoint{0.000000in}{0.000000in}}%
\pgfpathlineto{\pgfqpoint{-0.048611in}{0.000000in}}%
\pgfusepath{stroke,fill}%
}%
\begin{pgfscope}%
\pgfsys@transformshift{0.809624in}{1.260436in}%
\pgfsys@useobject{currentmarker}{}%
\end{pgfscope}%
\end{pgfscope}%
\begin{pgfscope}%
\definecolor{textcolor}{rgb}{0.000000,0.000000,0.000000}%
\pgfsetstrokecolor{textcolor}%
\pgfsetfillcolor{textcolor}%
\pgftext[x=0.295733in, y=1.212211in, left, base]{\color{textcolor}\rmfamily\fontsize{10.000000}{12.000000}\selectfont \(\displaystyle 150000\)}%
\end{pgfscope}%
\begin{pgfscope}%
\pgfsetbuttcap%
\pgfsetroundjoin%
\definecolor{currentfill}{rgb}{0.000000,0.000000,0.000000}%
\pgfsetfillcolor{currentfill}%
\pgfsetlinewidth{0.803000pt}%
\definecolor{currentstroke}{rgb}{0.000000,0.000000,0.000000}%
\pgfsetstrokecolor{currentstroke}%
\pgfsetdash{}{0pt}%
\pgfsys@defobject{currentmarker}{\pgfqpoint{-0.048611in}{0.000000in}}{\pgfqpoint{0.000000in}{0.000000in}}{%
\pgfpathmoveto{\pgfqpoint{0.000000in}{0.000000in}}%
\pgfpathlineto{\pgfqpoint{-0.048611in}{0.000000in}}%
\pgfusepath{stroke,fill}%
}%
\begin{pgfscope}%
\pgfsys@transformshift{0.809624in}{1.596349in}%
\pgfsys@useobject{currentmarker}{}%
\end{pgfscope}%
\end{pgfscope}%
\begin{pgfscope}%
\definecolor{textcolor}{rgb}{0.000000,0.000000,0.000000}%
\pgfsetstrokecolor{textcolor}%
\pgfsetfillcolor{textcolor}%
\pgftext[x=0.295733in, y=1.548124in, left, base]{\color{textcolor}\rmfamily\fontsize{10.000000}{12.000000}\selectfont \(\displaystyle 200000\)}%
\end{pgfscope}%
\begin{pgfscope}%
\pgfsetbuttcap%
\pgfsetroundjoin%
\definecolor{currentfill}{rgb}{0.000000,0.000000,0.000000}%
\pgfsetfillcolor{currentfill}%
\pgfsetlinewidth{0.803000pt}%
\definecolor{currentstroke}{rgb}{0.000000,0.000000,0.000000}%
\pgfsetstrokecolor{currentstroke}%
\pgfsetdash{}{0pt}%
\pgfsys@defobject{currentmarker}{\pgfqpoint{-0.048611in}{0.000000in}}{\pgfqpoint{0.000000in}{0.000000in}}{%
\pgfpathmoveto{\pgfqpoint{0.000000in}{0.000000in}}%
\pgfpathlineto{\pgfqpoint{-0.048611in}{0.000000in}}%
\pgfusepath{stroke,fill}%
}%
\begin{pgfscope}%
\pgfsys@transformshift{0.809624in}{1.932262in}%
\pgfsys@useobject{currentmarker}{}%
\end{pgfscope}%
\end{pgfscope}%
\begin{pgfscope}%
\definecolor{textcolor}{rgb}{0.000000,0.000000,0.000000}%
\pgfsetstrokecolor{textcolor}%
\pgfsetfillcolor{textcolor}%
\pgftext[x=0.295733in, y=1.884037in, left, base]{\color{textcolor}\rmfamily\fontsize{10.000000}{12.000000}\selectfont \(\displaystyle 250000\)}%
\end{pgfscope}%
\begin{pgfscope}%
\pgfsetbuttcap%
\pgfsetroundjoin%
\definecolor{currentfill}{rgb}{0.000000,0.000000,0.000000}%
\pgfsetfillcolor{currentfill}%
\pgfsetlinewidth{0.803000pt}%
\definecolor{currentstroke}{rgb}{0.000000,0.000000,0.000000}%
\pgfsetstrokecolor{currentstroke}%
\pgfsetdash{}{0pt}%
\pgfsys@defobject{currentmarker}{\pgfqpoint{-0.048611in}{0.000000in}}{\pgfqpoint{0.000000in}{0.000000in}}{%
\pgfpathmoveto{\pgfqpoint{0.000000in}{0.000000in}}%
\pgfpathlineto{\pgfqpoint{-0.048611in}{0.000000in}}%
\pgfusepath{stroke,fill}%
}%
\begin{pgfscope}%
\pgfsys@transformshift{0.809624in}{2.268175in}%
\pgfsys@useobject{currentmarker}{}%
\end{pgfscope}%
\end{pgfscope}%
\begin{pgfscope}%
\definecolor{textcolor}{rgb}{0.000000,0.000000,0.000000}%
\pgfsetstrokecolor{textcolor}%
\pgfsetfillcolor{textcolor}%
\pgftext[x=0.295733in, y=2.219949in, left, base]{\color{textcolor}\rmfamily\fontsize{10.000000}{12.000000}\selectfont \(\displaystyle 300000\)}%
\end{pgfscope}%
\begin{pgfscope}%
\pgfsetbuttcap%
\pgfsetroundjoin%
\definecolor{currentfill}{rgb}{0.000000,0.000000,0.000000}%
\pgfsetfillcolor{currentfill}%
\pgfsetlinewidth{0.803000pt}%
\definecolor{currentstroke}{rgb}{0.000000,0.000000,0.000000}%
\pgfsetstrokecolor{currentstroke}%
\pgfsetdash{}{0pt}%
\pgfsys@defobject{currentmarker}{\pgfqpoint{-0.048611in}{0.000000in}}{\pgfqpoint{0.000000in}{0.000000in}}{%
\pgfpathmoveto{\pgfqpoint{0.000000in}{0.000000in}}%
\pgfpathlineto{\pgfqpoint{-0.048611in}{0.000000in}}%
\pgfusepath{stroke,fill}%
}%
\begin{pgfscope}%
\pgfsys@transformshift{0.809624in}{2.604087in}%
\pgfsys@useobject{currentmarker}{}%
\end{pgfscope}%
\end{pgfscope}%
\begin{pgfscope}%
\definecolor{textcolor}{rgb}{0.000000,0.000000,0.000000}%
\pgfsetstrokecolor{textcolor}%
\pgfsetfillcolor{textcolor}%
\pgftext[x=0.295733in, y=2.555862in, left, base]{\color{textcolor}\rmfamily\fontsize{10.000000}{12.000000}\selectfont \(\displaystyle 350000\)}%
\end{pgfscope}%
\begin{pgfscope}%
\pgfsetbuttcap%
\pgfsetroundjoin%
\definecolor{currentfill}{rgb}{0.000000,0.000000,0.000000}%
\pgfsetfillcolor{currentfill}%
\pgfsetlinewidth{0.803000pt}%
\definecolor{currentstroke}{rgb}{0.000000,0.000000,0.000000}%
\pgfsetstrokecolor{currentstroke}%
\pgfsetdash{}{0pt}%
\pgfsys@defobject{currentmarker}{\pgfqpoint{-0.048611in}{0.000000in}}{\pgfqpoint{0.000000in}{0.000000in}}{%
\pgfpathmoveto{\pgfqpoint{0.000000in}{0.000000in}}%
\pgfpathlineto{\pgfqpoint{-0.048611in}{0.000000in}}%
\pgfusepath{stroke,fill}%
}%
\begin{pgfscope}%
\pgfsys@transformshift{0.809624in}{2.940000in}%
\pgfsys@useobject{currentmarker}{}%
\end{pgfscope}%
\end{pgfscope}%
\begin{pgfscope}%
\definecolor{textcolor}{rgb}{0.000000,0.000000,0.000000}%
\pgfsetstrokecolor{textcolor}%
\pgfsetfillcolor{textcolor}%
\pgftext[x=0.295733in, y=2.891775in, left, base]{\color{textcolor}\rmfamily\fontsize{10.000000}{12.000000}\selectfont \(\displaystyle 400000\)}%
\end{pgfscope}%
\begin{pgfscope}%
\definecolor{textcolor}{rgb}{0.000000,0.000000,0.000000}%
\pgfsetstrokecolor{textcolor}%
\pgfsetfillcolor{textcolor}%
\pgftext[x=0.240178in,y=1.695000in,,bottom,rotate=90.000000]{\color{textcolor}\rmfamily\fontsize{10.000000}{12.000000}\selectfont Wartość funkcji przystosowania}%
\end{pgfscope}%
\begin{pgfscope}%
\pgfpathrectangle{\pgfqpoint{0.809624in}{0.450000in}}{\pgfqpoint{3.778243in}{2.490000in}}%
\pgfusepath{clip}%
\pgfsetrectcap%
\pgfsetroundjoin%
\pgfsetlinewidth{1.003750pt}%
\definecolor{currentstroke}{rgb}{0.000000,0.000000,1.000000}%
\pgfsetstrokecolor{currentstroke}%
\pgfsetdash{}{0pt}%
\pgfpathmoveto{\pgfqpoint{0.809624in}{2.798628in}}%
\pgfpathlineto{\pgfqpoint{0.824797in}{2.787482in}}%
\pgfpathlineto{\pgfqpoint{0.839971in}{2.729497in}}%
\pgfpathlineto{\pgfqpoint{0.855144in}{2.702086in}}%
\pgfpathlineto{\pgfqpoint{0.870318in}{2.706621in}}%
\pgfpathlineto{\pgfqpoint{0.885492in}{2.688670in}}%
\pgfpathlineto{\pgfqpoint{0.900666in}{2.696672in}}%
\pgfpathlineto{\pgfqpoint{0.915839in}{2.653245in}}%
\pgfpathlineto{\pgfqpoint{0.931013in}{2.652203in}}%
\pgfpathlineto{\pgfqpoint{0.961360in}{2.626936in}}%
\pgfpathlineto{\pgfqpoint{0.976534in}{2.665815in}}%
\pgfpathlineto{\pgfqpoint{0.991707in}{2.662462in}}%
\pgfpathlineto{\pgfqpoint{1.006881in}{2.633668in}}%
\pgfpathlineto{\pgfqpoint{1.022055in}{2.623080in}}%
\pgfpathlineto{\pgfqpoint{1.037228in}{2.632479in}}%
\pgfpathlineto{\pgfqpoint{1.067576in}{2.634078in}}%
\pgfpathlineto{\pgfqpoint{1.082750in}{2.632136in}}%
\pgfpathlineto{\pgfqpoint{1.097923in}{2.628327in}}%
\pgfpathlineto{\pgfqpoint{1.113097in}{2.665647in}}%
\pgfpathlineto{\pgfqpoint{1.128270in}{2.679459in}}%
\pgfpathlineto{\pgfqpoint{1.143444in}{2.633587in}}%
\pgfpathlineto{\pgfqpoint{1.158618in}{2.623248in}}%
\pgfpathlineto{\pgfqpoint{1.173791in}{2.652647in}}%
\pgfpathlineto{\pgfqpoint{1.188965in}{2.627259in}}%
\pgfpathlineto{\pgfqpoint{1.204139in}{2.689805in}}%
\pgfpathlineto{\pgfqpoint{1.219313in}{2.641777in}}%
\pgfpathlineto{\pgfqpoint{1.234486in}{2.633889in}}%
\pgfpathlineto{\pgfqpoint{1.249660in}{2.634252in}}%
\pgfpathlineto{\pgfqpoint{1.264833in}{2.638344in}}%
\pgfpathlineto{\pgfqpoint{1.280007in}{2.649570in}}%
\pgfpathlineto{\pgfqpoint{1.295181in}{2.626163in}}%
\pgfpathlineto{\pgfqpoint{1.310355in}{2.619795in}}%
\pgfpathlineto{\pgfqpoint{1.325528in}{2.635119in}}%
\pgfpathlineto{\pgfqpoint{1.340702in}{2.624135in}}%
\pgfpathlineto{\pgfqpoint{1.355876in}{2.658008in}}%
\pgfpathlineto{\pgfqpoint{1.371049in}{2.651182in}}%
\pgfpathlineto{\pgfqpoint{1.386223in}{2.636637in}}%
\pgfpathlineto{\pgfqpoint{1.401396in}{2.640111in}}%
\pgfpathlineto{\pgfqpoint{1.431744in}{2.656409in}}%
\pgfpathlineto{\pgfqpoint{1.446917in}{2.627709in}}%
\pgfpathlineto{\pgfqpoint{1.462091in}{2.642932in}}%
\pgfpathlineto{\pgfqpoint{1.477265in}{2.630423in}}%
\pgfpathlineto{\pgfqpoint{1.492439in}{2.639063in}}%
\pgfpathlineto{\pgfqpoint{1.507612in}{2.653218in}}%
\pgfpathlineto{\pgfqpoint{1.522786in}{2.624168in}}%
\pgfpathlineto{\pgfqpoint{1.537960in}{2.639849in}}%
\pgfpathlineto{\pgfqpoint{1.553133in}{2.638559in}}%
\pgfpathlineto{\pgfqpoint{1.568307in}{2.645787in}}%
\pgfpathlineto{\pgfqpoint{1.583481in}{2.616086in}}%
\pgfpathlineto{\pgfqpoint{1.598654in}{2.631820in}}%
\pgfpathlineto{\pgfqpoint{1.613828in}{2.622368in}}%
\pgfpathlineto{\pgfqpoint{1.629001in}{2.605216in}}%
\pgfpathlineto{\pgfqpoint{1.644175in}{2.639150in}}%
\pgfpathlineto{\pgfqpoint{1.659349in}{2.632647in}}%
\pgfpathlineto{\pgfqpoint{1.674522in}{2.635435in}}%
\pgfpathlineto{\pgfqpoint{1.689696in}{2.664619in}}%
\pgfpathlineto{\pgfqpoint{1.704870in}{2.638169in}}%
\pgfpathlineto{\pgfqpoint{1.720043in}{2.644592in}}%
\pgfpathlineto{\pgfqpoint{1.735217in}{2.661535in}}%
\pgfpathlineto{\pgfqpoint{1.750391in}{2.616341in}}%
\pgfpathlineto{\pgfqpoint{1.765564in}{2.618814in}}%
\pgfpathlineto{\pgfqpoint{1.780738in}{2.662845in}}%
\pgfpathlineto{\pgfqpoint{1.795912in}{2.648475in}}%
\pgfpathlineto{\pgfqpoint{1.811085in}{2.605035in}}%
\pgfpathlineto{\pgfqpoint{1.826259in}{2.616650in}}%
\pgfpathlineto{\pgfqpoint{1.841433in}{2.597967in}}%
\pgfpathlineto{\pgfqpoint{1.856606in}{2.643900in}}%
\pgfpathlineto{\pgfqpoint{1.871780in}{2.634857in}}%
\pgfpathlineto{\pgfqpoint{1.886954in}{2.658156in}}%
\pgfpathlineto{\pgfqpoint{1.902127in}{2.603812in}}%
\pgfpathlineto{\pgfqpoint{1.917301in}{2.633003in}}%
\pgfpathlineto{\pgfqpoint{1.932475in}{2.616288in}}%
\pgfpathlineto{\pgfqpoint{1.947648in}{2.649079in}}%
\pgfpathlineto{\pgfqpoint{1.962822in}{2.634407in}}%
\pgfpathlineto{\pgfqpoint{1.977996in}{2.628763in}}%
\pgfpathlineto{\pgfqpoint{1.993170in}{2.650040in}}%
\pgfpathlineto{\pgfqpoint{2.008343in}{2.627064in}}%
\pgfpathlineto{\pgfqpoint{2.023517in}{2.606385in}}%
\pgfpathlineto{\pgfqpoint{2.038690in}{2.650625in}}%
\pgfpathlineto{\pgfqpoint{2.053864in}{2.622529in}}%
\pgfpathlineto{\pgfqpoint{2.069038in}{2.666722in}}%
\pgfpathlineto{\pgfqpoint{2.084211in}{2.679977in}}%
\pgfpathlineto{\pgfqpoint{2.099385in}{2.643483in}}%
\pgfpathlineto{\pgfqpoint{2.114559in}{2.639714in}}%
\pgfpathlineto{\pgfqpoint{2.129733in}{2.630772in}}%
\pgfpathlineto{\pgfqpoint{2.144906in}{2.630759in}}%
\pgfpathlineto{\pgfqpoint{2.160080in}{2.635791in}}%
\pgfpathlineto{\pgfqpoint{2.175254in}{2.618639in}}%
\pgfpathlineto{\pgfqpoint{2.190427in}{2.636006in}}%
\pgfpathlineto{\pgfqpoint{2.205601in}{2.621897in}}%
\pgfpathlineto{\pgfqpoint{2.220775in}{2.662946in}}%
\pgfpathlineto{\pgfqpoint{2.235948in}{2.617174in}}%
\pgfpathlineto{\pgfqpoint{2.251122in}{2.644323in}}%
\pgfpathlineto{\pgfqpoint{2.266296in}{2.708321in}}%
\pgfpathlineto{\pgfqpoint{2.281469in}{2.596885in}}%
\pgfpathlineto{\pgfqpoint{2.296643in}{2.635986in}}%
\pgfpathlineto{\pgfqpoint{2.311816in}{2.624531in}}%
\pgfpathlineto{\pgfqpoint{2.326990in}{2.606257in}}%
\pgfpathlineto{\pgfqpoint{2.342164in}{2.646325in}}%
\pgfpathlineto{\pgfqpoint{2.357337in}{2.638263in}}%
\pgfpathlineto{\pgfqpoint{2.372511in}{2.645808in}}%
\pgfpathlineto{\pgfqpoint{2.387685in}{2.642153in}}%
\pgfpathlineto{\pgfqpoint{2.402858in}{2.636348in}}%
\pgfpathlineto{\pgfqpoint{2.418032in}{2.603590in}}%
\pgfpathlineto{\pgfqpoint{2.433206in}{2.620513in}}%
\pgfpathlineto{\pgfqpoint{2.448379in}{2.617389in}}%
\pgfpathlineto{\pgfqpoint{2.463553in}{2.629644in}}%
\pgfpathlineto{\pgfqpoint{2.478727in}{2.632082in}}%
\pgfpathlineto{\pgfqpoint{2.493900in}{2.617685in}}%
\pgfpathlineto{\pgfqpoint{2.509074in}{2.611995in}}%
\pgfpathlineto{\pgfqpoint{2.524248in}{2.614420in}}%
\pgfpathlineto{\pgfqpoint{2.539421in}{2.597409in}}%
\pgfpathlineto{\pgfqpoint{2.554595in}{2.656530in}}%
\pgfpathlineto{\pgfqpoint{2.569769in}{2.639755in}}%
\pgfpathlineto{\pgfqpoint{2.584942in}{2.628562in}}%
\pgfpathlineto{\pgfqpoint{2.600116in}{2.621098in}}%
\pgfpathlineto{\pgfqpoint{2.615290in}{2.621051in}}%
\pgfpathlineto{\pgfqpoint{2.645637in}{2.616892in}}%
\pgfpathlineto{\pgfqpoint{2.660811in}{2.666318in}}%
\pgfpathlineto{\pgfqpoint{2.675984in}{2.661804in}}%
\pgfpathlineto{\pgfqpoint{2.691158in}{2.623402in}}%
\pgfpathlineto{\pgfqpoint{2.706332in}{2.636731in}}%
\pgfpathlineto{\pgfqpoint{2.721506in}{2.671223in}}%
\pgfpathlineto{\pgfqpoint{2.736679in}{2.673655in}}%
\pgfpathlineto{\pgfqpoint{2.751853in}{2.613923in}}%
\pgfpathlineto{\pgfqpoint{2.767027in}{2.627023in}}%
\pgfpathlineto{\pgfqpoint{2.782200in}{2.624618in}}%
\pgfpathlineto{\pgfqpoint{2.797374in}{2.626264in}}%
\pgfpathlineto{\pgfqpoint{2.812547in}{2.633876in}}%
\pgfpathlineto{\pgfqpoint{2.827721in}{2.620742in}}%
\pgfpathlineto{\pgfqpoint{2.842895in}{2.626600in}}%
\pgfpathlineto{\pgfqpoint{2.873242in}{2.631901in}}%
\pgfpathlineto{\pgfqpoint{2.888416in}{2.612344in}}%
\pgfpathlineto{\pgfqpoint{2.918763in}{2.621561in}}%
\pgfpathlineto{\pgfqpoint{2.933937in}{2.614675in}}%
\pgfpathlineto{\pgfqpoint{2.949110in}{2.632391in}}%
\pgfpathlineto{\pgfqpoint{2.964284in}{2.654367in}}%
\pgfpathlineto{\pgfqpoint{2.979458in}{2.603046in}}%
\pgfpathlineto{\pgfqpoint{2.994632in}{2.647957in}}%
\pgfpathlineto{\pgfqpoint{3.009805in}{2.660816in}}%
\pgfpathlineto{\pgfqpoint{3.024979in}{2.647198in}}%
\pgfpathlineto{\pgfqpoint{3.040153in}{2.667407in}}%
\pgfpathlineto{\pgfqpoint{3.055326in}{2.636086in}}%
\pgfpathlineto{\pgfqpoint{3.070500in}{2.654326in}}%
\pgfpathlineto{\pgfqpoint{3.085673in}{2.646540in}}%
\pgfpathlineto{\pgfqpoint{3.100847in}{2.618558in}}%
\pgfpathlineto{\pgfqpoint{3.116021in}{2.624786in}}%
\pgfpathlineto{\pgfqpoint{3.131194in}{2.586761in}}%
\pgfpathlineto{\pgfqpoint{3.146368in}{2.629845in}}%
\pgfpathlineto{\pgfqpoint{3.161542in}{2.683020in}}%
\pgfpathlineto{\pgfqpoint{3.176715in}{2.623349in}}%
\pgfpathlineto{\pgfqpoint{3.207063in}{2.644061in}}%
\pgfpathlineto{\pgfqpoint{3.222236in}{2.636577in}}%
\pgfpathlineto{\pgfqpoint{3.237410in}{2.631955in}}%
\pgfpathlineto{\pgfqpoint{3.252584in}{2.673379in}}%
\pgfpathlineto{\pgfqpoint{3.267757in}{2.602461in}}%
\pgfpathlineto{\pgfqpoint{3.282931in}{2.612834in}}%
\pgfpathlineto{\pgfqpoint{3.298105in}{2.620352in}}%
\pgfpathlineto{\pgfqpoint{3.313278in}{2.617947in}}%
\pgfpathlineto{\pgfqpoint{3.328452in}{2.603617in}}%
\pgfpathlineto{\pgfqpoint{3.343626in}{2.651216in}}%
\pgfpathlineto{\pgfqpoint{3.373973in}{2.633453in}}%
\pgfpathlineto{\pgfqpoint{3.389147in}{2.633816in}}%
\pgfpathlineto{\pgfqpoint{3.404321in}{2.756901in}}%
\pgfpathlineto{\pgfqpoint{3.419494in}{2.624175in}}%
\pgfpathlineto{\pgfqpoint{3.434668in}{2.587748in}}%
\pgfpathlineto{\pgfqpoint{3.449842in}{2.655536in}}%
\pgfpathlineto{\pgfqpoint{3.465015in}{2.634709in}}%
\pgfpathlineto{\pgfqpoint{3.480189in}{2.676725in}}%
\pgfpathlineto{\pgfqpoint{3.495363in}{2.641481in}}%
\pgfpathlineto{\pgfqpoint{3.510536in}{2.644142in}}%
\pgfpathlineto{\pgfqpoint{3.525710in}{2.649462in}}%
\pgfpathlineto{\pgfqpoint{3.540884in}{2.661723in}}%
\pgfpathlineto{\pgfqpoint{3.556057in}{2.681623in}}%
\pgfpathlineto{\pgfqpoint{3.571231in}{2.623295in}}%
\pgfpathlineto{\pgfqpoint{3.586404in}{2.649913in}}%
\pgfpathlineto{\pgfqpoint{3.601578in}{2.652902in}}%
\pgfpathlineto{\pgfqpoint{3.616752in}{2.636059in}}%
\pgfpathlineto{\pgfqpoint{3.631925in}{2.641669in}}%
\pgfpathlineto{\pgfqpoint{3.647099in}{2.641031in}}%
\pgfpathlineto{\pgfqpoint{3.662273in}{2.622509in}}%
\pgfpathlineto{\pgfqpoint{3.692620in}{2.687743in}}%
\pgfpathlineto{\pgfqpoint{3.707794in}{2.632620in}}%
\pgfpathlineto{\pgfqpoint{3.722967in}{2.631357in}}%
\pgfpathlineto{\pgfqpoint{3.738141in}{2.635200in}}%
\pgfpathlineto{\pgfqpoint{3.753315in}{2.647071in}}%
\pgfpathlineto{\pgfqpoint{3.768488in}{2.624014in}}%
\pgfpathlineto{\pgfqpoint{3.783662in}{2.630698in}}%
\pgfpathlineto{\pgfqpoint{3.798836in}{2.625344in}}%
\pgfpathlineto{\pgfqpoint{3.829183in}{2.651632in}}%
\pgfpathlineto{\pgfqpoint{3.844357in}{2.640823in}}%
\pgfpathlineto{\pgfqpoint{3.859530in}{2.624329in}}%
\pgfpathlineto{\pgfqpoint{3.874704in}{2.647642in}}%
\pgfpathlineto{\pgfqpoint{3.889878in}{2.620728in}}%
\pgfpathlineto{\pgfqpoint{3.905052in}{2.623664in}}%
\pgfpathlineto{\pgfqpoint{3.920225in}{2.610268in}}%
\pgfpathlineto{\pgfqpoint{3.935399in}{2.696860in}}%
\pgfpathlineto{\pgfqpoint{3.950573in}{2.653587in}}%
\pgfpathlineto{\pgfqpoint{3.965746in}{2.640514in}}%
\pgfpathlineto{\pgfqpoint{3.980920in}{2.625700in}}%
\pgfpathlineto{\pgfqpoint{3.996094in}{2.620574in}}%
\pgfpathlineto{\pgfqpoint{4.026441in}{2.636906in}}%
\pgfpathlineto{\pgfqpoint{4.041614in}{2.639721in}}%
\pgfpathlineto{\pgfqpoint{4.056788in}{2.656087in}}%
\pgfpathlineto{\pgfqpoint{4.071962in}{2.659923in}}%
\pgfpathlineto{\pgfqpoint{4.087136in}{2.649315in}}%
\pgfpathlineto{\pgfqpoint{4.102309in}{2.636798in}}%
\pgfpathlineto{\pgfqpoint{4.117483in}{2.626096in}}%
\pgfpathlineto{\pgfqpoint{4.132657in}{2.635112in}}%
\pgfpathlineto{\pgfqpoint{4.147830in}{2.638377in}}%
\pgfpathlineto{\pgfqpoint{4.163004in}{2.643920in}}%
\pgfpathlineto{\pgfqpoint{4.178178in}{2.642764in}}%
\pgfpathlineto{\pgfqpoint{4.193351in}{2.650799in}}%
\pgfpathlineto{\pgfqpoint{4.208525in}{2.628495in}}%
\pgfpathlineto{\pgfqpoint{4.238872in}{2.647628in}}%
\pgfpathlineto{\pgfqpoint{4.254046in}{2.633500in}}%
\pgfpathlineto{\pgfqpoint{4.269220in}{2.615146in}}%
\pgfpathlineto{\pgfqpoint{4.284393in}{2.615683in}}%
\pgfpathlineto{\pgfqpoint{4.299567in}{2.646849in}}%
\pgfpathlineto{\pgfqpoint{4.314741in}{2.645082in}}%
\pgfpathlineto{\pgfqpoint{4.329914in}{2.624645in}}%
\pgfpathlineto{\pgfqpoint{4.345088in}{2.613258in}}%
\pgfpathlineto{\pgfqpoint{4.360262in}{2.641542in}}%
\pgfpathlineto{\pgfqpoint{4.390609in}{2.618525in}}%
\pgfpathlineto{\pgfqpoint{4.405782in}{2.634991in}}%
\pgfpathlineto{\pgfqpoint{4.420956in}{2.623530in}}%
\pgfpathlineto{\pgfqpoint{4.436130in}{2.613815in}}%
\pgfpathlineto{\pgfqpoint{4.451303in}{2.637564in}}%
\pgfpathlineto{\pgfqpoint{4.466477in}{2.651733in}}%
\pgfpathlineto{\pgfqpoint{4.481651in}{2.646795in}}%
\pgfpathlineto{\pgfqpoint{4.496824in}{2.633318in}}%
\pgfpathlineto{\pgfqpoint{4.527172in}{2.645304in}}%
\pgfpathlineto{\pgfqpoint{4.542345in}{2.636745in}}%
\pgfpathlineto{\pgfqpoint{4.557519in}{2.618626in}}%
\pgfpathlineto{\pgfqpoint{4.572693in}{2.668173in}}%
\pgfpathlineto{\pgfqpoint{4.587867in}{2.635609in}}%
\pgfpathlineto{\pgfqpoint{4.587867in}{2.635609in}}%
\pgfusepath{stroke}%
\end{pgfscope}%
\begin{pgfscope}%
\pgfpathrectangle{\pgfqpoint{0.809624in}{0.450000in}}{\pgfqpoint{3.778243in}{2.490000in}}%
\pgfusepath{clip}%
\pgfsetrectcap%
\pgfsetroundjoin%
\pgfsetlinewidth{1.003750pt}%
\definecolor{currentstroke}{rgb}{0.411765,0.411765,0.411765}%
\pgfsetstrokecolor{currentstroke}%
\pgfsetdash{}{0pt}%
\pgfpathmoveto{\pgfqpoint{0.809624in}{2.196302in}}%
\pgfpathlineto{\pgfqpoint{0.824797in}{2.171927in}}%
\pgfpathlineto{\pgfqpoint{0.855144in}{2.114702in}}%
\pgfpathlineto{\pgfqpoint{0.870318in}{2.110022in}}%
\pgfpathlineto{\pgfqpoint{0.885492in}{2.075741in}}%
\pgfpathlineto{\pgfqpoint{0.915839in}{2.053665in}}%
\pgfpathlineto{\pgfqpoint{0.931013in}{2.049280in}}%
\pgfpathlineto{\pgfqpoint{0.946187in}{2.066193in}}%
\pgfpathlineto{\pgfqpoint{0.961360in}{2.045529in}}%
\pgfpathlineto{\pgfqpoint{0.976534in}{2.041870in}}%
\pgfpathlineto{\pgfqpoint{0.991707in}{2.040175in}}%
\pgfpathlineto{\pgfqpoint{1.006881in}{2.060877in}}%
\pgfpathlineto{\pgfqpoint{1.022055in}{2.052931in}}%
\pgfpathlineto{\pgfqpoint{1.037228in}{2.042484in}}%
\pgfpathlineto{\pgfqpoint{1.052402in}{2.036696in}}%
\pgfpathlineto{\pgfqpoint{1.067576in}{2.050496in}}%
\pgfpathlineto{\pgfqpoint{1.082750in}{2.024572in}}%
\pgfpathlineto{\pgfqpoint{1.097923in}{2.019990in}}%
\pgfpathlineto{\pgfqpoint{1.128270in}{2.037753in}}%
\pgfpathlineto{\pgfqpoint{1.143444in}{2.031977in}}%
\pgfpathlineto{\pgfqpoint{1.158618in}{2.033729in}}%
\pgfpathlineto{\pgfqpoint{1.173791in}{2.024989in}}%
\pgfpathlineto{\pgfqpoint{1.188965in}{2.035411in}}%
\pgfpathlineto{\pgfqpoint{1.204139in}{2.043277in}}%
\pgfpathlineto{\pgfqpoint{1.219313in}{2.024270in}}%
\pgfpathlineto{\pgfqpoint{1.234486in}{2.032405in}}%
\pgfpathlineto{\pgfqpoint{1.249660in}{2.033616in}}%
\pgfpathlineto{\pgfqpoint{1.264833in}{2.018221in}}%
\pgfpathlineto{\pgfqpoint{1.280007in}{2.017224in}}%
\pgfpathlineto{\pgfqpoint{1.295181in}{2.020190in}}%
\pgfpathlineto{\pgfqpoint{1.310355in}{2.035014in}}%
\pgfpathlineto{\pgfqpoint{1.325528in}{2.031849in}}%
\pgfpathlineto{\pgfqpoint{1.340702in}{2.027232in}}%
\pgfpathlineto{\pgfqpoint{1.355876in}{2.018981in}}%
\pgfpathlineto{\pgfqpoint{1.371049in}{2.001066in}}%
\pgfpathlineto{\pgfqpoint{1.386223in}{2.034416in}}%
\pgfpathlineto{\pgfqpoint{1.401396in}{2.027755in}}%
\pgfpathlineto{\pgfqpoint{1.416570in}{2.027587in}}%
\pgfpathlineto{\pgfqpoint{1.431744in}{2.034856in}}%
\pgfpathlineto{\pgfqpoint{1.446917in}{2.044906in}}%
\pgfpathlineto{\pgfqpoint{1.462091in}{2.028783in}}%
\pgfpathlineto{\pgfqpoint{1.477265in}{2.028207in}}%
\pgfpathlineto{\pgfqpoint{1.492439in}{2.015479in}}%
\pgfpathlineto{\pgfqpoint{1.507612in}{2.026819in}}%
\pgfpathlineto{\pgfqpoint{1.522786in}{2.023493in}}%
\pgfpathlineto{\pgfqpoint{1.537960in}{2.038995in}}%
\pgfpathlineto{\pgfqpoint{1.553133in}{2.032054in}}%
\pgfpathlineto{\pgfqpoint{1.568307in}{2.030225in}}%
\pgfpathlineto{\pgfqpoint{1.583481in}{2.029760in}}%
\pgfpathlineto{\pgfqpoint{1.598654in}{2.032039in}}%
\pgfpathlineto{\pgfqpoint{1.613828in}{2.018753in}}%
\pgfpathlineto{\pgfqpoint{1.629001in}{2.028938in}}%
\pgfpathlineto{\pgfqpoint{1.644175in}{2.027634in}}%
\pgfpathlineto{\pgfqpoint{1.659349in}{2.030557in}}%
\pgfpathlineto{\pgfqpoint{1.674522in}{2.027777in}}%
\pgfpathlineto{\pgfqpoint{1.689696in}{2.030541in}}%
\pgfpathlineto{\pgfqpoint{1.704870in}{2.021249in}}%
\pgfpathlineto{\pgfqpoint{1.720043in}{2.036185in}}%
\pgfpathlineto{\pgfqpoint{1.735217in}{2.039909in}}%
\pgfpathlineto{\pgfqpoint{1.750391in}{2.026317in}}%
\pgfpathlineto{\pgfqpoint{1.765564in}{2.022292in}}%
\pgfpathlineto{\pgfqpoint{1.780738in}{2.030422in}}%
\pgfpathlineto{\pgfqpoint{1.795912in}{2.018871in}}%
\pgfpathlineto{\pgfqpoint{1.811085in}{2.032055in}}%
\pgfpathlineto{\pgfqpoint{1.826259in}{2.023360in}}%
\pgfpathlineto{\pgfqpoint{1.841433in}{2.030357in}}%
\pgfpathlineto{\pgfqpoint{1.856606in}{2.019588in}}%
\pgfpathlineto{\pgfqpoint{1.871780in}{2.020156in}}%
\pgfpathlineto{\pgfqpoint{1.886954in}{2.010145in}}%
\pgfpathlineto{\pgfqpoint{1.902127in}{2.027348in}}%
\pgfpathlineto{\pgfqpoint{1.917301in}{2.026332in}}%
\pgfpathlineto{\pgfqpoint{1.932475in}{2.030018in}}%
\pgfpathlineto{\pgfqpoint{1.947648in}{2.029406in}}%
\pgfpathlineto{\pgfqpoint{1.962822in}{2.021369in}}%
\pgfpathlineto{\pgfqpoint{1.977996in}{2.026956in}}%
\pgfpathlineto{\pgfqpoint{1.993170in}{2.017902in}}%
\pgfpathlineto{\pgfqpoint{2.008343in}{2.010792in}}%
\pgfpathlineto{\pgfqpoint{2.023517in}{2.034649in}}%
\pgfpathlineto{\pgfqpoint{2.038690in}{2.043434in}}%
\pgfpathlineto{\pgfqpoint{2.053864in}{2.030558in}}%
\pgfpathlineto{\pgfqpoint{2.069038in}{2.034207in}}%
\pgfpathlineto{\pgfqpoint{2.084211in}{2.027544in}}%
\pgfpathlineto{\pgfqpoint{2.099385in}{2.027059in}}%
\pgfpathlineto{\pgfqpoint{2.114559in}{2.021343in}}%
\pgfpathlineto{\pgfqpoint{2.129733in}{2.030656in}}%
\pgfpathlineto{\pgfqpoint{2.144906in}{2.034175in}}%
\pgfpathlineto{\pgfqpoint{2.175254in}{2.028064in}}%
\pgfpathlineto{\pgfqpoint{2.190427in}{2.026250in}}%
\pgfpathlineto{\pgfqpoint{2.205601in}{2.038698in}}%
\pgfpathlineto{\pgfqpoint{2.220775in}{2.027267in}}%
\pgfpathlineto{\pgfqpoint{2.235948in}{2.030491in}}%
\pgfpathlineto{\pgfqpoint{2.251122in}{2.035260in}}%
\pgfpathlineto{\pgfqpoint{2.266296in}{2.028389in}}%
\pgfpathlineto{\pgfqpoint{2.281469in}{2.033286in}}%
\pgfpathlineto{\pgfqpoint{2.296643in}{2.022616in}}%
\pgfpathlineto{\pgfqpoint{2.311816in}{2.032491in}}%
\pgfpathlineto{\pgfqpoint{2.326990in}{2.032872in}}%
\pgfpathlineto{\pgfqpoint{2.342164in}{2.029392in}}%
\pgfpathlineto{\pgfqpoint{2.357337in}{2.030839in}}%
\pgfpathlineto{\pgfqpoint{2.372511in}{2.038022in}}%
\pgfpathlineto{\pgfqpoint{2.387685in}{2.034685in}}%
\pgfpathlineto{\pgfqpoint{2.402858in}{2.029866in}}%
\pgfpathlineto{\pgfqpoint{2.418032in}{2.035576in}}%
\pgfpathlineto{\pgfqpoint{2.433206in}{2.025464in}}%
\pgfpathlineto{\pgfqpoint{2.448379in}{2.033682in}}%
\pgfpathlineto{\pgfqpoint{2.463553in}{2.024734in}}%
\pgfpathlineto{\pgfqpoint{2.478727in}{2.022581in}}%
\pgfpathlineto{\pgfqpoint{2.493900in}{2.025979in}}%
\pgfpathlineto{\pgfqpoint{2.509074in}{2.024565in}}%
\pgfpathlineto{\pgfqpoint{2.524248in}{2.013416in}}%
\pgfpathlineto{\pgfqpoint{2.539421in}{2.026167in}}%
\pgfpathlineto{\pgfqpoint{2.554595in}{2.025629in}}%
\pgfpathlineto{\pgfqpoint{2.569769in}{2.020529in}}%
\pgfpathlineto{\pgfqpoint{2.584942in}{2.027713in}}%
\pgfpathlineto{\pgfqpoint{2.600116in}{2.028979in}}%
\pgfpathlineto{\pgfqpoint{2.615290in}{2.022868in}}%
\pgfpathlineto{\pgfqpoint{2.630463in}{2.024972in}}%
\pgfpathlineto{\pgfqpoint{2.645637in}{2.029156in}}%
\pgfpathlineto{\pgfqpoint{2.660811in}{2.014245in}}%
\pgfpathlineto{\pgfqpoint{2.675984in}{2.033466in}}%
\pgfpathlineto{\pgfqpoint{2.691158in}{2.024070in}}%
\pgfpathlineto{\pgfqpoint{2.706332in}{2.023797in}}%
\pgfpathlineto{\pgfqpoint{2.721506in}{2.037742in}}%
\pgfpathlineto{\pgfqpoint{2.736679in}{2.033239in}}%
\pgfpathlineto{\pgfqpoint{2.751853in}{2.026628in}}%
\pgfpathlineto{\pgfqpoint{2.767027in}{2.015651in}}%
\pgfpathlineto{\pgfqpoint{2.782200in}{2.028868in}}%
\pgfpathlineto{\pgfqpoint{2.797374in}{2.028943in}}%
\pgfpathlineto{\pgfqpoint{2.827721in}{2.038090in}}%
\pgfpathlineto{\pgfqpoint{2.842895in}{2.020374in}}%
\pgfpathlineto{\pgfqpoint{2.858068in}{2.031546in}}%
\pgfpathlineto{\pgfqpoint{2.873242in}{2.022119in}}%
\pgfpathlineto{\pgfqpoint{2.888416in}{2.033558in}}%
\pgfpathlineto{\pgfqpoint{2.903589in}{2.031405in}}%
\pgfpathlineto{\pgfqpoint{2.918763in}{2.016858in}}%
\pgfpathlineto{\pgfqpoint{2.933937in}{2.016994in}}%
\pgfpathlineto{\pgfqpoint{2.949110in}{2.021880in}}%
\pgfpathlineto{\pgfqpoint{2.964284in}{2.020542in}}%
\pgfpathlineto{\pgfqpoint{2.979458in}{2.031140in}}%
\pgfpathlineto{\pgfqpoint{2.994632in}{2.036312in}}%
\pgfpathlineto{\pgfqpoint{3.009805in}{2.034466in}}%
\pgfpathlineto{\pgfqpoint{3.024979in}{2.026488in}}%
\pgfpathlineto{\pgfqpoint{3.040153in}{2.023491in}}%
\pgfpathlineto{\pgfqpoint{3.055326in}{2.023459in}}%
\pgfpathlineto{\pgfqpoint{3.070500in}{2.017340in}}%
\pgfpathlineto{\pgfqpoint{3.085673in}{2.026954in}}%
\pgfpathlineto{\pgfqpoint{3.100847in}{2.030442in}}%
\pgfpathlineto{\pgfqpoint{3.116021in}{2.027167in}}%
\pgfpathlineto{\pgfqpoint{3.131194in}{2.035763in}}%
\pgfpathlineto{\pgfqpoint{3.146368in}{2.025484in}}%
\pgfpathlineto{\pgfqpoint{3.161542in}{2.019537in}}%
\pgfpathlineto{\pgfqpoint{3.176715in}{2.026700in}}%
\pgfpathlineto{\pgfqpoint{3.191889in}{2.025183in}}%
\pgfpathlineto{\pgfqpoint{3.207063in}{2.042431in}}%
\pgfpathlineto{\pgfqpoint{3.222236in}{2.032476in}}%
\pgfpathlineto{\pgfqpoint{3.237410in}{2.029529in}}%
\pgfpathlineto{\pgfqpoint{3.252584in}{2.020352in}}%
\pgfpathlineto{\pgfqpoint{3.267757in}{2.034189in}}%
\pgfpathlineto{\pgfqpoint{3.282931in}{2.039450in}}%
\pgfpathlineto{\pgfqpoint{3.298105in}{2.030093in}}%
\pgfpathlineto{\pgfqpoint{3.313278in}{2.026168in}}%
\pgfpathlineto{\pgfqpoint{3.343626in}{2.027657in}}%
\pgfpathlineto{\pgfqpoint{3.358799in}{2.017347in}}%
\pgfpathlineto{\pgfqpoint{3.373973in}{2.021040in}}%
\pgfpathlineto{\pgfqpoint{3.389147in}{2.020494in}}%
\pgfpathlineto{\pgfqpoint{3.404321in}{2.017629in}}%
\pgfpathlineto{\pgfqpoint{3.419494in}{2.027189in}}%
\pgfpathlineto{\pgfqpoint{3.434668in}{2.015024in}}%
\pgfpathlineto{\pgfqpoint{3.449842in}{2.022579in}}%
\pgfpathlineto{\pgfqpoint{3.465015in}{2.016982in}}%
\pgfpathlineto{\pgfqpoint{3.480189in}{2.013532in}}%
\pgfpathlineto{\pgfqpoint{3.495363in}{2.030043in}}%
\pgfpathlineto{\pgfqpoint{3.510536in}{2.018508in}}%
\pgfpathlineto{\pgfqpoint{3.525710in}{2.015883in}}%
\pgfpathlineto{\pgfqpoint{3.540884in}{2.040625in}}%
\pgfpathlineto{\pgfqpoint{3.556057in}{2.025180in}}%
\pgfpathlineto{\pgfqpoint{3.571231in}{2.037369in}}%
\pgfpathlineto{\pgfqpoint{3.586404in}{2.031345in}}%
\pgfpathlineto{\pgfqpoint{3.601578in}{2.023924in}}%
\pgfpathlineto{\pgfqpoint{3.616752in}{2.038050in}}%
\pgfpathlineto{\pgfqpoint{3.631925in}{2.023773in}}%
\pgfpathlineto{\pgfqpoint{3.647099in}{2.032993in}}%
\pgfpathlineto{\pgfqpoint{3.662273in}{2.031429in}}%
\pgfpathlineto{\pgfqpoint{3.677446in}{2.033716in}}%
\pgfpathlineto{\pgfqpoint{3.707794in}{2.018100in}}%
\pgfpathlineto{\pgfqpoint{3.722967in}{2.020544in}}%
\pgfpathlineto{\pgfqpoint{3.738141in}{2.028362in}}%
\pgfpathlineto{\pgfqpoint{3.753315in}{2.020004in}}%
\pgfpathlineto{\pgfqpoint{3.768488in}{2.032711in}}%
\pgfpathlineto{\pgfqpoint{3.783662in}{2.031105in}}%
\pgfpathlineto{\pgfqpoint{3.798836in}{2.038192in}}%
\pgfpathlineto{\pgfqpoint{3.829183in}{2.015923in}}%
\pgfpathlineto{\pgfqpoint{3.844357in}{2.031841in}}%
\pgfpathlineto{\pgfqpoint{3.859530in}{2.026456in}}%
\pgfpathlineto{\pgfqpoint{3.874704in}{2.010343in}}%
\pgfpathlineto{\pgfqpoint{3.889878in}{2.034916in}}%
\pgfpathlineto{\pgfqpoint{3.905052in}{2.026380in}}%
\pgfpathlineto{\pgfqpoint{3.920225in}{2.030121in}}%
\pgfpathlineto{\pgfqpoint{3.935399in}{2.030667in}}%
\pgfpathlineto{\pgfqpoint{3.965746in}{2.027066in}}%
\pgfpathlineto{\pgfqpoint{3.980920in}{2.030612in}}%
\pgfpathlineto{\pgfqpoint{3.996094in}{2.018532in}}%
\pgfpathlineto{\pgfqpoint{4.011267in}{2.034207in}}%
\pgfpathlineto{\pgfqpoint{4.026441in}{2.027995in}}%
\pgfpathlineto{\pgfqpoint{4.041614in}{2.033053in}}%
\pgfpathlineto{\pgfqpoint{4.056788in}{2.032512in}}%
\pgfpathlineto{\pgfqpoint{4.071962in}{2.019878in}}%
\pgfpathlineto{\pgfqpoint{4.087136in}{2.025847in}}%
\pgfpathlineto{\pgfqpoint{4.102309in}{2.025062in}}%
\pgfpathlineto{\pgfqpoint{4.117483in}{2.026984in}}%
\pgfpathlineto{\pgfqpoint{4.132657in}{2.016761in}}%
\pgfpathlineto{\pgfqpoint{4.147830in}{2.025017in}}%
\pgfpathlineto{\pgfqpoint{4.163004in}{2.012664in}}%
\pgfpathlineto{\pgfqpoint{4.178178in}{2.026336in}}%
\pgfpathlineto{\pgfqpoint{4.193351in}{2.027330in}}%
\pgfpathlineto{\pgfqpoint{4.208525in}{2.030524in}}%
\pgfpathlineto{\pgfqpoint{4.223699in}{2.024174in}}%
\pgfpathlineto{\pgfqpoint{4.238872in}{2.020420in}}%
\pgfpathlineto{\pgfqpoint{4.254046in}{2.024463in}}%
\pgfpathlineto{\pgfqpoint{4.269220in}{2.021863in}}%
\pgfpathlineto{\pgfqpoint{4.284393in}{2.010583in}}%
\pgfpathlineto{\pgfqpoint{4.299567in}{2.018665in}}%
\pgfpathlineto{\pgfqpoint{4.314741in}{2.028481in}}%
\pgfpathlineto{\pgfqpoint{4.345088in}{2.041366in}}%
\pgfpathlineto{\pgfqpoint{4.375435in}{2.028406in}}%
\pgfpathlineto{\pgfqpoint{4.390609in}{2.036310in}}%
\pgfpathlineto{\pgfqpoint{4.405782in}{2.023309in}}%
\pgfpathlineto{\pgfqpoint{4.420956in}{2.030718in}}%
\pgfpathlineto{\pgfqpoint{4.451303in}{2.022084in}}%
\pgfpathlineto{\pgfqpoint{4.466477in}{2.024995in}}%
\pgfpathlineto{\pgfqpoint{4.481651in}{2.020244in}}%
\pgfpathlineto{\pgfqpoint{4.496824in}{2.012288in}}%
\pgfpathlineto{\pgfqpoint{4.511998in}{2.024374in}}%
\pgfpathlineto{\pgfqpoint{4.527172in}{2.030861in}}%
\pgfpathlineto{\pgfqpoint{4.542345in}{2.032496in}}%
\pgfpathlineto{\pgfqpoint{4.557519in}{2.032559in}}%
\pgfpathlineto{\pgfqpoint{4.572693in}{2.022022in}}%
\pgfpathlineto{\pgfqpoint{4.587867in}{2.038647in}}%
\pgfpathlineto{\pgfqpoint{4.587867in}{2.038647in}}%
\pgfusepath{stroke}%
\end{pgfscope}%
\begin{pgfscope}%
\pgfpathrectangle{\pgfqpoint{0.809624in}{0.450000in}}{\pgfqpoint{3.778243in}{2.490000in}}%
\pgfusepath{clip}%
\pgfsetrectcap%
\pgfsetroundjoin%
\pgfsetlinewidth{1.003750pt}%
\definecolor{currentstroke}{rgb}{0.000000,0.500000,0.000000}%
\pgfsetstrokecolor{currentstroke}%
\pgfsetdash{}{0pt}%
\pgfpathmoveto{\pgfqpoint{0.809624in}{2.161509in}}%
\pgfpathlineto{\pgfqpoint{0.824797in}{2.146742in}}%
\pgfpathlineto{\pgfqpoint{0.839971in}{2.126964in}}%
\pgfpathlineto{\pgfqpoint{0.855144in}{2.053647in}}%
\pgfpathlineto{\pgfqpoint{0.870318in}{2.083349in}}%
\pgfpathlineto{\pgfqpoint{0.885492in}{2.010449in}}%
\pgfpathlineto{\pgfqpoint{0.900666in}{2.030147in}}%
\pgfpathlineto{\pgfqpoint{0.915839in}{2.019465in}}%
\pgfpathlineto{\pgfqpoint{0.931013in}{2.013942in}}%
\pgfpathlineto{\pgfqpoint{0.946187in}{2.029468in}}%
\pgfpathlineto{\pgfqpoint{0.961360in}{2.004920in}}%
\pgfpathlineto{\pgfqpoint{0.976534in}{2.010946in}}%
\pgfpathlineto{\pgfqpoint{0.991707in}{1.972437in}}%
\pgfpathlineto{\pgfqpoint{1.006881in}{2.021883in}}%
\pgfpathlineto{\pgfqpoint{1.037228in}{2.025740in}}%
\pgfpathlineto{\pgfqpoint{1.052402in}{2.009838in}}%
\pgfpathlineto{\pgfqpoint{1.067576in}{2.000069in}}%
\pgfpathlineto{\pgfqpoint{1.082750in}{1.970724in}}%
\pgfpathlineto{\pgfqpoint{1.097923in}{1.950327in}}%
\pgfpathlineto{\pgfqpoint{1.113097in}{1.956071in}}%
\pgfpathlineto{\pgfqpoint{1.128270in}{2.005128in}}%
\pgfpathlineto{\pgfqpoint{1.143444in}{1.989380in}}%
\pgfpathlineto{\pgfqpoint{1.158618in}{1.987708in}}%
\pgfpathlineto{\pgfqpoint{1.173791in}{1.995125in}}%
\pgfpathlineto{\pgfqpoint{1.204139in}{2.027621in}}%
\pgfpathlineto{\pgfqpoint{1.219313in}{1.969488in}}%
\pgfpathlineto{\pgfqpoint{1.234486in}{1.983542in}}%
\pgfpathlineto{\pgfqpoint{1.249660in}{2.015615in}}%
\pgfpathlineto{\pgfqpoint{1.264833in}{1.973404in}}%
\pgfpathlineto{\pgfqpoint{1.280007in}{1.981372in}}%
\pgfpathlineto{\pgfqpoint{1.295181in}{1.987909in}}%
\pgfpathlineto{\pgfqpoint{1.310355in}{1.996535in}}%
\pgfpathlineto{\pgfqpoint{1.325528in}{1.995172in}}%
\pgfpathlineto{\pgfqpoint{1.340702in}{2.001393in}}%
\pgfpathlineto{\pgfqpoint{1.355876in}{1.983630in}}%
\pgfpathlineto{\pgfqpoint{1.371049in}{1.979693in}}%
\pgfpathlineto{\pgfqpoint{1.401396in}{2.006505in}}%
\pgfpathlineto{\pgfqpoint{1.416570in}{1.989112in}}%
\pgfpathlineto{\pgfqpoint{1.431744in}{2.009387in}}%
\pgfpathlineto{\pgfqpoint{1.446917in}{2.015629in}}%
\pgfpathlineto{\pgfqpoint{1.462091in}{1.976327in}}%
\pgfpathlineto{\pgfqpoint{1.477265in}{1.998490in}}%
\pgfpathlineto{\pgfqpoint{1.492439in}{1.949723in}}%
\pgfpathlineto{\pgfqpoint{1.507612in}{1.982178in}}%
\pgfpathlineto{\pgfqpoint{1.537960in}{2.016348in}}%
\pgfpathlineto{\pgfqpoint{1.553133in}{2.007278in}}%
\pgfpathlineto{\pgfqpoint{1.568307in}{2.000681in}}%
\pgfpathlineto{\pgfqpoint{1.583481in}{1.989145in}}%
\pgfpathlineto{\pgfqpoint{1.598654in}{1.998719in}}%
\pgfpathlineto{\pgfqpoint{1.613828in}{1.964200in}}%
\pgfpathlineto{\pgfqpoint{1.629001in}{1.994963in}}%
\pgfpathlineto{\pgfqpoint{1.644175in}{2.000318in}}%
\pgfpathlineto{\pgfqpoint{1.659349in}{2.011215in}}%
\pgfpathlineto{\pgfqpoint{1.674522in}{1.985034in}}%
\pgfpathlineto{\pgfqpoint{1.689696in}{2.008863in}}%
\pgfpathlineto{\pgfqpoint{1.704870in}{1.967036in}}%
\pgfpathlineto{\pgfqpoint{1.720043in}{2.011967in}}%
\pgfpathlineto{\pgfqpoint{1.735217in}{2.018067in}}%
\pgfpathlineto{\pgfqpoint{1.750391in}{1.985383in}}%
\pgfpathlineto{\pgfqpoint{1.765564in}{1.975756in}}%
\pgfpathlineto{\pgfqpoint{1.780738in}{1.994204in}}%
\pgfpathlineto{\pgfqpoint{1.795912in}{1.982514in}}%
\pgfpathlineto{\pgfqpoint{1.811085in}{2.014480in}}%
\pgfpathlineto{\pgfqpoint{1.826259in}{1.981527in}}%
\pgfpathlineto{\pgfqpoint{1.841433in}{2.010933in}}%
\pgfpathlineto{\pgfqpoint{1.856606in}{1.966646in}}%
\pgfpathlineto{\pgfqpoint{1.871780in}{1.977321in}}%
\pgfpathlineto{\pgfqpoint{1.886954in}{1.968467in}}%
\pgfpathlineto{\pgfqpoint{1.902127in}{1.978913in}}%
\pgfpathlineto{\pgfqpoint{1.917301in}{2.002367in}}%
\pgfpathlineto{\pgfqpoint{1.932475in}{2.003099in}}%
\pgfpathlineto{\pgfqpoint{1.947648in}{1.987620in}}%
\pgfpathlineto{\pgfqpoint{1.962822in}{1.983744in}}%
\pgfpathlineto{\pgfqpoint{1.977996in}{1.997012in}}%
\pgfpathlineto{\pgfqpoint{1.993170in}{1.962749in}}%
\pgfpathlineto{\pgfqpoint{2.008343in}{1.968487in}}%
\pgfpathlineto{\pgfqpoint{2.023517in}{2.007862in}}%
\pgfpathlineto{\pgfqpoint{2.038690in}{2.029502in}}%
\pgfpathlineto{\pgfqpoint{2.053864in}{2.001238in}}%
\pgfpathlineto{\pgfqpoint{2.069038in}{2.015494in}}%
\pgfpathlineto{\pgfqpoint{2.084211in}{1.999108in}}%
\pgfpathlineto{\pgfqpoint{2.099385in}{1.990099in}}%
\pgfpathlineto{\pgfqpoint{2.114559in}{1.962259in}}%
\pgfpathlineto{\pgfqpoint{2.129733in}{1.997725in}}%
\pgfpathlineto{\pgfqpoint{2.144906in}{2.002676in}}%
\pgfpathlineto{\pgfqpoint{2.160080in}{1.989475in}}%
\pgfpathlineto{\pgfqpoint{2.175254in}{2.001077in}}%
\pgfpathlineto{\pgfqpoint{2.190427in}{1.987405in}}%
\pgfpathlineto{\pgfqpoint{2.205601in}{2.008360in}}%
\pgfpathlineto{\pgfqpoint{2.220775in}{1.983314in}}%
\pgfpathlineto{\pgfqpoint{2.235948in}{1.953055in}}%
\pgfpathlineto{\pgfqpoint{2.251122in}{1.994251in}}%
\pgfpathlineto{\pgfqpoint{2.266296in}{2.007150in}}%
\pgfpathlineto{\pgfqpoint{2.281469in}{2.008460in}}%
\pgfpathlineto{\pgfqpoint{2.296643in}{1.974110in}}%
\pgfpathlineto{\pgfqpoint{2.311816in}{1.987123in}}%
\pgfpathlineto{\pgfqpoint{2.326990in}{1.996663in}}%
\pgfpathlineto{\pgfqpoint{2.342164in}{1.999955in}}%
\pgfpathlineto{\pgfqpoint{2.357337in}{2.010758in}}%
\pgfpathlineto{\pgfqpoint{2.372511in}{2.000613in}}%
\pgfpathlineto{\pgfqpoint{2.387685in}{2.005954in}}%
\pgfpathlineto{\pgfqpoint{2.402858in}{1.988877in}}%
\pgfpathlineto{\pgfqpoint{2.418032in}{2.003885in}}%
\pgfpathlineto{\pgfqpoint{2.433206in}{1.975077in}}%
\pgfpathlineto{\pgfqpoint{2.448379in}{2.008877in}}%
\pgfpathlineto{\pgfqpoint{2.463553in}{1.998343in}}%
\pgfpathlineto{\pgfqpoint{2.478727in}{1.996146in}}%
\pgfpathlineto{\pgfqpoint{2.493900in}{1.936078in}}%
\pgfpathlineto{\pgfqpoint{2.509074in}{1.980291in}}%
\pgfpathlineto{\pgfqpoint{2.524248in}{1.964093in}}%
\pgfpathlineto{\pgfqpoint{2.539421in}{1.992216in}}%
\pgfpathlineto{\pgfqpoint{2.554595in}{1.992336in}}%
\pgfpathlineto{\pgfqpoint{2.569769in}{1.975621in}}%
\pgfpathlineto{\pgfqpoint{2.584942in}{1.969629in}}%
\pgfpathlineto{\pgfqpoint{2.600116in}{1.987560in}}%
\pgfpathlineto{\pgfqpoint{2.615290in}{1.975554in}}%
\pgfpathlineto{\pgfqpoint{2.630463in}{2.000183in}}%
\pgfpathlineto{\pgfqpoint{2.645637in}{2.004631in}}%
\pgfpathlineto{\pgfqpoint{2.660811in}{1.980600in}}%
\pgfpathlineto{\pgfqpoint{2.675984in}{2.007103in}}%
\pgfpathlineto{\pgfqpoint{2.691158in}{1.985504in}}%
\pgfpathlineto{\pgfqpoint{2.706332in}{1.971980in}}%
\pgfpathlineto{\pgfqpoint{2.721506in}{2.012753in}}%
\pgfpathlineto{\pgfqpoint{2.736679in}{2.007311in}}%
\pgfpathlineto{\pgfqpoint{2.751853in}{2.004920in}}%
\pgfpathlineto{\pgfqpoint{2.767027in}{1.962581in}}%
\pgfpathlineto{\pgfqpoint{2.782200in}{2.006028in}}%
\pgfpathlineto{\pgfqpoint{2.797374in}{1.996986in}}%
\pgfpathlineto{\pgfqpoint{2.812547in}{2.017805in}}%
\pgfpathlineto{\pgfqpoint{2.827721in}{2.014500in}}%
\pgfpathlineto{\pgfqpoint{2.842895in}{1.967969in}}%
\pgfpathlineto{\pgfqpoint{2.858068in}{2.011262in}}%
\pgfpathlineto{\pgfqpoint{2.873242in}{1.972195in}}%
\pgfpathlineto{\pgfqpoint{2.888416in}{1.980768in}}%
\pgfpathlineto{\pgfqpoint{2.903589in}{1.993069in}}%
\pgfpathlineto{\pgfqpoint{2.918763in}{1.973734in}}%
\pgfpathlineto{\pgfqpoint{2.933937in}{1.981251in}}%
\pgfpathlineto{\pgfqpoint{2.949110in}{1.993767in}}%
\pgfpathlineto{\pgfqpoint{2.964284in}{1.979659in}}%
\pgfpathlineto{\pgfqpoint{2.979458in}{2.004167in}}%
\pgfpathlineto{\pgfqpoint{2.994632in}{2.006868in}}%
\pgfpathlineto{\pgfqpoint{3.024979in}{1.966968in}}%
\pgfpathlineto{\pgfqpoint{3.040153in}{1.988924in}}%
\pgfpathlineto{\pgfqpoint{3.055326in}{1.959511in}}%
\pgfpathlineto{\pgfqpoint{3.070500in}{1.977671in}}%
\pgfpathlineto{\pgfqpoint{3.085673in}{1.958920in}}%
\pgfpathlineto{\pgfqpoint{3.100847in}{1.989669in}}%
\pgfpathlineto{\pgfqpoint{3.116021in}{1.961614in}}%
\pgfpathlineto{\pgfqpoint{3.131194in}{2.003496in}}%
\pgfpathlineto{\pgfqpoint{3.146368in}{1.992074in}}%
\pgfpathlineto{\pgfqpoint{3.161542in}{1.997221in}}%
\pgfpathlineto{\pgfqpoint{3.176715in}{1.970529in}}%
\pgfpathlineto{\pgfqpoint{3.191889in}{1.998074in}}%
\pgfpathlineto{\pgfqpoint{3.207063in}{2.022172in}}%
\pgfpathlineto{\pgfqpoint{3.222236in}{2.010476in}}%
\pgfpathlineto{\pgfqpoint{3.237410in}{2.004402in}}%
\pgfpathlineto{\pgfqpoint{3.252584in}{1.969421in}}%
\pgfpathlineto{\pgfqpoint{3.267757in}{2.004046in}}%
\pgfpathlineto{\pgfqpoint{3.282931in}{2.012061in}}%
\pgfpathlineto{\pgfqpoint{3.298105in}{1.997510in}}%
\pgfpathlineto{\pgfqpoint{3.313278in}{1.962642in}}%
\pgfpathlineto{\pgfqpoint{3.328452in}{2.001231in}}%
\pgfpathlineto{\pgfqpoint{3.343626in}{1.998786in}}%
\pgfpathlineto{\pgfqpoint{3.358799in}{1.965893in}}%
\pgfpathlineto{\pgfqpoint{3.373973in}{1.993290in}}%
\pgfpathlineto{\pgfqpoint{3.389147in}{2.007634in}}%
\pgfpathlineto{\pgfqpoint{3.404321in}{1.977899in}}%
\pgfpathlineto{\pgfqpoint{3.419494in}{2.001547in}}%
\pgfpathlineto{\pgfqpoint{3.434668in}{1.960458in}}%
\pgfpathlineto{\pgfqpoint{3.449842in}{1.968614in}}%
\pgfpathlineto{\pgfqpoint{3.465015in}{1.958953in}}%
\pgfpathlineto{\pgfqpoint{3.480189in}{1.942877in}}%
\pgfpathlineto{\pgfqpoint{3.495363in}{1.985188in}}%
\pgfpathlineto{\pgfqpoint{3.510536in}{1.983589in}}%
\pgfpathlineto{\pgfqpoint{3.525710in}{1.957146in}}%
\pgfpathlineto{\pgfqpoint{3.540884in}{2.011275in}}%
\pgfpathlineto{\pgfqpoint{3.556057in}{1.975050in}}%
\pgfpathlineto{\pgfqpoint{3.571231in}{2.014843in}}%
\pgfpathlineto{\pgfqpoint{3.586404in}{1.980183in}}%
\pgfpathlineto{\pgfqpoint{3.601578in}{1.963314in}}%
\pgfpathlineto{\pgfqpoint{3.616752in}{1.997133in}}%
\pgfpathlineto{\pgfqpoint{3.631925in}{1.996401in}}%
\pgfpathlineto{\pgfqpoint{3.647099in}{2.019277in}}%
\pgfpathlineto{\pgfqpoint{3.662273in}{2.011866in}}%
\pgfpathlineto{\pgfqpoint{3.677446in}{1.974426in}}%
\pgfpathlineto{\pgfqpoint{3.692620in}{1.991268in}}%
\pgfpathlineto{\pgfqpoint{3.707794in}{1.969756in}}%
\pgfpathlineto{\pgfqpoint{3.722967in}{1.993700in}}%
\pgfpathlineto{\pgfqpoint{3.738141in}{2.000620in}}%
\pgfpathlineto{\pgfqpoint{3.753315in}{1.978510in}}%
\pgfpathlineto{\pgfqpoint{3.768488in}{2.013788in}}%
\pgfpathlineto{\pgfqpoint{3.783662in}{1.995864in}}%
\pgfpathlineto{\pgfqpoint{3.798836in}{2.006821in}}%
\pgfpathlineto{\pgfqpoint{3.829183in}{1.964415in}}%
\pgfpathlineto{\pgfqpoint{3.844357in}{2.002044in}}%
\pgfpathlineto{\pgfqpoint{3.859530in}{2.001352in}}%
\pgfpathlineto{\pgfqpoint{3.874704in}{1.981453in}}%
\pgfpathlineto{\pgfqpoint{3.889878in}{1.995843in}}%
\pgfpathlineto{\pgfqpoint{3.905052in}{1.993754in}}%
\pgfpathlineto{\pgfqpoint{3.920225in}{1.979935in}}%
\pgfpathlineto{\pgfqpoint{3.935399in}{1.976387in}}%
\pgfpathlineto{\pgfqpoint{3.950573in}{1.953431in}}%
\pgfpathlineto{\pgfqpoint{3.965746in}{1.995272in}}%
\pgfpathlineto{\pgfqpoint{3.980920in}{1.998880in}}%
\pgfpathlineto{\pgfqpoint{3.996094in}{1.985061in}}%
\pgfpathlineto{\pgfqpoint{4.011267in}{1.992827in}}%
\pgfpathlineto{\pgfqpoint{4.026441in}{1.992323in}}%
\pgfpathlineto{\pgfqpoint{4.041614in}{1.995985in}}%
\pgfpathlineto{\pgfqpoint{4.056788in}{2.009253in}}%
\pgfpathlineto{\pgfqpoint{4.071962in}{1.977543in}}%
\pgfpathlineto{\pgfqpoint{4.087136in}{1.967123in}}%
\pgfpathlineto{\pgfqpoint{4.102309in}{2.007291in}}%
\pgfpathlineto{\pgfqpoint{4.117483in}{1.949259in}}%
\pgfpathlineto{\pgfqpoint{4.132657in}{1.963078in}}%
\pgfpathlineto{\pgfqpoint{4.147830in}{1.996320in}}%
\pgfpathlineto{\pgfqpoint{4.163004in}{1.954513in}}%
\pgfpathlineto{\pgfqpoint{4.178178in}{2.002306in}}%
\pgfpathlineto{\pgfqpoint{4.193351in}{1.983401in}}%
\pgfpathlineto{\pgfqpoint{4.208525in}{1.993190in}}%
\pgfpathlineto{\pgfqpoint{4.223699in}{1.982252in}}%
\pgfpathlineto{\pgfqpoint{4.238872in}{1.985887in}}%
\pgfpathlineto{\pgfqpoint{4.254046in}{1.994003in}}%
\pgfpathlineto{\pgfqpoint{4.269220in}{1.977684in}}%
\pgfpathlineto{\pgfqpoint{4.284393in}{1.935836in}}%
\pgfpathlineto{\pgfqpoint{4.299567in}{1.995279in}}%
\pgfpathlineto{\pgfqpoint{4.314741in}{1.987298in}}%
\pgfpathlineto{\pgfqpoint{4.345088in}{2.012397in}}%
\pgfpathlineto{\pgfqpoint{4.360262in}{2.003932in}}%
\pgfpathlineto{\pgfqpoint{4.375435in}{1.991718in}}%
\pgfpathlineto{\pgfqpoint{4.390609in}{2.018497in}}%
\pgfpathlineto{\pgfqpoint{4.405782in}{1.975998in}}%
\pgfpathlineto{\pgfqpoint{4.420956in}{1.990214in}}%
\pgfpathlineto{\pgfqpoint{4.436130in}{2.002851in}}%
\pgfpathlineto{\pgfqpoint{4.451303in}{1.992874in}}%
\pgfpathlineto{\pgfqpoint{4.466477in}{1.991544in}}%
\pgfpathlineto{\pgfqpoint{4.481651in}{1.977744in}}%
\pgfpathlineto{\pgfqpoint{4.496824in}{1.946095in}}%
\pgfpathlineto{\pgfqpoint{4.511998in}{1.998034in}}%
\pgfpathlineto{\pgfqpoint{4.527172in}{2.004597in}}%
\pgfpathlineto{\pgfqpoint{4.542345in}{2.017584in}}%
\pgfpathlineto{\pgfqpoint{4.557519in}{2.001776in}}%
\pgfpathlineto{\pgfqpoint{4.572693in}{1.991799in}}%
\pgfpathlineto{\pgfqpoint{4.587867in}{2.018074in}}%
\pgfpathlineto{\pgfqpoint{4.587867in}{2.018074in}}%
\pgfusepath{stroke}%
\end{pgfscope}%
\begin{pgfscope}%
\pgfsetrectcap%
\pgfsetmiterjoin%
\pgfsetlinewidth{0.401500pt}%
\definecolor{currentstroke}{rgb}{0.000000,0.000000,0.000000}%
\pgfsetstrokecolor{currentstroke}%
\pgfsetdash{}{0pt}%
\pgfpathmoveto{\pgfqpoint{0.809624in}{0.450000in}}%
\pgfpathlineto{\pgfqpoint{0.809624in}{2.940000in}}%
\pgfusepath{stroke}%
\end{pgfscope}%
\begin{pgfscope}%
\pgfsetrectcap%
\pgfsetmiterjoin%
\pgfsetlinewidth{0.401500pt}%
\definecolor{currentstroke}{rgb}{0.000000,0.000000,0.000000}%
\pgfsetstrokecolor{currentstroke}%
\pgfsetdash{}{0pt}%
\pgfpathmoveto{\pgfqpoint{4.587867in}{0.450000in}}%
\pgfpathlineto{\pgfqpoint{4.587867in}{2.940000in}}%
\pgfusepath{stroke}%
\end{pgfscope}%
\begin{pgfscope}%
\pgfsetrectcap%
\pgfsetmiterjoin%
\pgfsetlinewidth{0.401500pt}%
\definecolor{currentstroke}{rgb}{0.000000,0.000000,0.000000}%
\pgfsetstrokecolor{currentstroke}%
\pgfsetdash{}{0pt}%
\pgfpathmoveto{\pgfqpoint{0.809624in}{0.450000in}}%
\pgfpathlineto{\pgfqpoint{4.587867in}{0.450000in}}%
\pgfusepath{stroke}%
\end{pgfscope}%
\begin{pgfscope}%
\pgfsetrectcap%
\pgfsetmiterjoin%
\pgfsetlinewidth{0.401500pt}%
\definecolor{currentstroke}{rgb}{0.000000,0.000000,0.000000}%
\pgfsetstrokecolor{currentstroke}%
\pgfsetdash{}{0pt}%
\pgfpathmoveto{\pgfqpoint{0.809624in}{2.940000in}}%
\pgfpathlineto{\pgfqpoint{4.587867in}{2.940000in}}%
\pgfusepath{stroke}%
\end{pgfscope}%
\begin{pgfscope}%
\pgfsetbuttcap%
\pgfsetmiterjoin%
\definecolor{currentfill}{rgb}{1.000000,1.000000,1.000000}%
\pgfsetfillcolor{currentfill}%
\pgfsetfillopacity{0.800000}%
\pgfsetlinewidth{1.003750pt}%
\definecolor{currentstroke}{rgb}{0.800000,0.800000,0.800000}%
\pgfsetstrokecolor{currentstroke}%
\pgfsetstrokeopacity{0.800000}%
\pgfsetdash{}{0pt}%
\pgfpathmoveto{\pgfqpoint{3.717110in}{2.247871in}}%
\pgfpathlineto{\pgfqpoint{4.490644in}{2.247871in}}%
\pgfpathquadraticcurveto{\pgfqpoint{4.518422in}{2.247871in}}{\pgfqpoint{4.518422in}{2.275648in}}%
\pgfpathlineto{\pgfqpoint{4.518422in}{2.842778in}}%
\pgfpathquadraticcurveto{\pgfqpoint{4.518422in}{2.870556in}}{\pgfqpoint{4.490644in}{2.870556in}}%
\pgfpathlineto{\pgfqpoint{3.717110in}{2.870556in}}%
\pgfpathquadraticcurveto{\pgfqpoint{3.689332in}{2.870556in}}{\pgfqpoint{3.689332in}{2.842778in}}%
\pgfpathlineto{\pgfqpoint{3.689332in}{2.275648in}}%
\pgfpathquadraticcurveto{\pgfqpoint{3.689332in}{2.247871in}}{\pgfqpoint{3.717110in}{2.247871in}}%
\pgfpathclose%
\pgfusepath{stroke,fill}%
\end{pgfscope}%
\begin{pgfscope}%
\pgfsetrectcap%
\pgfsetroundjoin%
\pgfsetlinewidth{1.003750pt}%
\definecolor{currentstroke}{rgb}{0.000000,0.000000,1.000000}%
\pgfsetstrokecolor{currentstroke}%
\pgfsetdash{}{0pt}%
\pgfpathmoveto{\pgfqpoint{3.744888in}{2.766389in}}%
\pgfpathlineto{\pgfqpoint{4.022666in}{2.766389in}}%
\pgfusepath{stroke}%
\end{pgfscope}%
\begin{pgfscope}%
\definecolor{textcolor}{rgb}{0.000000,0.000000,0.000000}%
\pgfsetstrokecolor{textcolor}%
\pgfsetfillcolor{textcolor}%
\pgftext[x=4.133777in,y=2.717778in,left,base]{\color{textcolor}\rmfamily\fontsize{10.000000}{12.000000}\selectfont worst}%
\end{pgfscope}%
\begin{pgfscope}%
\pgfsetrectcap%
\pgfsetroundjoin%
\pgfsetlinewidth{1.003750pt}%
\definecolor{currentstroke}{rgb}{0.411765,0.411765,0.411765}%
\pgfsetstrokecolor{currentstroke}%
\pgfsetdash{}{0pt}%
\pgfpathmoveto{\pgfqpoint{3.744888in}{2.572716in}}%
\pgfpathlineto{\pgfqpoint{4.022666in}{2.572716in}}%
\pgfusepath{stroke}%
\end{pgfscope}%
\begin{pgfscope}%
\definecolor{textcolor}{rgb}{0.000000,0.000000,0.000000}%
\pgfsetstrokecolor{textcolor}%
\pgfsetfillcolor{textcolor}%
\pgftext[x=4.133777in,y=2.524105in,left,base]{\color{textcolor}\rmfamily\fontsize{10.000000}{12.000000}\selectfont avg}%
\end{pgfscope}%
\begin{pgfscope}%
\pgfsetrectcap%
\pgfsetroundjoin%
\pgfsetlinewidth{1.003750pt}%
\definecolor{currentstroke}{rgb}{0.000000,0.500000,0.000000}%
\pgfsetstrokecolor{currentstroke}%
\pgfsetdash{}{0pt}%
\pgfpathmoveto{\pgfqpoint{3.744888in}{2.379043in}}%
\pgfpathlineto{\pgfqpoint{4.022666in}{2.379043in}}%
\pgfusepath{stroke}%
\end{pgfscope}%
\begin{pgfscope}%
\definecolor{textcolor}{rgb}{0.000000,0.000000,0.000000}%
\pgfsetstrokecolor{textcolor}%
\pgfsetfillcolor{textcolor}%
\pgftext[x=4.133777in,y=2.330432in,left,base]{\color{textcolor}\rmfamily\fontsize{10.000000}{12.000000}\selectfont best}%
\end{pgfscope}%
\end{pgfpicture}%
\makeatother%
\endgroup%

  \end{center}
  \caption{Przebieg algorytmu dla $P_m = 0.05$.}
\end{figure}
\begin{table}[H]
  \centering
  \caption{Jakaś tabela}
  \begin{tabular}{| c | c | c | c | c |}
    \hline
    $P_m$ & best & worst & avg & std \\
    \hline
    0 & 75707 & 372449 & 79931.20 & 3131.11 \\
    0.01 & 135494 & 378228 & 139589.90 & 6256.59 \\
    \hline
  \end{tabular}
\end{table}

\begin{thebibliography}{9}
  \bibitem{tsplib95}
  Gerhard Reinelt: Dokumenacja TSPLIB 95. Universit{\"a}t Heidelberg, Institut f{\"u}ur Angewandte Mathematik,\\
  \texttt{http://comopt.ifi.uni-heidelberg.de/software/TSPLIB95\\/tsp95.pdf}
\end{thebibliography}
 
\end{document}
