\documentclass[12pt,a4paper]{article}
\usepackage[T1]{fontenc}
\usepackage[utf8]{inputenc}
\usepackage{polski}
\usepackage{minted}

\renewcommand\refname{Bibliografia}

\begin{document}

\begin{titlepage}
  \centering
  \vspace*{2cm}
  {\scshape\large Politechnika Wrocławska \par}
  \vspace{1cm}
  {\itshape\large Sztuczna Inteligencja i Inżynieria Wiedzy\par}
  \vspace{1.5cm}
  {\LARGE\bfseries Implementacja i badanie algorytmów ewolucyjnych na bazie problemu komiwojażera\par}
  \vspace{1cm}
  {\Large Sprawozdanie nr 1\par}
  \vfill
  {\large Karol Belina, 242499\par grupa śr. 11:15}
  \vfill
  {\large \today\par}
\end{titlepage}

\tableofcontents

\section*{Streszczenie}

W sprawozdaniu opisano

\section{Wstęp}

Problem komiwojażera (ang. Travelling salesperson problem) jest zagadnieniem optymalizacyjnym polegającym na znalezieniu mimalnego cyklu w grafie ważonym obejmującego wszystkie wierzchołki. Dane wejściowe to lista wierzchołków wraz z ich współrzędnymi. Do rozwiązania tego problemu można użyć różnych metod m.in. algorytmu zachłannego, algorytmu wspinaczkowego (ang. Hill climbing algorithm) lub algorytmu genetycznego.

\section{Implementacja}

\subsection{Format plików}

Dane wejściowe pochodzą z biblioteki TSPLIB 95 \cite{tsplib95}. W programie zaimplementowano uproszczony parser plików TSPLIB skupiający się jedynie na Symetrycznym problemie komiwojażera w przestrzeni dwuwymiarowej. Parser jest w stanie wykryć dwa typy wag: \verb|EUC_2D| oraz \verb|GEO| i użyć odpowiedniej funkcji odległości do obliczenia macierzy odległości. Macierz odległości zwalnia program z potrzeby przechowywania w pamięci współrzędnych. Dodatkowo, wszystkie odległości liczone są tylko raz.

\subsection{Model}

Podczas tworzenia modelu szczególną uwagę zwrócono na jego modularność. Zauważono, że do zdefiniowania problemu wymagana jest jedynie wiedza o typie rozwiązania i miary jakości tego rozwiązania, jak również definicja funkcji przystosowania mapująca rozwiązania na miarę jakości. Wszystkie te wymagania zawarto w implementacji jako jedna wspólna cecha \texttt{Problem} (ang. trait).
\begin{listing}[H]
  \begin{minted}[breaklines,linenos,frame=single]{rust}
pub trait Problem {
  type Solution: ToOwned + Clone;
  type Measure: Clone;

  fn fitness(&self, solution: &Self::Solution) -> Self::Measure;
}
  \end{minted}
  \caption{Definicja cechy \texttt{Problem}.}
\end{listing}

\begin{thebibliography}{9}
  \bibitem{tsplib95}
  Gerhard Reinelt: Dokumenacja TSPLIB 95.\\
  Universit{\"a}t Heidelberg, Institut f{\"u}ur Angewandte Mathematik,
  \texttt{http://comopt.ifi.uni-heidelberg.de/software/TSPLIB95\\/tsp95.pdf}
\end{thebibliography}
 
\end{document}
